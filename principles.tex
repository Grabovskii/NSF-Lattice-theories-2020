\section{Preliminaries}
\subsection{The main principles of classical lattice models}
Let $\Omega$ be a finite set (the set of \emph{microstates}), let $\hh : \Omega \rightarrow \mathbb R$ be a \emph{hamiltonian}, a specifically chosen random variable. Let $\mathcal M(\Omega)$ be the space of probability measures on $\Omega$. In this theory, the expectation of a random variable $f : \Omega \rightarrow \mathbb R$ with respect to a measure $\mu$ (if not implicitly understood; it's also called the \emph{thermal average}) is denoted as
\[
\left\langle f \right\rangle_{\mu} := \mathbb E_{\mu} f = \int_{\Omega} f d\mu.
\]
One of the first questions in statistical mechanics is devoted to the choice of the right measure $\mu$. The choice is governed using Shannon's entropy $S : \mathcal M(\Omega) \rightarrow \mathbb R$, defined as $S(\mu) := -\int_{\Omega}x\log x d\mu(x)$ (there's a way to understand why $S$ has this form; see \cite{friedli}). \emph{The maximum entropy principle says}: for a given model of statistical mechanics, choose $\mu$ that maximizes $S$. For example, if there's no other information avaiable about the system, then the measure that maximizes $S$ is the uniform distribution. There are other two typical situations:
\begin{enumerate}[(a)]
\item If we know that $\left\langle \mathcal H\right\rangle = U$ for some fixed $U$ (the \emph{internal energy} of the system), then the measure that maximizes $S$ is 
\[
\mu(\omega) = \frac{e^{-\beta H(\omega)}}{Z}, \ \ Z:= \sum_{\omega \in \Omega} e^{-\beta \mathcal H(\omega)} 
\]
which is the \emph{Gibbs measure}. This corresponds to the situation when we know the system exchanges its energy with some external thermal reservoir. The result can be obtained using Lagrange multipliers; the parameter $\beta$, called the \emph{inverse temperature}, is uniquely determined by $U$ (and vice versa: $U$ is uniquely determined by $\beta$). In the theory, $\beta = (kT)^{-1}$, where $k$ is the Boltzmann constant and $T$ is the temperature of the system.
\item If we know additionally that the system exchanges its particles $\mathcal N$ with the external environment, and that the expected value of the particles if $\left\langle \mathcal N \right\rangle = N$, then we obtain a \emph{grand canonical Gibbs distribution} via a similar procedure. It's given by
\[
\mu(\omega) := \frac{e^{-\beta(\hh(\omega)-\mu N)}}{Z}, \ \ Z := \sum_{N} e^{\beta \mu N} \sum_{\omega \in \Omega} e^{-\beta \hh (\omega)}.
\]
The parameter $\mu$ is identified with the so-called \emph{chemical potential}.
\end{enumerate}

\subsection{The thermodynamic limit}

\subsection{Quantum lattices}
\subsubsection{A general set-up}
I will follow closely the treatment in \cite{israel}. Again, we have a lattice $X \subset \mathbb Z^n$, but to each site $i \in X$ we attach a copy of a finite-dimensional Hilbert space $H_i$. To a finite $X$ we attach the tensor product $H_X := \otimes_{i \in X} H_i$. %For example, in spin-$1/2$ systems each Hilbert space is two-dimensional, with an orthonormal basis corresponding to spins up and down.

For infinite lattices, the author of \cite{israel} suggests proceeding as follows. Let $\mathfrak{A}_X := \End(H_X)$, and for any two finite subsets $X \subseteq Y \subset \mathbb Z^n$, let $\imath : \mathfrak{A}_X \rightarrow \mathfrak{A}_Y$ be the inclusion that sends $A$ to $A \otimes 1$ (where $1$ is viewed as an endomorphism of $\mathfrak{A}_{Y\setminus X}$). For an infinite $\Lambda \subseteq \mathbb Z^n$, the family of all its finite subsets with the inclusions form a direct system. Let $\mathfrak A_{\Lambda} := \varinjlim \mathfrak{A}_X$ be the direct limit taken in the category of $C^*$-algebras over all finite subsets of $\Lambda$. In the literature, this algebra is known as an AF (approximately finite-dimensional) $C^*$-algebra. The first reference in this theory goes back to Bratteli \cite{bratteli}. See the next subsection for an elaboration on the inductive limit.

Further, for a finite $\Lambda \subset \mathbb Z^n$ and a Hamiltonian $\mathcal H_{\Lambda}$, the partition function is defined as
\[
Z = \tr_{\Lambda} e^{-\beta \mathcal H_{\Lambda}}
\]
and the expectation of an observable $A \in \mathfrak A_{\Lambda}$ is 
\[
\left\langle A \right\rangle_{\Lambda} := Z^{-1} \tr_{\Lambda} (A e^{-\mathcal H_{\Lambda}}).
\]
The trace in these formulas is normalized: it's $1/d$ of the usual trace, where $d$ is the dimension of the Hilbert space at one site. An interesting consequence of such normalization is that $\tr$ extends than to a norm-one linear functional on the whole $\mathfrak A:= \mathfrak A_{\mathbb Z^n}$ (see \cite{israel}). The Hamiltonian they choose is given by
\[
\mathcal H_{\Lambda} = \sum_{X \subseteq \Lambda} \Phi(X),
\]
where $\Phi$ is a so-called \emph{interaction}: it's a function from the non-empty finite subsets of $\mathbb Z^n$ to self-adjoint operators on them, such that $\Phi(X+i) = \Phi(X)$ for any $i \in \mathbb Z^n$ (i.e., it's translational invariant).

The pressure for a finite region $\Lambda$ in the quantum lattice system is given by
\[
P_{\Lambda}(\Phi) := |\Lambda|^{-1} \ln \tr e^{-H_{\Lambda}}.
\]
One can show that the limit in the sense of van Hove of $P_{\Lambda}$ does exist in the quantum setting as well (\cite{israel}).

%\begin{fur}
%Not yet have I delved into the properties of AF $C^*$-algebras. How is the norm defined there at least? Not yet clear.
%\end{fur}
%
%\begin{fur}
%It's said in \cite{bratteli} that one might refer to \cite{haag} for limits of lattices in case the number of states is infinite.
%\end{fur}
\subsubsection{The inductive limit in more detail}
For two finite subsets $X \subseteq Y \subset \mathbb Z^n$, the inclusion $\imath : \mathfrak A_{X} \rightarrow \mathfrak A_{Y}$ that sends $A$ to $A \otimes 1$ is injective; therefore, whenever $X \subseteq Y$, we can view $\mathfrak A_{X}$ as a subalgebra of $\mathfrak A_{Y}$. Hence we can take a union of all such subalgebras coming from finite subsets of $\Lambda \subseteq \mathbb Z^n$, and then, to be safe and ensure it's a Banach space, take the closure. So, one can identify (see a proposition below for a rigorous proof)
\[
\mathfrak A_{\Lambda} = \varinjlim \mathfrak A_{X} = \cl \left(\bigcup_{X \subset \Lambda, \ |X| < \infty} \mathfrak A_{X}\right)/_{u \sim u \otimes 1}
\]
From this point of view, it's easy to understand what the norm is. For $A$ from the dense subspace (the union itself), we just set $\|A\|_{\Lambda} := \|A\|_{X}$ if $A \in \mathfrak A_{X}$. The norm extends to the closure by the very process of completeness: for $A \in \mathfrak A_{\Lambda}$, we choose a sequence $A_n \in \mathfrak A_{X_n}$ and then set $\|A\|_{\Lambda} := \lim_{n \rightarrow \infty} \|A_n\|$.

From Appendix on $C^*$ algebras, we see that Gelfand-Naymark theorem ensures there is a Hilbert space $H$ such that $\mathfrak A_{\Lambda} \cong \End(H)$.
\begin{fur}
I can elaborate on the construction of this Hilbert space. It's more or less constructive and relies on finding pure states. In particular, it would be interesting to see how this $H$ is related to the infinite tensor product $\otimes_{i \in \Lambda} H_i$: what exactly goes wrong?
\end{fur}
\begin{proposition}
In the above set-up, we indeed have $\varinjlim \mathfrak A_{X} = \cl \left(\bigcup_{X \subset \Lambda, \ |X| < \infty} \mathfrak A_{X}\right)/_{u \sim u \otimes 1}$ (isometrically and preserving the $\ast$-structure).
\end{proposition}
\begin{proof}
Denote $\mathfrak A^\prime := \left(\bigcup_{X \subset \Lambda, \ |X|<\infty} \mathfrak A_X\right)/_{u \sim u \otimes 1}$.
So, we choose morphisms in the category of unital $C^*$-algebras as bounded unital $\ast$-homorphisms with norm less then or equal to one\footnote{Otherwise I don't think there's a way to prove that the map induced on the diagram of the injective limit is a bounded operator}. To prove the statement, all we need to show is that for a unital $C^*$-algebra $A$ and a bunch of morphisms $\alpha_X : \mathfrak A_X \rightarrow \mathfrak A_Y$ where $X \subseteq Y$ and such that $\alpha_X(u) = \alpha_Y(u \otimes 1)$ (but remember that $u$ is identified with $u \otimes 1$ in the union), there's a unique morphism $\alpha : \cl \mathfrak A^\prime \rightarrow A$. In the language of diagrams, this is saying that
\[
\xymatrix{
& & A & & \\
& &  & & \\
& & \cl \mathfrak A^\prime\ar@{-->}[uu]_{\exists ! \alpha} & &\\
\mathfrak A_X\ar[rrrr]^{u \mapsto u \otimes 1} \ar[rru]^{\imath_X} \ar[rruuu]^{\alpha_X} & & & & \mathfrak A_Y\ar[lluuu]_{\alpha_Y} \ar[llu]_{\imath_Y}
}
\]
Once $\alpha$ is defined on $\mathfrak A^\prime$ with all the mentioned properties, it automatically extends to the closure. So, for $u \in \mathfrak A_X$ we set $\alpha(u) := \alpha_X(u)$. This is well defined, for $u$ is identified with $u \otimes 1$ in the union. We get automatically that $\alpha$ is a unital $\ast$-homorphism since all $\alpha_X$'s are. It's norm is bounded by $1$, for $\|\alpha(u)\| \leq \|\alpha_X\|\|u\| \leq \|u\|$. Thus $\alpha$ is a morphism in the corresponding category.
\end{proof}

\subsection{Relation between classical and quantum lattices}
I follow \cite{israel} with some minor modifications more appealing to my taste. Let $\Omega_0$ be a finite set of microstates at one site, and let $H_0$ be a Hilbert space of dimension equal to $|\Omega|$ (which is assigned to one site as well). Let $C(\Omega)$ be the space of observables on $\Omega$. Choose an orthonormal basis $e_{\mu}$ of $H_0$ labeled my microstates $\mu \in \Omega_0$. Then we have an injection $\imath : C(\Omega_0) \rightarrow \End(H_0)$ given by
\[
[\imath(f)](e_{\mu}) := e_{f(\mu)}.
\]
In other words, the classical observables are embedded into the quantum observables as diagonal matrices.

\subsection{Continuous spins: general principles}\label{ss:gen_princ}
I follow closely Section 6.10 of \cite{friedli}. In case the space of states $\Omega_0$ at a single site is non-compact, the existence of Gibbs measures is no longer guaranteed. For $\Omega_0$ a topological space, one defines the following ingredients. Let $\mathcal B_0$ be the Borel $\sigma$-algebra on $\Omega_0$. For a finite lattice $\Lambda \subset \mathbb Z^n$, we supply the space of states with the $\sigma$-algebra $\mathcal B_{\Lambda} := \bigotimes_{i \in \Lambda} \mathcal B_0$. The natural projections $\pi_{\Lambda} : \Omega \rightarrow \Omega_{\Lambda}$ allow us to define a $\sigma$-algebra on $\Omega$ with base in $\Lambda$:
\[
\sigma_{\Lambda} := \pi^{-1}_{\Lambda}(\mathcal B_{\Lambda}).
\]
If $S \subseteq \mathbb Z^n$ is a possibly infinite lattice, then we supply it with the $\sigma$-algebra
\[
\sigma_{S} := \sigma(\bigcup_{\Lambda \subset S, \ \Lambda \text{ finite}} \sigma_{\Lambda})
\]
(by the last equality I mean the smallest $\sigma$-algebra generated by the union).