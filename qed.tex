\section{Free Bosons}
	\subsection{Truncated Free Bosons}
	We have here one site with Hilbert space $\mathcal{H}_M$ of dimension $M$,  and basis $\{|0\rangle, |1\rangle, \cdots, |M-1\rangle\}$.
	We should here say $|0\rangle_M$, $|1\rangle_M$, etc, but will omit the extra subindex for clarity.
	We define creation $a^\dagger$ and destruction operators\footnote{Have checked that they are indeed adjoint to each other even in the truncated case.} $a$, (they should really be $a_M$ and $a^\dagger_M$ but will omit the subindex)
	such that $a|0\rangle=0$ (the zero of the Hilbert space), $a|n\rangle=\sqrt{n}|n-1\rangle$ for all $0<n<M$, 
	$a^\dagger|n\rangle=\sqrt{n+1}|n+1\rangle$ for all $0\le n < M - 1$, $a^\dagger|M-1\rangle=0$.
	Let $H:=H_M:=M^{-1}a^\dagger a$ be the ``Hamiltonian'' operator for free bosons.
	Then $H$ is self-adjoint, every element of the basis is an eigenvector of $H$, and with eigenvalues\footnote{Indeed, $H|0\rangle = 0$ and for $M-1 \geq m > 0$ we have $H|m \rangle = \frac{\sqrt{m}\sqrt{m}}{M}$. } $\{0, 1/M, 2/M, \cdots, (M-1)/M\}$.
	
	Consider the the algebra $\mathcal{U}_M$ generated by $a$ and $a^\dagger$ and the identity $I$. $H$ belongs to this algebra. It's a $C^*$-algebra (even a von Neumann algebra): we included in the list of generators all conjugates and we supplied the algebra with the operator norm, which always satisfies the $C^*$-identity. It's a von Neumann algebra because any finite-dimensional $C^*$-algebra is a von Neumann algebra.
	
Regarding the dimension, it's clear to me that the algebra is finite-dimensional and that the dimension can be bounded from below by $3M-2$. I haven't figured out yet the exact dimension.

	\subsection{Infinite Free Bosons}
		We have here  one site with separable Hilbert space $\mathcal{H}_\infty$ of infinite dimension,  and basis $\{|0\rangle, |1\rangle, \cdots\}$; 
	we define creation $a^\dagger$ and destruction operators $a$,
	such that $a|0\rangle=0$, $a|n\rangle=\sqrt{n}|n-1\rangle$ for all $0<n$, 
	$a^\dagger|n\rangle=\sqrt{n+1}|n+1\rangle$ for all $0\le n$. Notice that both $a$ and $a^\dagger$ are unbounded but their domains are dense in $\mathcal H_{\infty}$.
	
	\begin{statement}
	The approach with energy density does not work. Precisely, the following is true: for every $M \geq 1$, extend $H_M = \frac{1}{M} a_M^\dagger a_M$ to act on $\mathcal{H}_\infty$ by setting it equal to zero on the orthogonal complement of $\mathcal H_{M} \subset \mathcal H_{\infty}$. Then there is a well defined limit in the strong topology: $H_{\infty}|\psi \rangle = \lim_{M \rightarrow \infty} H_{M}|\psi \rangle$. However, $H_{\infty} = 0$. It might make sense if we divide not by the dimension but by the size of the lattice.
	\end{statement}
	\begin{proof}
	Indeed, for any basis ket $|k\rangle$, we have $\lim_{M \rightarrow} H_M|k\rangle = \lim_{k \rightarrow \infty}(k/M)|k\rangle = 0$, so the sequence $H_M$ converges strongly to zero.
	\end{proof}
	\begin{statement}
	Redefine $H_M$ as $H_M := a_M^\dagger a_M$ and extend by zero onto the orthogonal complement of $\mathcal H_{M}$, so $H_M |k \rangle = k |k\rangle$ for $k \leq M-1$. Then $H_M$ converges strongly to some unbounded self-adjoint (hence closed) operator $H_{\infty}$ on a dense subspace of $\mathcal H_{\infty}$.
	\end{statement}
	\begin{proof}
The domain of $H_{\infty}$ would be the dense subspace
\[
\Dom H_{\infty} = \{ |\psi\rangle = \sum_{k=0}^\infty \psi_k |k \rangle \in \mathcal H_{\infty} \ | \ \sum_{k=0}^\infty k |\psi_k| < \infty \}.
\]
So, for any $|\psi\rangle$ such that $\sum_{k=0}^\infty k |\psi_k| < \infty$, we can safely define\footnote{It's a general fact from the theory of Banach spaces that a series $\sum_{k=1}^\infty x_k$ converges iff the series $\sum_{k=1}^\infty \|x_k\|$ converges.}
\[
H_{\infty}|\psi \rangle :=\lim_{M \rightarrow \infty} H_{M}|\psi \rangle = \sum_{k=0}^\infty k\psi_k |k \rangle.
\]
The operator is clearly unbounded, for $\|H_{\infty}|k\rangle\| \rightarrow \infty$.

Now let's find the domain of $H_{\infty}^\dagger$. By definition of the adjoint of an unbounded operator, $|\phi \rangle \in \Dom (H_{\infty}^\dagger)$ if and only if there exists $|\theta\rangle \in \mathcal H_{\infty}$ such that for every $|\psi \rangle \in \Dom(H_{\infty})$ we have $\left\langle \phi | H_{\infty} | \psi \right\rangle = \langle \theta | \psi \rangle$. In components, this equality means that $\phi_k^* \psi_k k = \theta_k^* \psi_k$, hence $|\phi\rangle$ must reside in $\Dom(H_\infty)$. Since $H_{\infty}$ is obviously symmetric and $\Dom(H_{\infty}) = \Dom(H_{\infty}^\dagger)$, it is self-adjoint. From the general theory of unbounded operators we know that the adjoint is always closed, hence any $H_{\infty} = H_{\infty}^\dagger$ is closed.
	\end{proof}
	
	
	%For every $M\ge 1$ we extend $H_M$ to act  Since the norm of $H_M$ on $\mathcal H_{M}$ is equal to $(M-1)/M$, such is the norm of the extension as well.
	The limit (in operator norm, weak, strong?) of the sequence of operators $H_M$ exits (?), is unique, and let's call it $H_\infty$.
	Then $H_\infty$ is self adjoint (is it really?) and its spectrum is $\sigma(H_\infty) = [0, 1)$ (does it include 1?).
    Moreover, every $1/n$ for $n>=1$ is an eigenvalue of $H_\infty$, and the corresponding eigenvectors form a basis of $\mathcal{H}_\infty$.
    
	For $P\ge 1$, let  $v_P,\,w_P$ be both vectors in $\mathcal{H}_P$, and consider their extension to any $M\ge P$ by the same name. 
	then $lim_{M\rightarrow\infty} \langle v_P | H_M | w_P\rangle$ exists, and is equal to
	$\langle v_P | H_\infty | w_P\rangle$, where $v_P,\,w_P$ are considered vectors of $\mathcal{H}_\infty$.
	
Consider the algebra $\mathcal{U}_\infty$ generated by $a$ and $a^\dagger$ and the identity $I$. Does $H_\infty$ belong to this algebra?
What's the dimension of this algebra? Is this a C* algebra? Is it a von Neumann algebra? Etc.
	Is this algebra a limit in some sense from the sequence of algebras $\mathcal{U}_M$.
	
	\section{Free Fermions}
	\subsection{Finite Chain}
	Consider one site with Hilbert space of dimension two,  and basis $\{|0\rangle, |1\rangle\}$; physically we say
	that the state $|0\rangle$ is empty, and the state $|1\rangle$ occupied;
	we define creation $c^\dagger$ and destruction operators $c$,
	such that $c|0\rangle=0$, $c|1\rangle=|0\rangle$, $c^\dagger|0\rangle=|1\rangle$, $c^\dagger|1\rangle=0$.
	For $N$ sites, we extend the definitions by making $N$ tensor products, so that the total dimension is $2^N$.
	(Even though we use the word \emph{site} or \emph{sites}, we use here momentum space for simplicity, 
	so that the $H$ below will be already in ``diagonal form.'')
	Given a $N-$tensor product state $|v\rangle\equiv|v_0\rangle\otimes|v_1\rangle\otimes\cdots\otimes|v_{N-1}\rangle$, 
	the $v_m$ are either $0$ or $1$. Let the parity up to 
	$m$ operator $P_m$ be the diagonal operator such that  $P_m|v\rangle = p(|v\rangle, m)|v\rangle$, where
	$p(|v\rangle, m)$ is the number $1$ or $-1$, and is given by 
	\begin{equation}
	p(|v\rangle, m)\equiv(-1)^{\sum_{0\le m'<m} v_{m'}}.
	\end{equation}
	In other words, $P_m |v \rangle$ is $|v\rangle$ if the number of $|1\rangle$ strictly preceding $m$th position is even; $-|v\rangle$ if odd.
	
	We write $d_m$ to mean $I\otimes I\otimes\cdots \otimes c \otimes I \otimes \cdots\otimes I$, where $c$ is at location $m$,
	and $I$ is the one site identity. We write $c_m (N) \equiv d_m P_m$, where $P_m$ is the parity operator up to $m$, as defined before.
	Note that $c_m (N)$ acts on the full space $\{|0\rangle, |1\rangle\}^{\otimes N}$. \emph{I drop now the $(N)$ from the $c_m$.}
	
	\begin{example}
Let $|w\rangle=|0\rangle\otimes|1\rangle\otimes|1\rangle\otimes|0\rangle\cdots$, and remember that we count locations from $0$.
	Then $c_2|w\rangle=-|0\rangle\otimes|1\rangle\otimes|0\rangle\otimes|0\rangle\otimes\cdots$, with a -1 because the sum of 1s before location 2 is odd.
	On the other hand $c^\dagger_3|w\rangle = |0\rangle\otimes|1\rangle\otimes|1\rangle\otimes|1\rangle\otimes\cdots$, with a +1 because
	the sum of 1s before location 3 is even. Obviously  $c_2^\dagger|w\rangle = c_3|w\rangle = 0$ as we can't destroy if there's no particle,
	and we can't create if there's already a particle, because these are fermions and accept only one particle per location.
	\end{example}
	 
	We write $c^\dagger_m c_m$ to mean the tensor product operator with $c^\dagger$ and $c$ at location $m$, 
	identities elsewhere, \emph{and with appropriate parities,} so it's $d^\dagger_m  d_m$ (the parity operators commute with $d_m$'s and cancel each other out).
	Let 
	\begin{equation}
	H_N=N^{-1}\sum_{m=0}^{m=N-1} e_m c^\dagger_m c_m
	\end{equation} be the ``Hamiltonian'' operator for free fermions, where $e_m = -2\cos(2\pi m/N)$ for $0\le m < N$ integers;
	this is called the \emph{dispersion} relation for free periodic fermions.
	Then $H_N$ is a self-adjoint (symmetric) matrix of rank $2^N$. Note that $c^\dagger c |0\rangle = 0$ and $c^\dagger c |1 \rangle = |1 \rangle$.
	
	\subsubsection{Eigenvalues and Eigenvectors of the Hamiltonian}
	All this is just matrix diagonalization, but because of the form of the $2^N$ matrix $H$, we can find its
	eigenvalues and eigenvectors in a compact way, that we know how to describe.
	
	Let $S = \{ f: \{0, 1, \cdots, N - 1\} \rightarrow \{0, 1\}\}$, that can be thought of as binary numbers with $N$ digits.
	There are $2^N$ functions $f$ in $S$ and for each, the number
	$N^{-1}\sum_{m=0}^{m < N} f(m) e_m$ is an eigenvalue of $H_N$ with eigenvector\footnote{Checked on a sheet of paper.}
	\[
	\bigotimes_{m\in I(f)}c^\dagger_{m}|0\rangle,
	\]
	 where $I(f) = \{m\in\{0, 1, \cdots, N - 1\}; f(m) = 1\}$,  
	and $|0\rangle = |0\rangle\otimes|0\rangle\cdots|0\rangle$ the fully ``empty'' state.
 	 In particular, the lowest eigenvector of $H_N$ is
 	 \[
 	 \bigotimes_{m\in I_{min}}c^\dagger_{m}|0\rangle,
 	 \]
 	 with eigenvalue $N^{-1}\sum_{m \in I_{min}} e_m$, where $I_{min} = \{m; 0\le m \le N/4\}\cup\{m; 3N/4\le m < N\}$.
 	
\begin{exercise}
Express the largest eigenvector and eigenvalue to make sure you understand the construction.
\end{exercise}
\begin{proof}
The largest eigenvalue is determined by the set $I_{max} := \{m; \ \frac{N}{4} \leq m \leq \frac{3N}{4} \}$. I have understood the construction and worked out the details.
\end{proof}
 	 
 	 \subsubsection{Fermionic Density}
 	 Let $D_N=N^{-1}\sum_{m=0}^{m < N} c^\dagger_m c_m$ be the density operator. Then $D_N$ is self-adjoint, diagonal, and
 	 with eigenvalues\footnote{The eigenvectors are given by all possible functions $f:\{0,\ldots,N-1\} \rightarrow \{|0\rangle, |1\rangle\}$; i.e., all elements $f(0) \otimes f(1) \otimes \cdots \otimes f(N-1)$ are eigenvectors.} $\{0, 1/N, 2/N, \cdots, 1\}$. Moreover, $D_N$ and $H_N$ commute (physically, $H_N$ ``conserves'' the particle density
 	 or the number of particles), and the eigenvector of $H_N$ characterized by $f\in S$, is also an eigenvector of $D_N$
 	 with eigenvalue equal to the number of elements of $I(f)$ divided by $N$.
 	 
 	 Consider the algebra $\mathcal{U}_M$ generated by $c_m$ (with parity and in the $2^N$ space) and $c^\dagger_m$ and the identity $I$. The operators $H_N$ and $D_N$ belong to this algebra. Since the involution preserves the list of generators, it's also a finite-dimensional $C^*$-algebra. Because of its dimension, it's also a von Neumann algebra. What is its dimension?
 	 
 	 \subsection{Infinite Chain}
\emph{(A possible way, a guess of mine)} Define a set
\[
\mathcal B_{\infty} := \{|\psi_f\rangle = |f(0)\rangle \otimes |f(1)\rangle \otimes \cdots \ | \ f : \mathbb N \rightarrow \{0, 1\}, \ f(i) = 0 \ \text{for all but finitely many } i\}.
\]
 For the infinite chain, define the space of states via
\[
\mathcal H_{\infty} := \cl \Sp \mathcal B_{\infty} %\cl \Sp\{f(0) \otimes f(1) \otimes \cdots \ | \ \text{for all but finitely many }i \in \mathbb N, \ f(i) = |0\rangle \},
\]
where by $\cl$ I mean the completion. This space is obviously separable. The elements from $\mathcal B_{\infty}$ are declared to constitute an orthonormal basis. The basic kets, therefore, describe a state when some finite number of fermions occupies some sites. Define also subspaces
\[
\mathcal H_{M} := \Sp \{ |\psi_f\rangle \in \mathcal B_{\infty} \ | \ f(k+M) = 0 \ \text{for all} \ k \geq 0\}.
\]
Note that
\[
\mathcal H_{M}^{\perp} = \Sp \{ |\psi_f\rangle \in \mathcal B_{\infty} \ | \ f(k+M) = 1  \ \text{for some} \ k \geq 0\}.
\]
 	For every $M\ge 1$ we extend $H_M$ to act on $\mathcal{H}_\infty$ by setting it equal to zero on $\mathcal H_{M}^{\perp}$.
	
\begin{statement}
When $H_M$ are defined with $1/N$ factor, they converge strongly to zero.
\end{statement}
\noindent\textbf{A comment on the statement.} It's interesting that we get zero no matter whether we divide by the dimension of the space of states (the case of bosons) or by the size of the chain. Here, the reason is that I prohibited infinitely many fermions to occupy the chain. I think, first, it's not physically relevant, and second, this makes $\mathcal H_{\infty}$ non-separable. But if we allow infinitely many fermions, the limiting operator $\mathcal H_{\infty}$ is non-trivial.
\begin{proof}
Indeed, choose a basic ket $|\psi_f \rangle \in \mathcal B_{\infty}$ and let $N := \max\{ i \ | \ f(i) = 1 \}$. According to the finite chain case, we have
\[
\lim_{M \rightarrow \infty} H_{M} |\psi_f \rangle = \lim_{M \rightarrow \infty, \ M \geq N} \frac{1}{M} \sum_{m=0}^{m < N} f(m) (-2\cos(\frac{2\pi m}{M})) = 0.
\]
Thus $H_{\infty} = 0$. Note that the sum goes up to $N$, but inside of cosines we have $1/M$.
\end{proof}
\begin{statement}
Define $H_{M}$ without the factor of $1/M$. Then $H_{M}$ converge strongly to an unbounded self-adjoint operator $H_{\infty}$ defined via
\[
H_{\infty} |\psi_f\rangle := |I(f)| |\psi_f\rangle, 
\]
where by $|I(f)|$ I mean the number of elements in $I(f)$.
\end{statement}
\begin{proof}
We see that
\[
\lim_{M \rightarrow \infty} H_M|\psi_f\rangle =  \lim_{M \rightarrow \infty, \ M \geq N} \sum_{m=0}^{m < N} f(m) (-2\cos\frac{2\pi m}{M} ) |\psi_f\rangle = |I(f)||\psi_f\rangle,
\]
so $H_M|\psi_f\rangle$ is diagonal, though unbounded. We've already encountered an unbounded diagonal operator in the case of infinite free bosons, and we've shown it's self-adjoint\footnote{I mention it carefully since in the theory of unbounded operators not everything that seems symmetric (swings freely in the slots of the inner product) is self-adjoint. The latter condition is stronger.}
%The formula is clear from the previous statement. The action of this $H_{\infty}$ on a general state is given by
%\[
%H_{\infty} \sum_{|\psi_f\rangle \in \mathcal B_{\infty}} c_f |\psi_f \rangle = \sum_{|\psi_f\rangle \in \mathcal B_{\infty}} c_f \left(\sum_{m=0}^{m < N(f)} f(m) e_m \right)|\psi_f\rangle;
%\]
%the formula makes sense when
%\[
%\sum_{|\psi_f\rangle \in \mathcal B_{\infty}} |c_f| \left|\sum_{m=0}^{m < N(f)} f(m) e_m \right| < \infty.
%\]
%I checked numerically in Matlab that the largest eigen-values of $H_{M}$, as $M \rightarrow \infty$, grow approximately as the linear function $y = 1.84x$. So there are infintely many terms $|c_f|$ in the sum that are multiplied by something bigger than $1$. This implies that $H_{\infty}$ is unbounded: it's not defined on the whole space. Let's check the operator is self-adjoint. Pick two kets $|\psi_f\rangle,|\psi_g \rangle \in \mathcal B_{\infty}$. Then
%\[
%\langle \psi_f | H_{\infty} | \psi_g \rangle = 
%\]
\end{proof}

The last statement actually suggests what a factor we might pick for $H_{M}$'s to obtain an observable in the limit. We see that we can take $1/|I(f)|$:
\begin{statement}
Now try $H_M$'s via the formula
\[
H_M|\psi_f\rangle = \frac{1}{|I(f)|} \left(\sum_{m=0}^{m < M} f(m)e_m\right)|\psi_f\rangle.
\]
Then they converge strongly to the identity operator on $\mathcal H_{\infty}$.
\end{statement}
\begin{statement}
The density operators $D_N$ converge strongly to zero. If we redefine the density operators as $D_N := \sum_{m=0}^{m < N} c_m^\dagger c_m$, then $D_N$ converge strongly to the operator defined as $D_{\infty}|\psi_f\rangle = |I_f||\psi_f\rangle$ (which is unbounded, as shown before). Therefore, if we define $D_{N}$'s as $D_{N}|\psi_f\rangle := \frac{1}{|I_f|}\sum_{m=0}^{m < N} c_m^\dagger c_m$, then they converge to the identity operator.
\end{statement}

\emph{(What was written before) }The limit (in operator norm, weak, strong?) of the sequence of operators $H_M$ exits (?), is unique, and let's call it $H_\infty$.
 	 	Then $H_\infty$ is self adjoint (is it really?) and its spectrum is $\sigma(H_\infty) = [-2/\pi, 2/\pi]$.
 	 	
 	 	What are the eigenvalues (if any) and eigenvectors of $H_\infty$?
 	 	
 	 	Let $D_\infty$ be the corresponding limit of the $D_N$ operators. Show it's self-adjoint with spectrum $\sigma(D_\infty) = [0, 1]$.
 	 	What are the eigenvalues (if any) and eigenvectors of $D_\infty$?
 	 	
 	 Consider the algebra $\mathcal{U}_\infty$ generated by $c_i$ and $c^\dagger_i$ and the identity $I$ with some completion.
 	 How does one ``complete'' or ``close'' this algebra?
 	 Does $H_\infty$ belong to this algebra? What about $D_\infty$?
 	 What's the dimension of this algebra? Is this a C* algebra? Is it a von Neumann algebra? Etc.
 	 Is this algebra a limit in some sense from the sequence of algebras $\mathcal{U}_N$.
	\section{Pure Gauge in 1D}
TBW
	
	\section{Pure Gauge in 2D}
	TBW
	\section{Gauge and Matter in 1D}
	TBW
	
	
