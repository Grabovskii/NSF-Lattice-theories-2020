\subsection{Wilson loop for $m \gg 0$}
Split the Hamiltonian as $H = A + 2m\sqrt{n/2\pi} D$, where $A$ is the sum of the first term and the third terms of $H$ and $D = \sum_l (-1)^l \psi_l^\dagger \psi_l$.
\subsubsection{Dependence on $m$}
\begin{proposition}\label{p:dep_m}
Let $\lambda_1$ be the eigen-value of $H$ that corresponds to a ground state. Then $\lambda_1$ is a convex $C^1$-function of the mass. The ground state, as a function, can be chosen so that it's $C^1$.
\end{proposition}
\begin{proof}
Convexity is a consequence of Proposition \ref{p:conv_le}; differentiability of $\lambda_1$ is a consequence of Theorem \ref{thm:pert}. Differentiability of $|\gs\rangle$ is a result from Kato \cite{kato}, pp. 121-122 (more is stated there: for Hermitian differentiable operators we can choose differentiable orthonormal systems of eigen-vectors).
\end{proof}

\subsubsection{Estimate for $\lambda_1$}
\begin{proposition}\label{p:lambda_est_pos}
Let $\lambda_1$ be the eigen-value of $H$ that corresponds to a ground state. Then
\[
\lim_{m \rightarrow \infty} \frac{\lambda_1}{m} = -2\left[\frac{N}{2}\right]\sqrt{\frac{n}{2\pi}}.
\]
Also, for $m > 0$,
\[
\lambda_1(m) < -2m\left[\frac{N}{2}\right]\sqrt{\frac{n}{2\pi}}
\]
(note: the inequality is strict!). The limit implies that for $m \gg 0$ and $N$ even (or large), the lowest energy level can be approximated as 
\[
\lambda_1(m) \approx - Nm \sqrt{\frac{n}{2\pi}}.
\]
\end{proposition}
\begin{proof}
Let $m > 0$. From the definition of $\gs_2$, we see that
\[
m^{-1}\langle \st_2(s) | H| \st_2(s) \rangle \geq m^{-1}\langle \gs | H| \gs \rangle = m^{-1}\min_{\|\psi\|=1} \langle \psi |H|\psi\rangle \geq m^{-1}\min_{\|\psi\|=1} \langle \psi |A |\psi\rangle + 2\sqrt{\frac{n}{2\pi}} \langle \st_2(s) | D| \st_2(s)\rangle 
\]
%Note that 
%\[
%\lim_{m\rightarrow \infty} m^{-1}\langle \st_2(s) | H|\st_2(s) \rangle = 2\sqrt{\frac{n}{2\pi}} \langle \st_2(s) | D| \st_2(s)\rangle
%\]
%and
Note that
\[
\langle \st_2(s) | H|\st_2(s) \rangle = 2m\sqrt{\frac{n}{2\pi}} \langle \st_2(s) | D + \sum_l E_{l,l+1}^2 | \st_2(s)\rangle = -2m\left[\frac{N}{2}\right]\sqrt{\frac{n}{2\pi}}
\]
so, taking the limit in the above series of inequalities, we obtain the result.
%Hence, taking the limit in the above inequality, we obtain the statement. With regards to $\lambda_1$, we write $\langle \gs | H| \gs \rangle = \lambda_1\langle \gs |\gs\rangle = \lambda_1$.
With regards to the strict inequality for $\lambda_1$, $|\st_2(s)\rangle$ is clearly not an eigen-value of $H$ when $m$ is finite, so $\langle\st_2(s)|H|\st_2(s)\rangle$ cannot equal $\lambda_1$ (see the footnote\footnote{Let $|\psi_k\rangle$ constitute an orthonormal basis of eigen-vectors of $H$ (they can be chosen to have real components with respect to the standard basis for $H$ is hermitian and symmetric). Then $|\st_2(s)\rangle = \sum_{k} \alpha_k |\psi_k\rangle$ and $\sum_{k}\alpha_k^2 = 1$ (all $\alpha_k$'s are real). On the other hand, applying $\langle \st_2 |H$ to both sides yields $\lambda_1 = \sum_{k} \alpha_k^2 \lambda_k$. Since $|\st_2(s)\rangle$ is not an eigen-vector of $H$, there's $\lambda_j > \lambda_1$ such that $\alpha_j \neq 0$. We want to show that the above equality is not possible. Indeed, let $m$ be the index of the last $\lambda_i$ that is equal to $\lambda_1$. We can rewrite the equation as $\lambda_1 = \sum_{k=m+1}^n \frac{\alpha_k^2}{1-\sum_{i=1}^m \alpha_i^2} \lambda_k$, which is again a convex combination, so write $\lambda_1 = \sum_{k=m+1}^n \beta_k\lambda_k$. We know that $\lambda_1 < \lambda_{k}$ for every $k> m$. But then $\beta_k\lambda_1 \leq \beta_k\lambda_k$ for $k > m$, and for $j$ the inequality is strict: $\beta_j\lambda_1 < \beta_j\lambda_j$; summing up, we get $\sum_{k=m+1}^n \beta_k\lambda_1 < \sum_{k=m+1}^n \beta_k \lambda_k$, which contradicts the equality. Thus $\lambda_1 \neq \langle \st_2(s)|H|\st_2(s)\rangle$.

} for a detailed theoretical explanation).
\end{proof}
\subsubsection{Invariance properties of $H$}
Recall that the physical space (determined by the Gauss law) is given by
	\[
	P = \Sp ( |n_0\, n_1\, \ldots \, n_{N-1} \, b_{-1}\, b_0\, \ldots\, b_{N-2}\rangle \ | \ b_{\ell} =b_{\ell-1} + n_{\ell} + \frac{1}{2}((-1)^\ell-1) \ \text{for every} \ \ell).
	\]
	We can split $P$ into the direct sum according to what value of $b_{-1}$ is:
	\[
	P = P_0 \oplus P_1 \oplus \cdots \oplus P_{n-1},
	\]
	where
	\[
	P_j = \Sp( |n_0\, n_1\, \ldots \, n_{N-1} \, b_{-1}\, b_0\, \ldots\, b_{N-2}\rangle \in P \ | \ b_{-1} = j).
	\]

Also, let $N:= \sum_{l} \psi_l^\dagger \psi_l$ be the operator that counts fermions.

The following lemma is an obvious but very useful observation:

\begin{lemma}\emph{(Invariance properties of $H$)} The following hold:
\begin{enumerate}[(i)]
\item The restriction of $H$ to each subspace $P_j$. Denote $H_j := H|_{P_j}$.
\item $[N,H] = 0$. In particular, for a chosen $|\gs\rangle$, the basic kets $|\psi\rangle$ for which $\langle \psi | \gs \rangle \neq 0$ all have the same number of fermions.
\end{enumerate}
\end{lemma}

\subsubsection{Ground states for large mass}

\begin{lemma} \label{l:gs_conv_pos}
The following hold:
\begin{enumerate}
\item For any basic ket $|\psi\rangle$ such that $\langle \psi |D|\psi\rangle > -[N/2]$ (i.e., $|\psi\rangle \neq |\st_2(i)\rangle$ for any $i$), we have $\lim_{m \rightarrow \infty} \langle \gs | \psi \rangle = 0$;
\item For\footnote{By this I mean that there exists $m_0 > 0$ such that for all $m \geq m_0$ the statement is true.} $m \gg 0$ we have $\left\langle \gs | \st_2(s) \right\rangle \neq 0$. In particular, if for every large enough mass we choose $|\gs\rangle$ so that $\langle \gs | \st_2(s) \rangle > 0$ (which is possible), then $\lim_{m \rightarrow \infty} |\gs\rangle = |\st_2(s)\rangle$.
\end{enumerate}
\end{lemma}
\begin{proof}
\begin{enumerate}
\item Indeed, we see that
\[
\langle \psi | \gs \rangle = \frac{1}{\lambda_1} \langle \psi | H|\gs\rangle = \frac{1}{\lambda_1} \langle \psi |A|\gs\rangle + \frac{2m}{\lambda_1} \sqrt{\frac{n}{2\pi}} \langle \psi | D | \gs \rangle = \frac{1}{\lambda_1} \langle \psi |A|\gs\rangle + \frac{2m}{\lambda_1} \sqrt{\frac{n}{2\pi}} \cdot a \langle \psi | \gs \rangle
\]
where $a$ is an eigen-value of $D$ (which is not equal to $-[N/2]$ by assumption). The above can be rewritten as
\[
\langle \psi | \gs\rangle = \frac{1}{\lambda_1} \frac{1}{1- \frac{2ma}{\lambda_1}\sqrt{\frac{n}{2\pi}} } \langle \psi | A | gs\rangle.
\]
We see that 
\[
1- \frac{2ma}{\lambda_1}\sqrt{\frac{n}{2\pi}} \rightarrow 1 + a [N/2] \neq 0
\]
therefore
\[
\lim_{m \rightarrow \infty}\langle \psi | \gs \rangle = 0.
\]
\item Since each $P_i$ is invariant under $H$, we find $j$ such that $|\gs \rangle \in P_j$, so we can write it as $|\gs\rangle = \alpha |\st_2(j)\rangle + |v\rangle$, where $|v\rangle \perp |\st_2(j)\rangle$. The above implies together with $\rangle \gs | \gs \langle = 1$ implies
\[
\lim_{m \rightarrow \infty} (1 - |\alpha|^2) = 0
\]
or $\lim_{m \rightarrow \infty} |\langle \gs |\st_2(j)\rangle| = 1$. If we made the choice of ground states so that $\alpha > 0$, then $\lim_{m\rightarrow \infty} \alpha = 1$, so $|\gs\rangle \rightarrow |\st_2(j)\rangle$.

Let's prove that $j = s$. Write $|\gs\rangle = \alpha |\st_2(j)\rangle + \beta |v\rangle$ and $|\gs^\prime\rangle = \alpha |\st_2(s)\rangle + \beta |v^\prime \rangle$. The vector $|v^\prime \rangle$ is totally the same as $|v\rangle$ except that we substitute everywhere the bosonic state $b_{-1} = j$ with the state $b_{-1} = s$. The constants $\alpha$ and $\beta$ can be chosen real, $\alpha > 0$, and $\alpha^2+\beta^2 = 1$. Then we compare the expectation values of the Hamiltonian:
\[
\langle \gs | H|\gs \rangle  = \alpha^2 (-2m[N/2] \sqrt{n/2\pi} + (N-1)(j-s)^2) + O(\beta),
\]
\[
\langle \gs^\prime | H|\gs^\prime \rangle  = \alpha^2 (-2m[N/2] \sqrt{n/2\pi}) + O(\beta),
\]
hence their difference is
\[
\langle \gs | H|\gs \rangle - \langle \gs^\prime | H|\gs^\prime \rangle = \alpha^2(N-1)(j-s)^2 + O(\beta).
\]
It is supposed to be always negative. However, we can make $O(\beta)$ as small as we want, whereas $\alpha^2$ tends to $1$ and is multiplied by something stubbornly positive. Therefore, the difference can stay negative iff $j = s$.
\end{enumerate}
\end{proof}

\begin{conj}
It is true for any $m > 0$ that $\langle \gs|\st_2(s)\rangle \neq 0$. Numerical experiments suggest even more: $\langle \gs | \st_2(s)\rangle > \langle \gs | \psi \rangle$ for any other basic ket $\psi$.
\end{conj}

\begin{proposition}
Choose a family of ground states $|\gs\rangle$ that have $C^1$-dependence on $m$. Then $|\gs\rangle \in P_s$ for \emph{any} $m$, and $N|\gs\rangle = [N/2] |\gs\rangle$.
\end{proposition}
\begin{proof}
This is pure general topology. The image of the ground state curve lies in the disjoint union $P_1 \cap S^{\dim P} \cup \cdots \cup P_n \cap S^{\dim P}$. Since $|\gs\rangle$ is in particular continuous in $m$, the image of $\mathbb R$ lies only in one of the components of the union. Since for $m \gg 0$ it happens in $P_s\cap S^{\dim P}$, it's there for \emph{any} $m$.
\end{proof}

\begin{proposition}
For $m \gg 0$, a ground state is non-degenerate.
\end{proposition}
\begin{proof}
Assume on the contrary that it is degenerate for $m \gg 0$. This means there's a sequence of values of mass $m_k \rightarrow \infty$ and sequences of ground states $|\gs_k\rangle$ and $|\gs^\prime_k\rangle$ such that $\langle\gs_k^\prime|\gs_k\rangle = 0$. However, passing to the limit in the slots of the Hermitian product yields $\langle \st_2(s) | \st_2(s) \rangle \neq 0$. This is a contradiction, and thus the ground state is unique for large $m$.
\end{proof}
%\begin{proof}[Heuristic evidence]
%I believe that $|\gs\rangle$, if chosen so that $\langle \gs | \st_2(s)\rangle \in [0,1]$, is a continuous function of $m$. The eigen-value $\lambda_1$ is continuous as well. But recall that $P = P_1 \oplus \cdots \oplus P_n$. The sets $S^{\dim P} \cap P_i$ (an intersection of the unit sphere with a subspace) are disjoint, and since $|\gs\rangle$ is continuous, it maps the whole real line $\mathbb R$ into only one such intersection, i.e. into $S^{\dim P} \cap P_s$. So $b_{-1} = s$ always has to be the bosonic state for $|\gs\rangle$.
%\end{proof}

\begin{proposition}
Let $\Sigma = N^{-1}\sum \langle E_{l,l+1} \rangle$ be the electric field operator. Then\footnote{No matter how we choose the ground states for every value of mass since we're under the Hermitian inner product.} $\lim_{m \rightarrow \infty} \Sigma = 0$.
\end{proposition}
\begin{proof}
Indeed, by Lemma \ref{l:gs_conv_pos} we can assume that $\lim_{m \rightarrow \infty} |\gs\rangle = |\st_2(s)\rangle$. Then 
\[
\langle \gs | E_{l,l+1} | \gs \rangle \rightarrow \langle \st_2(s) | E_{l,l+1}|\st_2(s)\rangle = 0.\qedhere
\]
\end{proof}

\subsection{Wilson loop for $m \ll 0$}
%\begin{proposition}
%We have
%\[
%\lim_{m \rightarrow -\infty} \frac{\lambda_1}{m} = 2\left[\frac{N}{2}\right]\sqrt{\frac{n}{2\pi}}.
%\]
%\end{proposition}
%\begin{proof}
%As in Proposition \ref{p:lambda_est_pos}, we always have an inequality by the very definition of $|\gs\rangle$, but now, since the coefficient in front of $D$ is negative, we refer to $\st_1$:
%\[
%\langle \st_1 | H|\st_1 \rangle \geq \langle \gs | H|\gs\rangle \geq \min_{\|\psi\| = 1} A\psi + 2m \sqrt{\frac{n}{2\pi}} \langle \st_1 |D|\st_1\rangle.
%\]
%Multiplying by $m^{-1}$ for $m$ negative flips the inequality:
%\[
%m^{-1}\langle \st_1 | H|\st_1 \rangle \leq m^{-1}\langle \gs | H|\gs\rangle \leq m^{-1}\min_{\|\psi\| = 1} A\psi + 2 \sqrt{\frac{n}{2\pi}} \langle \st_1 |D|\st_1\rangle.
%\]
%Obviously,
%\[
%\lim_{m\rightarrow -\infty} m^{-1}\langle \st_1 | H|\st_1\rangle = 2 \sqrt{\frac{n}{2\pi}} \langle \st_1 |D|\st_1\rangle;
%\]
%therefore, passing to the limit as $m \rightarrow -\infty$ sandwiches $m^{-1}\langle \gs | H|\gs\rangle$ and yields
%\[
%\lim_{m \rightarrow -\infty} m^{-1}\langle \gs |H|\gs \rangle = 2 \sqrt{\frac{n}{2\pi}} \langle \st_1 |D|\st_1\rangle = 2\left[\frac{N}{2}\right]\sqrt{\frac{n}{2\pi}}.
%\]
%In particular, 
%\[
%\lim_{m \rightarrow -\infty} \frac{\lambda_1}{m} = 2\left[\frac{N}{2}\right]\sqrt{\frac{n}{2\pi}}.
%\]
%\end{proof}
%\begin{lemma}\label{l:gs_dec1}
%If $\langle \st_1(s)|\gs \rangle > 0$ and $\lambda_1 \neq -\langle \st_1(s)|\gs\rangle^{-2} + 2[N/2]$, the ground state admits a decomposition
%\[
%|\gs \rangle = \alpha \st_1(s)\rangle + |v \rangle
%\]
%where $|v\rangle$ is orthogonal to $|\st_1(i)\rangle$ for all $i$.
%\end{lemma}
%\begin{proof}
%We use the same idea as in \ref{l:gs_dec} except that now the coefficients $c_j$ are taken with respect to $|\st_1(j)\rangle$:
%\[\begin{split}
%c_j& = \langle\st_1(j) | H | \st_1(j) \rangle =\langle\st_1(j) | D+E | \st_1(j) \rangle=\\&= (j+1-s)^2[N/2] + (j-s)^2[(N-1)/2] + [N/2].
%\end{split}
%\]
%The resulting objective function is the same. When $\lambda_1 \neq -\langle \st_1(s)|\gs\rangle^{-2} + 2[N/2]$, we obtain a unique solution for the components of $|\gs\rangle$ with respect to $|\st_1(i)\rangle$'s.
%\end{proof}
%
%\begin{proposition}
%Assume that the choice of the ground state for every negative mass is made in such a way that $\langle \gs | \st_1(s)\rangle > 0$. Then the limit exists and is given by $\lim_{m \rightarrow -\infty} |\gs\rangle = |\st_1(s)\rangle$.
%\end{proposition}
%\begin{corp}
%Let $\Sigma$ be the electric field operator. Then
%\[
%\lim_{m \rightarrow -\infty} \Sigma = \frac{1}{N}\left[\frac{N}{2}\right].
%\]
%\end{corp}
%\begin{proof}
%The proposition tells us that
%\[
%\lim_{m \rightarrow -\infty} \left\langle \gs |W|\gs \right\rangle = \langle \st_1(s) |N^{-1}\sum_l E_{l,l+1} | \st_1(s)\rangle = \frac{1}{N}\left[\frac{N}{2}\right].\qedhere
%\]
%\end{proof}