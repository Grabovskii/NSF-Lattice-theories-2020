\documentclass[11pt]{article}%{amsart}%{article}
\renewcommand{\baselinestretch}{1.05}
\topmargin0.0cm
\headheight0.0cm
\headsep0.0cm
\oddsidemargin0.0cm
\textheight23.0cm
\textwidth16.5cm
\footskip1.0cm
%\usepackage[russian]{babel} %proof'ы становятся доказательствами, если включить
\usepackage{amsmath}%
\usepackage{amsfonts}%
\usepackage{amssymb}%
\usepackage{graphicx}
%\documentclass[a4paper,12pt]{article}
%вот эти три отвечают каким-то образом за могучий русский язык
\usepackage[T2A]{fontenc}
%\usepackage[utf8]{inputenc}
\usepackage[cp1251]{inputenc}

\usepackage{showlabels}

%\usepackage{enumitem}
%\setlist[enumerate]{label*=\arabic*.}

\usepackage{amsthm} %for theorem styles
\usepackage{hyperref} %for reference
\usepackage{enumerate}
%\usepackage{mbboard}

\usepackage[all,cmtip]{xy}

%\usepackage[small,nohug,heads=littlevee]{diagrams}
%\diagramstyle[labelstyle=\scriptstyle]


\DeclareMathOperator{\Ker}{Ker} 
\DeclareMathOperator{\supp}{supp}
\DeclareMathOperator{\Imm}{Im} 
\DeclareMathOperator{\cl}{cl}
%\DeclareMathOperator{\Sp}{Sp} %линейная оболочка
\DeclareMathOperator{\Spz}{\overline{Sp}}
\DeclareMathOperator{\Ree}{Re}
\DeclareMathOperator{\Int}{int}
\DeclareMathOperator{\Dom}{dom}
\DeclareMathOperator{\Sp}{span}
\DeclareMathOperator{\icr}{icr}
\DeclareMathOperator{\co}{co}
\DeclareMathOperator{\re}{Re}
\DeclareMathOperator{\id}{id}
\DeclareMathOperator{\Aut}{Aut}
\DeclareMathOperator{\Alt}{Alt}
\DeclareMathOperator{\Inner}{Inn}
\DeclareMathOperator{\lcm}{lcm}
\DeclareMathOperator{\res}{res}
\DeclareMathOperator{\card}{card}
\DeclareMathOperator{\trace}{trace}
\DeclareMathOperator{\Min}{Min}
\DeclareMathOperator{\Gal}{Gal}
\DeclareMathOperator{\charr}{char}
\DeclareMathOperator{\su}{su}
\DeclareMathOperator{\sll}{sl}
\DeclareMathOperator{\Ad}{Ad}
\DeclareMathOperator{\ad}{ad}
\DeclareMathOperator{\Tor}{Tor}
\DeclareMathOperator{\Tr}{Tr}
\DeclareMathOperator{\tr}{tr}
\DeclareMathOperator{\trdeg}{trdeg}
\DeclareMathOperator{\sgn}{sgn}
\DeclareMathOperator{\End}{End}
\DeclareMathOperator{\Ext}{Ext}
\DeclareMathOperator{\Ann}{Ann}
\DeclareMathOperator{\Mat}{Mat}
\DeclareMathOperator{\diag}{diag}
\DeclareMathOperator{\GL}{GL}
\DeclareMathOperator{\gl}{gl}
\DeclareMathOperator{\rank}{rank}
\DeclareMathOperator{\Ric}{Ric}
\DeclareMathOperator{\so}{so}
\DeclareMathOperator{\Hess}{Hess}
\DeclareMathOperator{\sff}{\mathrm{I\!I}}
\DeclareMathOperator{\pr}{pr}
\DeclareMathOperator{\Supp}{Supp}
\DeclareMathOperator{\Spec}{Spec}
\DeclareMathOperator{\codim}{codim}
\DeclareMathOperator{\Hom}{Hom}
\DeclareMathOperator{\Cl}{Cl}
\DeclareMathOperator{\SO}{SO}

%%%%%%%%%%%%%%
%% МАКРОСЫ
%%%%%%%%%%%%%%
\newcommand{\dif}{\left.\frac{d}{dt}\right|_{t=0}}
\newcommand{\difs}{\left.\frac{d}{ds}\right|_{s=0}}
\newcommand{\g}{\mathfrak g}
\newcommand{\h}{\mathfrak h}
\newcommand{\hh}{\mathcal{H}}
\newcommand{\frb}{\mathfrak b}
\newcommand{\pprime}{{\prime\prime}}


%------------------------------------------------------------
\newtheorem{myth}{Theorem}
\newtheorem{theorem}{Theorem}
\theoremstyle{definition}
\newtheorem{example}{Example}
\theoremstyle{definition}
\newtheorem{myexample}{Example}
\newtheorem{corollary}{Corollary}[theorem]
\newtheorem{mycor}{Corollary}[myth]
\theoremstyle{definition}\newtheorem{definition}{Definition}
%\theoremstyle{definition}\newtheorem{definition}{Определение}
\theoremstyle{definition} \newtheorem{cntex}{Counterexample}
\theoremstyle{definition}\newtheorem*{acknowledgement}{Acknowledgement}
\newtheorem{lemma}{Lemma}
\newtheorem{problem}{Problem}
\newtheorem{proposition}{Proposition}
\newtheorem{corp}{Corollary}[proposition]
\theoremstyle{definition}\newtheorem{remark}{Remark}
\theoremstyle{definition}\newtheorem{quest}{Question}
\numberwithin{equation}{section}
\newtheorem{exercise}{Exercise}
\newtheorem{fur}{Further Search}
\newtheorem{ths}{Thoughts Aloud}
\newtheorem*{statement}{Statement}
\newtheorem*{idea}{Idea}
%--------------------------------------------------------
\begin{document}
\tableofcontents
\section{Preliminaries}
\subsection{The main principles of classical lattice models}
Let $\Omega$ be a finite set (the set of \emph{microstates}), let $\hh : \Omega \rightarrow \mathbb R$ be a \emph{hamiltonian}, a specifically chosen random variable. Let $\mathcal M(\Omega)$ be the space of probability measures on $\Omega$. In this theory, the expectation of a random variable $f : \Omega \rightarrow \mathbb R$ with respect to a measure $\mu$ (if not implicitly understood; it's also called the \emph{thermal average}) is denoted as
\[
\left\langle f \right\rangle_{\mu} := \mathbb E_{\mu} f = \int_{\Omega} f d\mu.
\]
One of the first questions in statistical mechanics is devoted to the choice of the right measure $\mu$. The choice is governed using Shannon's entropy $S : \mathcal M(\Omega) \rightarrow \mathbb R$, defined as $S(\mu) := -\int_{\Omega}x\log x d\mu(x)$ (there's a way to understand why $S$ has this form; see \cite{friedli}). \emph{The maximum entropy principle says}: for a given model of statistical mechanics, choose $\mu$ that maximizes $S$. For example, if there's no other information avaiable about the system, then the measure that maximizes $S$ is the uniform distribution. There are other two typical situations:
\begin{enumerate}[(a)]
\item If we know that $\left\langle \mathcal H\right\rangle = U$ for some fixed $U$ (the \emph{internal energy} of the system), then the measure that maximizes $S$ is 
\[
\mu(\omega) = \frac{e^{-\beta H(\omega)}}{Z}, \ \ Z:= \sum_{\omega \in \Omega} e^{-\beta \mathcal H(\omega)} 
\]
which is the \emph{Gibbs measure}. This corresponds to the situation when we know the system exchanges its energy with some external thermal reservoir. The result can be obtained using Lagrange multipliers; the parameter $\beta$, called the \emph{inverse temperature}, is uniquely determined by $U$ (and vice versa: $U$ is uniquely determined by $\beta$). In the theory, $\beta = (kT)^{-1}$, where $k$ is the Boltzmann constant and $T$ is the temperature of the system.
\item If we know additionally that the system exchanges its particles $\mathcal N$ with the external environment, and that the expected value of the particles if $\left\langle \mathcal N \right\rangle = N$, then we obtain a \emph{grand canonical Gibbs distribution} via a similar procedure. It's given by
\[
\mu(\omega) := \frac{e^{-\beta(\hh(\omega)-\mu N)}}{Z}, \ \ Z := \sum_{N} e^{\beta \mu N} \sum_{\omega \in \Omega} e^{-\beta \hh (\omega)}.
\]
The parameter $\mu$ is identified with the so-called \emph{chemical potential}.
\end{enumerate}

\subsection{The thermodynamic limit}

\subsection{Quantum lattices}
\subsubsection{A general set-up}
I will follow closely the treatment in \cite{israel}. Again, we have a lattice $X \subset \mathbb Z^n$, but to each site $i \in X$ we attach a copy of a finite-dimensional Hilbert space $H_i$. To a finite $X$ we attach the tensor product $H_X := \otimes_{i \in X} H_i$. %For example, in spin-$1/2$ systems each Hilbert space is two-dimensional, with an orthonormal basis corresponding to spins up and down.

For infinite lattices, the author of \cite{israel} suggests proceeding as follows. Let $\mathfrak{A}_X := \End(H_X)$, and for any two finite subsets $X \subseteq Y \subset \mathbb Z^n$, let $\imath : \mathfrak{A}_X \rightarrow \mathfrak{A}_Y$ be the inclusion that sends $A$ to $A \otimes 1$ (where $1$ is viewed as an endomorphism of $\mathfrak{A}_{Y\setminus X}$). For an infinite $\Lambda \subseteq \mathbb Z^n$, the family of all its finite subsets with the inclusions form a direct system. Let $\mathfrak A_{\Lambda} := \varinjlim \mathfrak{A}_X$ be the direct limit taken in the category of $C^*$-algebras over all finite subsets of $\Lambda$. In the literature, this algebra is known as an AF (approximately finite-dimensional) $C^*$-algebra. The first reference in this theory goes back to Bratteli \cite{bratteli}. See the next subsection for an elaboration on the inductive limit.

Further, for a finite $\Lambda \subset \mathbb Z^n$ and a Hamiltonian $\mathcal H_{\Lambda}$, the partition function is defined as
\[
Z = \tr_{\Lambda} e^{-\beta \mathcal H_{\Lambda}}
\]
and the expectation of an observable $A \in \mathfrak A_{\Lambda}$ is 
\[
\left\langle A \right\rangle_{\Lambda} := Z^{-1} \tr_{\Lambda} (A e^{-\mathcal H_{\Lambda}}).
\]
The trace in these formulas is normalized: it's $1/d$ of the usual trace, where $d$ is the dimension of the Hilbert space at one site. An interesting consequence of such normalization is that $\tr$ extends than to a norm-one linear functional on the whole $\mathfrak A:= \mathfrak A_{\mathbb Z^n}$ (see \cite{israel}). The Hamiltonian they choose is given by
\[
\mathcal H_{\Lambda} = \sum_{X \subseteq \Lambda} \Phi(X),
\]
where $\Phi$ is a so-called \emph{interaction}: it's a function from the non-empty finite subsets of $\mathbb Z^n$ to self-adjoint operators on them, such that $\Phi(X+i) = \Phi(X)$ for any $i \in \mathbb Z^n$ (i.e., it's translational invariant).

The pressure for a finite region $\Lambda$ in the quantum lattice system is given by
\[
P_{\Lambda}(\Phi) := |\Lambda|^{-1} \ln \tr e^{-H_{\Lambda}}.
\]
One can show that the limit in the sense of van Hove of $P_{\Lambda}$ does exist in the quantum setting as well (\cite{israel}).

%\begin{fur}
%Not yet have I delved into the properties of AF $C^*$-algebras. How is the norm defined there at least? Not yet clear.
%\end{fur}
%
%\begin{fur}
%It's said in \cite{bratteli} that one might refer to \cite{haag} for limits of lattices in case the number of states is infinite.
%\end{fur}
\subsubsection{The inductive limit in more detail}
For two finite subsets $X \subseteq Y \subset \mathbb Z^n$, the inclusion $\imath : \mathfrak A_{X} \rightarrow \mathfrak A_{Y}$ that sends $A$ to $A \otimes 1$ is injective; therefore, whenever $X \subseteq Y$, we can view $\mathfrak A_{X}$ as a subalgebra of $\mathfrak A_{Y}$. Hence we can take a union of all such subalgebras coming from finite subsets of $\Lambda \subseteq \mathbb Z^n$, and then, to be safe and ensure it's a Banach space, take the closure. So, one can identify (see a proposition below for a rigorous proof)
\[
\mathfrak A_{\Lambda} = \varinjlim \mathfrak A_{X} = \cl \left(\bigcup_{X \subset \Lambda, \ |X| < \infty} \mathfrak A_{X}\right)/_{u \sim u \otimes 1}
\]
From this point of view, it's easy to understand what the norm is. For $A$ from the dense subspace (the union itself), we just set $\|A\|_{\Lambda} := \|A\|_{X}$ if $A \in \mathfrak A_{X}$. The norm extends to the closure by the very process of completeness: for $A \in \mathfrak A_{\Lambda}$, we choose a sequence $A_n \in \mathfrak A_{X_n}$ and then set $\|A\|_{\Lambda} := \lim_{n \rightarrow \infty} \|A_n\|$.

From Appendix on $C^*$ algebras, we see that Gelfand-Naymark theorem ensures there is a Hilbert space $H$ such that $\mathfrak A_{\Lambda} \cong \End(H)$.
\begin{fur}
I can elaborate on the construction of this Hilbert space. It's more or less constructive and relies on finding pure states. In particular, it would be interesting to see how this $H$ is related to the infinite tensor product $\otimes_{i \in \Lambda} H_i$: what exactly goes wrong?
\end{fur}
\begin{proposition}
In the above set-up, we indeed have $\varinjlim \mathfrak A_{X} = \cl \left(\bigcup_{X \subset \Lambda, \ |X| < \infty} \mathfrak A_{X}\right)/_{u \sim u \otimes 1}$ (isometrically and preserving the $\ast$-structure).
\end{proposition}
\begin{proof}
Denote $\mathfrak A^\prime := \left(\bigcup_{X \subset \Lambda, \ |X|<\infty} \mathfrak A_X\right)/_{u \sim u \otimes 1}$.
So, we choose morphisms in the category of unital $C^*$-algebras as bounded unital $\ast$-homorphisms with norm less then or equal to one\footnote{Otherwise I don't think there's a way to prove that the map induced on the diagram of the injective limit is a bounded operator}. To prove the statement, all we need to show is that for a unital $C^*$-algebra $A$ and a bunch of morphisms $\alpha_X : \mathfrak A_X \rightarrow \mathfrak A_Y$ where $X \subseteq Y$ and such that $\alpha_X(u) = \alpha_Y(u \otimes 1)$ (but remember that $u$ is identified with $u \otimes 1$ in the union), there's a unique morphism $\alpha : \cl \mathfrak A^\prime \rightarrow A$. In the language of diagrams, this is saying that
\[
\xymatrix{
& & A & & \\
& &  & & \\
& & \cl \mathfrak A^\prime\ar@{-->}[uu]_{\exists ! \alpha} & &\\
\mathfrak A_X\ar[rrrr]^{u \mapsto u \otimes 1} \ar[rru]^{\imath_X} \ar[rruuu]^{\alpha_X} & & & & \mathfrak A_Y\ar[lluuu]_{\alpha_Y} \ar[llu]_{\imath_Y}
}
\]
Once $\alpha$ is defined on $\mathfrak A^\prime$ with all the mentioned properties, it automatically extends to the closure. So, for $u \in \mathfrak A_X$ we set $\alpha(u) := \alpha_X(u)$. This is well defined, for $u$ is identified with $u \otimes 1$ in the union. We get automatically that $\alpha$ is a unital $\ast$-homorphism since all $\alpha_X$'s are. It's norm is bounded by $1$, for $\|\alpha(u)\| \leq \|\alpha_X\|\|u\| \leq \|u\|$. Thus $\alpha$ is a morphism in the corresponding category.
\end{proof}

\subsection{Relation between classical and quantum lattices}
I follow \cite{israel} with some minor modifications more appealing to my taste. Let $\Omega_0$ be a finite set of microstates at one site, and let $H_0$ be a Hilbert space of dimension equal to $|\Omega|$ (which is assigned to one site as well). Let $C(\Omega)$ be the space of observables on $\Omega$. Choose an orthonormal basis $e_{\mu}$ of $H_0$ labeled my microstates $\mu \in \Omega_0$. Then we have an injection $\imath : C(\Omega_0) \rightarrow \End(H_0)$ given by
\[
[\imath(f)](e_{\mu}) := e_{f(\mu)}.
\]
In other words, the classical observables are embedded into the quantum observables as diagonal matrices.

\subsection{Continuous spins: general principles}\label{ss:gen_princ}
I follow closely Section 6.10 of \cite{friedli}. In case the space of states $\Omega_0$ at a single site is non-compact, the existence of Gibbs measures is no longer guaranteed. For $\Omega_0$ a topological space, one defines the following ingredients. Let $\mathcal B_0$ be the Borel $\sigma$-algebra on $\Omega_0$. For a finite lattice $\Lambda \subset \mathbb Z^n$, we supply the space of states with the $\sigma$-algebra $\mathcal B_{\Lambda} := \bigotimes_{i \in \Lambda} \mathcal B_0$. The natural projections $\pi_{\Lambda} : \Omega \rightarrow \Omega_{\Lambda}$ allow us to define a $\sigma$-algebra on $\Omega$ with base in $\Lambda$:
\[
\sigma_{\Lambda} := \pi^{-1}_{\Lambda}(\mathcal B_{\Lambda}).
\]
If $S \subseteq \mathbb Z^n$ is a possibly infinite lattice, then we supply it with the $\sigma$-algebra
\[
\sigma_{S} := \sigma(\bigcup_{\Lambda \subset S, \ \Lambda \text{ finite}} \sigma_{\Lambda})
\]
(by the last equality I mean the smallest $\sigma$-algebra generated by the union).
\section{The Main Stage}
\subsection{Statement of the problem and ideas}
The current statement, I guess, is the following:

\begin{statement}
Consider $\mathbb Z^n$ where to each node we attach an infinite-dimensional separable Hilbert space $H$. Let $\Lambda \subseteq \mathbb Z^n$ be an infinite sublattice. Consider the limit
\[
\mathfrak A_{\Lambda} := \varinjlim_{X \subset \Lambda, \ \ X\ \text{finite}} \mathfrak A_{X}.
\]
If it turns out that $\mathfrak A_{\Lambda}$ is a $C^*$-algebra, I'd like to do the following: find an ideal $J \triangleleft \mathfrak A_{\Lambda}$ such that the Hilbert space $H$ promised by the Gelfand-Naymark theorem is separable; i.e., $\mathfrak A_{\Lambda}/J \cong \End(H)$ for $H$ separable. It would be also nice to keep embeddings $\mathfrak A_{X} \rightarrow \mathfrak A_{\Lambda}$ for finite sublattices $X \subset \Lambda$.
\end{statement}

\begin{fur}
Given a unital $C^*$-algebra $A$, under which conditions on $A$ the Hilbert space given by Gelfand-Naymark theorem is separable?
\end{fur}

\begin{idea}
To keep everything physically meaningful, I think that $J$ can be tried out as the ideal generated by operators with all but finitely many eigen-states concentrated in a finite sublattice of $\Lambda$. Here, I need to refresh my mind with regards to eigen-states.
\end{idea}

\subsection{Tests on quantum Ising model}
In Ising model, to each node of $\mathbb Z^2$ we attach a $2$-dimensional Hilbert space $H$ with some a priori chosen orthonormal basis $e_1,e_2$. It corresponds to the states \emph{spin up} and \emph{spin down}.
\subsubsection{The issue with the infinite tensor product}
Let's consider first the algebraic tensor product $H_{\infty}$ of all Hilbert spaces attached to all sites. It's spanned by simple tensors of the form
\[
e_{\lambda(1)} \otimes e_{\lambda(2)} \otimes e_{\lambda(3)} \otimes \cdots 
\]
where $\lambda : \mathbb N \rightarrow \{e_1,e_2\}$ is a function. There are as my such simple tensors as functions $\lambda$; the cardinal number is equal to $|2^{\mathbb N}| = |\mathbb R|$, i.e., there are uncountably many of them\footnote{More details on why it's uncountable. Any number $a \in \mathbb R$ can be represented as a power series $a = \sum_{i=-\infty}^\infty c_i 2^{i}$, where $c_i \in \{0,1\}$ and only finitely many $c_i$ for $i > 0$ might be non-zero. Restrict ourselves to $a = \sum_{i=1}^{\infty}c_i 2^{-i}$. Then we have a 1-1 correspondence between functions $\lambda : \mathbb N \rightarrow \{0,1\}$ and such numbers. }. It's natural to declare such simple tensors an orthonormal basis of $H_{\infty}$. But then, it's a result of metric spaces theory that if there are uncountably many points such that the distance between any of two is bounded by a positive constant (that doesn't depend on the points), then the space is not separable. This is the case with $H_{\infty}$. Since it's not separable, its completion $\cl H_{\infty}$ can't be separable as well. By the way, the same idea is used when one proves $l_{\infty}$ is not separable.

Just out of curiosity, the same cardinal number occurs when all Hilbert spaces are infinite-dimensional but separable. In this case, we deal with all functions $\lambda : \mathbb N \rightarrow \mathbb N$; the cardinal number of them is again ${|\mathbb N}^{\mathbb N}| = |\mathbb R|$.

%\subsubsection{The limit of algebras of observables in combination with Gelfand-Naymark theorem}
\subsubsection{The limit and G-F theorem}

For an infinite sublattice $\Lambda \subseteq \mathbb Z^n$, the injective limit $\varinjlim_{X \subset \Lambda, \ X \ \text{finite}} \mathfrak A_{X}$ can be thought of as the completion of the union of those subalgebras. 

The following example explains how the union works.
\begin{example}The algebra $\mathfrak A_{1}$ attached to a single site can be identified with the algebra of $2\times 2$ matrices over $\mathbb C$. For two nodes, the algebra $\mathfrak A_{2}$ can be identified with matrices of dimension $4 \times 4$. The embedding $\mathfrak A_{1} \rightarrow \mathfrak A_{2}$ then does the following:
\[
A:=\begin{pmatrix} a_{11} & a_{12}\\ a_{21} & a_{22}\end{pmatrix}
\mapsto A \otimes 1 = \begin{pmatrix} A & 0 \\ 0 & A\end{pmatrix}.
\]
\end{example}

Therefore, for an inifnite sublattice $\Lambda$, the algebra $\mathfrak A_{\Lambda}$ can be identified with the space of matrices of infinite size such that only finitely many entries of each of them are non-zero. \emph{Note} the difference with the attempt to take an infinite tensor product of Hilbert spaces: these infinite matrices naturally act upon the space 
\[
T:=\bigoplus_{n=1}^\infty \, \bigotimes_{i \in A \subset \Lambda, \ |A| = n} H_i 
\]
It's easy to see that this space has an infinite countable basis. This implies that, whatever norm we put on $T$, the space will not be complete (that's a standard result from functional analysis: in a Banach space, a vector space basis is at least uncountable).

\textbf{Way 1 (just a fantasy)}. Let $\mathfrak A_{\Lambda}^\prime$ be the union of all $\mathfrak A_X$ for $X \subset \Lambda$ and $X$ finite. We can substitute the norm on $\mathfrak A_{\Lambda}^\prime$ with the Hilbert-Schmidt norm (see appendix), and then complete $\mathfrak A_{\Lambda}^\prime$ with respect to it. The elements $A$ of the resulting space can be represented as infinite matrices $(a_{nk})_{k,n = 1}^{\infty}$ such that $\sum_{n,k} |a_{nk}|^2 < \infty$. We can act with these on a completion of $T$. However, it's not a $C^*$-algebra; it is a Banach algebra though. There might be something in this approach.

\textbf{Way 2.} We can complete with respect to the operator norm. I can't prove this, but the evidence is that we obtain the space of compact operators on $\cl T$. That's very good. If our Hamiltonian is normalized in such a way that in the limit it gives a bounded operator, then we can employ the Hilbert-Schmidt theorem and find an orthonormal basis in $\cl T$ of eigen-values of the limiting Hamiltonian. 

%\begin{fur}
%I see so far two ways one can go here, I need to delve into both of them. One way leads to Hilbert-Schmidt operators (if we had a different norm at the begininning), the other (assuming the operator norm), I guess, leads to compact operators. I think we need to study both these cases. Reference \cite{conway} tells how to realize the first way (page 268).
%\end{fur}

%
%Therefore, the union is made of infinite matrices with the property that each contains a submatrix of a finite dimension outside of which the matrix entries equal to zero.
%
%\begin{fur}
%I haven't understood yet how the completion of such a space looks like. In the process of looking this up and at the same time I'm pondering over this.
%\end{fur}

\subsection{Thoughts on $C^*$-algebras approach}
So the idea was the following: since it's not sometimes clear what a limiting Hilbert space should look like, we can take the limit of the corresponding algebras of observables and then, by Gelfand-Naymark theorem, find an underlying Hilbert space, hopefully a separable one. But the dream will not come true:
\begin{statement}
Even in quantum Ising model, the Gelfand-Naymark representation from the proof of the theorem (see Appendix) yields a non-separable Hilbert space when corresponds to an infinite lattice.
\end{statement}
\begin{proof}[Evidence]
Let's have a more careful look at how the representation is constructed. Let $A$ be a unital $C^*$-algebra, let's say. If $A$ is taken as the $C^*$-algebra corresponding to an infinite lattice in quantum Ising model, then $A$ is an AF-algebra (approximately finite). In particular, it is separable and infinite-dimensional. Now, to construct the representation, for every non-zero $z \in A$ we pick a representation $\pi_z$ such that $\|\pi_z(z)\xi_z\| = \|z\|$, where $\xi_z$ is the cyclic vector of $\pi_z$, and then we take the direct sum of those. Clearly, the sum is uncountable, for as a set the algebra $A$ is uncountable, so a basis of the resulting Hilbert space cannot be countable.
\end{proof}
So, the proof of Gelfand-Naymark theorem, even though more or less constructive, does not yield a way to construct a separable Hilbert space. The algorithm might be polished, I guess. Which $z \in A$ we might restrict to? Which are sufficient? The problem is that the choice of $z$'s is \emph{set-theoretic}, it's not \emph{functionally-analytic}. 

The $C^*$-algebras approach also has a downside that we lose unbounded observables. For example, in free fermions on infinite chain, if we don't normalize the Hamiltonian, the limit results in an unbounded operator, so the limiting $C^*$-algebra doesn't capture this. We've tried to use different normalizations, but this yielded either the zero operator or the identity, something trivial. One might tweak the eigen-values so that their absolute values are less than $1$; but then, what's the meaning of the limiting observable?

\emph{To sum up, the downside:} the limiting $C^*$-algebra loses both separability of the Hilbert space and does not contain unbounded observables like the total energy of the lattice.

I'd like to mention Segal's article \cite{segal}. From what I understood from his article, it's not necessary to find a faithful representation of the whole limiting $C^*$-algebra. There's the following result, \emph{the upside of the $C^*$-algebras approach}, which is a corollary of a more general statement that can be found in the article:
\begin{statement}
Let $A$ be a $C^*$-algebra and $u \in A$ be self-adjoint.Then for any $\alpha$ from the spectrum of $u$, there exists an irreducible representation $\phi$ of $A$ and a non-zero element $x$ of the space on which $A$ is represented such that $\phi(u)x = \alpha x$.
\end{statement}
The result is great in a sense that even the continuous spectrum of a self-adjoint element can be realized (at least partly) as a point spectrum. This reminds of the rigged Hilbert space approach that also gives a way to realize the continuum spectrum as a point one (through generalized eigen-values).


\begin{fur}
Segal in \cite{segal} mentions that it's actually not an issue that unbounded operators don't land in the limiting $C^*$-algebras, for they can be treated in terms of bounded operators. What did he mean?
\end{fur}
\section{Classical lattice models}
\section{Ising model}
\subsection{A general description of the IRF version}
There are two versions of the Ising model: the IRF (interaction-round-a-face) model and the vertex model. In the first one, the energy is assigned to vertices; in the second one, the energy is assigned to the bonds between the sites. 

Let $\Lambda \subseteq \mathbb Z^n$ be a subset of the integer lattice of dimension $n$. We associate with the lattice the space of microstates $\Omega_{\Lambda} := \{-1,+1\}^{\Lambda}$. Therefore, to each node $i \in \Lambda$ there corresponds a \emph{spin} $\omega_i = \pm 1$. For a finite $\Lambda$, the hamiltonian of the model is given by
\[
\mathcal H = \sum_{i,j \in \Lambda, \ i \sim j} \omega_i \omega_j - h \sum_{i \in \Lambda} \omega_i,
\]
where $h \in \mathbb R$ is some real number that corresponds to the external magnetic field, and $i \sim j$ means the nodes $i$ and $j$ are neighbors on the lattice. We also supply the model with the Gibbs measure defined previously.

\subsection{Transfer matrices in IRF model (not finished)}
To describe the transfer matrices, I restrict myself to a finite cubic lattice $\Lambda\subset \mathbb Z^2$ with periodic boundary conditions. Then we can assign energy to each face of the lattice:
\[
\epsilon(\text{face},\omega) := \sum_{i,j \in \text{face}, \ i \sim j} \omega_i \omega_j - h \sum_{i \in \text{face}} \omega_i.
\]
So the Hamiltonian breaks up into the sum of energies over all faces in $\Lambda$: 
\[
H(\omega) = \sum_{F \in \{\text{faces of }\Lambda\}} \epsilon(F,\omega).
\]
A \emph{Boltzmann weight} is the quantity $R(F,\omega):=\exp(-\beta \epsilon(F,\omega))$ assigned to a face $F$. The partition function can be rewritten as
\[
Z = \sum_{\omega \in \Omega}\prod_{F \in \text{faces}} R(F,\omega).
\]

\subsection{The vertex model and its transfer matrix}
I follow closely \cite{chari}. Let $\Lambda$ be an $n \times m$ cubic lattice in $\mathbb Z^2$ with periodic boundary conditions. The states are assigned to the bonds between vertices rather than to the vertices themselves in this model. Let $\Omega_0 = \{1,\ldots,n\}$ be the set of possible states of a single bond. For a picture of kind
\[
\xymatrix{
& &\\
& \bullet \ar@{-}[r]^{k} \ar@{-}[d]^{l} \ar@{-}[u]^{j} \ar@{-}[l]^{i} &\\
& &
}
\]
let $\varepsilon_{ij}^{kl}$ denote the energy assigned to the site in this setting. We assume that it doesn't depend on the position of the site but only on the states of the bonds around the sit. The Hamiltonian $\mathcal H$ of this model for a particular choice of the state of the lattice is then the sum of $\varepsilon_{ij}^{kl}$ over all vertices. The partition function is given by $Z = \sum_{\omega \in \Omega} \exp(-\beta H(\omega))$. A \emph{Boltzmann weight} is the quantity
\[
R_{ij}^{kl} := \exp(-\beta \varepsilon_{ij}^{kl}).
\]
\begin{proposition}
Let $V$ be an $m$-dimensional vector space. There exists an endomorphism $T \in \End(V \otimes V^{m})$, which is called a \emph{transfer matrix}, such that the partition function of the model is given by
\[
Z = \tr_{V^{\otimes m}} (\tr_{V} T)^{n}
\]
where the trace is the usual one (the sum of diagonal elements).
\end{proposition}
\begin{proof}
Consider a row in the cubic lattice, for a moment assuming that the boundary conditions on the ends (the states $i_1$ and $i_1^\prime$) may not be the same
\[
\xymatrix{
& & & & & & \\
& \bullet \ar@{-}[l]^{i_1}\ar@{-}[u]^{k_1}\ar@{-}[d]^{l_1} \ar@{-}[r]^{r_1} & \bullet \ar@{-}[u]^{k_2}\ar@{-}[d]^{l_2} \ar@{-}[rr]^{\cdots}&  & \bullet \ar@{-}[u]^{k_{m-1}}\ar@{-}[d]^{l_{m-1}} \ar@{-}[r]^{r_{m-1}} & \bullet\ar@{-}[u]^{k_m}\ar@{-}[d]^{l_m} \ar@{-}[r]^{i_1^\prime} & \\
& & & & & &
}
\]
Let us fix the end states $i_1$, $i_1^\prime$, $k_1,\ldots,k_m$ and $l_1,\ldots,l_m$. The contribution to $Z$ when only $r_i$'s are running over $\Omega_0$ is given by
\[
T^{i_1^\prime l_1 \cdots l_m}_{i_1 k_1 \cdots k_m} := \sum_{r_1,\ldots,r_{m-1}} R_{i_1k_1}^{r_1l_1}\cdots R_{r_{m-1}k_{m}}^{i_1^\prime l_m}.
\]
Let $V$ be an $m$-dimensional vector space spanned by some $e_1,\ldots,e_m$. Define an endomorphism $T \in \End(V \otimes V^{\otimes m})$ by setting on the basis elements
\[
T(e_{i_1}\otimes e_{k_1} \otimes \cdots \otimes e_{k_m}) = \sum_{i_1^\prime,l_1,\ldots,l_m} T^{i_1^\prime l_1 \cdots l_m}_{i_1 k_1 \cdots k_m} e_{i_1^\prime} \otimes e_{l_1} \otimes \cdots \otimes e_{l_m}.
\]
If wee unfreeze the endpoints with states $i_1$ and $i_1^\prime$ and let them run over $\Omega_0$, then we see that the contribution to $Z$ of the whole row (with still fixed states on the vertical bonds and now $i_1 = i_1^\prime$) is given by $\tr_V(T)_{k_1\ldots k_m}^{l_1\ldots l_m}$. Now, if the row was the first one and we consider the next one to it, and let $l_1,\ldots,l_m$ run over $\Omega_0$, then the contribution to $Z$ is
\[
\sum_{l_1,\ldots,l_m} \tr_V(T)_{k_1\ldots k_m}^{l_1\ldots l_m} \tr_V(T)_{l_1\ldots l_m}^{j_1\ldots j_m} = [(\tr_V(T))^2]_{k_1\ldots k_m}^{j_1\ldots j_m}
\]
(the last equality was not obvious to me due to a mess with indices, but it can be checked easily). Continuing in this fashion, the contribution to $Z$ with fixed states of the vertical bonds on the ends is given by  $[(\tr_V(T))^n]_{k_1\ldots k_m}^{l_1\ldots l_m}$. Now, applying the periodic condition $k_j = l_j$ and summing over all possible states of the ends, we finally find that $Z = \tr_{V^{\otimes m}} [\tr_V(T)]^n$.
\end{proof}
I think I can say that a transfer matrix is just a batch of all possible microstates of a row ingeniously packed into a linear endomorphism.
\section{Gaussian free field model}
I follow closely Chapter 8 from \cite{friedli}. In Gaussian free field model, the space of states at a single site is chosen to be $\Omega_0 := \mathbb R$. Accordingly, the space of states on a region $\Lambda \subseteq \mathbb Z^n$ is given by $\Omega_{\Lambda} := \mathbb R^{\Lambda}$. The Hamiltonian of the model is on the lattice $\Lambda$ is chosen to be
\[
\hh := \frac{\beta}{4n} \sum_{i \sim j, \ \{i,j\}\cap \Lambda \neq \emptyset} (\omega_i - \omega_j)^2 + \frac{m^2}{2} \sum_{i \in \mathbb Z^n} \omega_i^2,
\]
where $\beta$ is the inverse temperature, $\omega_i \in \Omega_0$ is the assigned spin at site $i \in \mathbb Z^n$, and $m$ is the mass.

A couple of comments on the choice of the Hamiltonian:
\begin{enumerate}[1)]
\item The factor $(\omega_i-\omega_j)^2$ tells us that the interaction favors the agreement of neighboring spins;
\item Since the space of states at single site is non-compact, we penalize large values of spin by adding the factor $m^2/2 \cdot \omega_i^2$ for each one;
\item Notice the condition under the first summation. It tells us that we also take into the account the boundary of $\Lambda$ (there might be different boundary conditions though).
\end{enumerate}
Fix a finite lattice $\Lambda \subset \mathbb Z^n$ and a state $\eta \in \Omega$ (it serves as a boundary condition for $\Lambda$). For a state $\omega_{\Lambda} \in \Omega_{\Lambda}$, by $\mathcal H(\omega_{\Lambda})$ we mean that we plug into the Hamiltionian the state that equals $\omega_{\Lambda}$ on $\Lambda$ and $\eta$ on the complement of $\Lambda$.

In Subsection \ref{ss:gen_princ} we specified a way of choosing $\sigma$-algebras on the spaces of states. Let $\sigma_{\mathbb Z^n}$ be such $\sigma$-algebra on the whole $\mathbb Z^n$. For $A \in \sigma_{\mathbb Z^n}$, the Gibbs measure in this model is defined as
\[
\mu(A) := \int_A \frac{e^{-\hh(\omega_{\Lambda})}}{Z} \prod_{i \in \Lambda} d\omega_i,
\]
where $d\omega_i$ is the Lebesgue measure on $\mathbb R$ assigned to the site $i \in \mathbb Z^n$ and $Z$ is the obviously chosen partition function. 

There's a way to define Gibbs measures for infinite $\Lambda$ as well (explained in \cite{friedli}, I postpone its description here for a moment). The case of massless GFF is drastically different from the case of massive GFF. For instance, Theorem 8.19 in \cite{friedli} says that there are no infinite-volume Gibbs measures in $n=1$ and $n=2$ cases. Nevertheless, Theorem 8.21 in the same reference tells us that there are infintely many infinite-volume Gibbs measures when $n \geq 3$. In the massive case, the GFF model has infinitely many infinite-volume Gibbs measures for any $n$ (see Theorem 8.28 in \cite{friedli}).
\subsection{$\text{O}(N)$-symmetric model}
I follow Chapter 9 from \cite{friedli}. In $\text{O}(N)$-model, we take $\Omega_0 := S^{N-1}$, so the spins might have an arbitrary direction. For a finite lattice $\Lambda \subseteq \mathbb Z^n$, the Hamiltonian (in the absence of a magnetic field) is usually written as
\[
\hh = - \beta \sum_{i\sim j, \ \ \{i,j\}\cap \Lambda \neq \emptyset} \left\langle \omega_i,\omega_j \right\rangle,
\]
where $\omega_i \in \Omega_0$ is a spin at site $i$, and the brackets denote the standard inner product in $\mathbb R^N$. For different $N$'s, we obtain some familiar models: for $N=1$ we have the Ising model; for $N=2$ we get the $XY$-model; and for $N=3$ we obtain the Heisenberg model.

The definition of finite-volume Gibbs measures is similar to the case of GFF model. At each site $i$, we have Lebesgue measure $d\omega_i$ on $S^{N-1}$. We fix a boundary condition, which is the choice of a state $\eta \in \Omega$, and then for measurable sets $A$ we set
\[
\mu(A) := \int_A \frac{e^{-\hh(\omega_{\Lambda})}}{Z} \prod_{i \in \Lambda} d\omega_i,
\]
where $Z$ is the obvious partition function and $\omega_{\Lambda} \in \Omega_{\Lambda}$; by $\hh(\omega_{\Lambda})$ I mean that we plug in a state equal to $\omega_{\Lambda}$ on $\Lambda$ and $\eta$ outside of $\Lambda$.

One might be interested in the following questions with regards to $\text{O}(N)$-models:
\begin{enumerate}[1)]
\item Is there an orientational long-range order? In my understanding, the mathematical formalism of this question is whether the correlations $\mathbb E_{\mu}\left\langle \omega_i, \omega_j \right\rangle$ converge to zero as $\|i-j\| \rightarrow \infty$;
\item Is there a spontaneous magnetization? The formalism in my understanding is: for any infinite-volume Gibbs measure $\mu$, is it true that $\lim_{n \rightarrow \infty} \left\langle \|m_{B(n)}\| \right\rangle_{\mu} \neq 0$? Here $B(n)$ is a cube of size $n$ and $m_{B(n)} := \frac{1}{|B(n)|} \sum_{i \in B(n)} \omega_i$ is the \emph{magnetization density}.

%the expectations $\left\langle \omega_i \right\rangle_\mu$ are equal to the same number for every $i$?
\end{enumerate}
The answers to both questions are negative for $N \geq 2$ and $n=1,2$. This is due to the following theorem, which can be also stated for a more general Hamiltonian:
%One of the questions for $\text{O}(N)$-models is about the \emph{orientational long-range order}. As I understand, it's formalized in the following way: the correlations $\mathbb E_{\mu}\left\langle \omega_i, \omega_j \right\rangle$ converge to zero as $\|i-j\| \rightarrow \infty$. The following theorem is claimed to provide a negative answer in some cases:

%I think that this can be formalized in the followi way that in the Van Hove limit $B(k) \Uparrow \mathbb Z^n$, where $B(k)$ is, say, a cube of size $k$, the corresponding Gibbs measures converge weakly to a measure $\mu$ such that $\left\langle \omega_i\right\rangle_{\mu}$ is the same non-zero number for all $i \in \mathbb Z^n$. The negative answer to this question in some settings is given by the following theorem (for a slightly more general Hamiltonian see \cite{friedli}, Chapter 9):

\begin{theorem}
(Mermin-Wagner) For $N \geq 2$ and $n=1,2$, all infinite-volume Gibbs measures are invariant under the action of the rotation group.
\end{theorem}

Maybe, I will write why the answers are negative a bit later. 

%Let's see why this theorem yields negative answers to the above questions. Assume $N \geq 2$ and $n=1$ or $n=2$.
%\begin{enumerate}[1)]
%\item Consider the long-range orientability problem. Then... 
%\item Now consider the spontaneous magnetization problem. The theorem actually says that each $\omega_i$ is distributed uniformly: if $A$ is measurable and $T \in \SO(N)$, then $\mu(\omega_i \in A) = (T\mu)(\omega_i \in A) = \mu(\omega_i \in T^{-1}(A))$. This implies that $\left\langle \omega_i \right\rangle_\mu = 0$.
%\end{enumerate}
%It turns out that for $n=1,2$ and $N \geq 2$, the expectation values of the spins are zero: $\left\langle \omega_i \right\rangle_{\mu} = 0$, even at a low temperature and for any measure $\mu$ (see \cite{friedli}).
\section{Questions}
\subsection{For a discussion}

\subsection{Just for myself}

In one article found this: the totality of numbers associated with a given observable is called its \emph{spectrum}. And indeed, if we regard the algebra of continuous functions as a Banach algebra, then the spectrum of anything is its image; in quantum mechanics, we indeed look at the eigenvalues, which from Banach algebras point of view form their spectra.

In Segal's article I saw that a product of two observables may not be an observable, so he proposed an algebraic structure where a multiplication is not assumed but only powers of elements.

Also from one article:
More recent studies have indicated that all self-adjoint operators may not be adequate as the model for the algebra of observables of every physical system. The $C*$-algebras are a long step from this special model, but still not into the chaos of abstract structures consistent with the general features of physical systems. 

From Kadison:
Given a state $f$ and an observable $A$, the value $f(A)$ is the expectation of the observable $A$ when these two have physical meanings.

\subsection{Notes from conversations}
\subsubsection{07/08}
We might want:
\begin{enumerate}
\item Prove that the limits for the Wilson loop taken in different orders yield different results;
\item Is there a gap between the ground state and the first excited state in QED 1+1?
\item Look at the papers of people who tried to prove these statements in other models.
\end{enumerate}
In free fermions, the gap is $O(N^{-1})$. \emph{Spontaneous symmetry breaking} means that the ground state is \emph{degenerate} (i.e., the corresponding eigen-space is of dimension bigger than 1). One can indeed encounter this phenomenon in nature. There are arguments from thermodynamics (not rigorous) about phase transitions (the most challenging problems).
\section{Appendix}
\subsection{$C^*$-algebras}
I think a few results from this theory are worth mentioning in the notes. Recall the definition of a $C^*$-algebra itself:
\begin{definition}
A (unital) $C^*$-algebra $A$ is a Banach space over $\mathbb C$ that is also a (unital) algebra such that the multiplication is a bounded bilinear map of norm $1$. It's also supplied with an involution $*:A\rightarrow A$ such that $\|a^*a\| = \|a\|^2$ for all $a \in A$. 
\end{definition}
A straightforward\footnote{From the boundedness of the multiplication we obtain $\|a\|^2 = \|a^*a\| \leq \|a^*\|\|a\|$, hence $\|a\| \leq \|a\|^*$. Switching to $a \mapsto a^*$, get $\|a\| = \|a\|^*$. For the unit, $\|1\| = \|1\|^2$ since $1^* = 1$.} consequence from the axioms is that $\|a\| = \|a^*\|$ and in the unital case $\|1\| = 1$.

The results of interest, I think, are the following:
\begin{theorem}
(Gelfand-Naymark, 1st) We have the following:
\begin{enumerate}[(i)]
\item Any possibly non-unital $C^*$-algebra is isomorphic to the space $C_0(\Omega)$ of continuous functions vanishing\footnote{Precisely, a function $\varphi : \Omega \rightarrow \mathbb C$ is vanishing at infinity if for any $\varepsilon >0$ there exists a compact subset $\Delta \subseteq \Omega$ such that $|\phi(t)|< \varepsilon$ when $t \notin \Delta$.} at $\infty$ on some locally compact Hausdorff topological space $\Omega$;
\item Any unital $C^*$-algebra is isomorphic to $C(\Omega)$ for $\Omega$ a compact Hausdorff space.
\end{enumerate}
\end{theorem}

There's an accurate description of both the isomorphism and the space $\Omega$: the space $\Omega$ is the Gelfand spectrum of $A$, and the isomorphism is the Gelfand representation.

Here are the definitions. Let $A$ be a Banach algebra\footnote{I.e. a Banach space that's also an algebra such that the multiplication is a bounded bilinear map with norm $1$.}. Its \emph{Gelfand spectrum} is the space $\Omega \subset A^*$ of all characters\footnote{A character is a non-zero linear functional that preserves the multiplication. It turns out that any character $\chi : A \rightarrow \mathbb C$ has norm $\|\chi\| \leq 1$, so they're automatically continuous, for if $\chi(a) = 1$ for some $a \in A$ with $\|a\| <1$, then let $b:=\sum_{k=1}^\infty a^k$. It follows that $a + ab = b$, hence $1 + \chi(b) = \chi(b)$, which is absurd. } of $A$  that is endowed with $\omega^*$-topology induced from $A^*$. The isomorphism then is the Gelfand representation $\Gamma_A : A \rightarrow C_0(\Omega)$ that sends an element $a \in A$ to a continuous map (vanishing at $\infty$) $\hat a$ such that\footnote{The continuity of $\hat a$ follows right from the definition of $\omega^*$-topology. It indeed vanishes at $\infty$, for $\hat a$ is a continuous map on the compactification $\Omega^\prime := \Omega \cup \{0\}$ such that $\hat{a}(0) = 0$ (look at the formula defining $\hat{a}$), hence for any $\varepsilon > 0$ there's $U_{\varepsilon} \ni 0$ such that $|\hat{a}(\chi)| < \varepsilon$ when $t \in U_{\varepsilon}$. But $U_{\varepsilon}^c$ is compact, as a closed subset of the compact space $\Omega^\prime$, so $\hat a$ vanishes at $\infty$.}
\begin{proposition}
The Gelfand spectrum $\Omega$ of a Banach algebra $A$ is a locally compact Hausdorff space; if $A$ is in addition unital, then $\Omega$ is Hausdorff and compact.
\end{proposition}
\begin{proof}
Obviously, $\Omega$ is Hausdorff since $\omega^*$-topology is Hausdorff. Consider $\Omega^\prime := \Omega \cup \{0\}$. If $\chi_\alpha \rightarrow f$ in $\omega^*$-topology, then obviously $f$ preserves multiplication (but it might become zero), so $\Omega^\prime$ is a closed subset of the unit ball in $A^*$. Now it's the consequence of Banach-Alaoglu theorem that $\Omega^\prime$ is compact. Being a subset of a compact Hausdorff space, we see that $\Omega$ is locally compact and Hausdorff. In case $A$ is unital, $0$ is an isolated point of $\Omega^\prime$, for if there was a net $\chi_\alpha \in \Omega$ such that $\chi_\alpha \rightarrow 0$, then $1 = \chi_\alpha(1) \rightarrow 0$, which is a contradiction, and thus $\Omega$ is compact.
\end{proof}

In general, for possibly non-commutative $C^*$-algebras we have 
\begin{theorem}
(Gelfand-Naymark, 2nd) Any $C^*$-algebra is $*$-isometrically isomorphic to $\End(H)$ for some Hilbert space $H$.
\end{theorem}
\begin{fur}
Might understand the construction of this operator algebra in detail. I've seen that the proof is constructive. That's the matter of finding pure states and defining the corresponding irreducible representations. The direct sum of those will yield the result.
\end{fur}

\begin{thebibliography}{10}
\bibitem{arveson} Arveson, William. An invitation to C*-algebras. Vol. 39. Springer Science \& Business Media, 2012.
\bibitem{bratteli} Bratteli, Ola. Inductive limits of finite-dimensional $C^*$-algebras. Transactions of the American Mathematical Society 171 (1972): 195-234.\label{bratteli}
\bibitem{chari} Chari, Vyjayanthi, and Andrew N. Pressley. A guide to quantum groups. Cambridge university press, 1995.
\bibitem{conway} Conway, John B. A course in functional analysis. Vol. 96. Springer, 2019.
\bibitem{friedli} S. Friedli and Y. Velenik. Statistical Mechanics of Lattice Systems: A Concrete Mathematical Introduction. Cambridge University Press, 2017.
\bibitem{haag} Haag, Rudolf, R. V. Kadison, and Daniel Kastler. "Nets of $C^*$-algebras and classification of states." Communications in Mathematical Physics 16.2 (1970): 81-104.
\bibitem{israel} Israel, Robert B. Convexity in the theory of lattice gases. Vol. 64. Princeton University Press, 2015.
\end{thebibliography}
\end{document}