\documentclass[11pt]{article}%{amsart}%{article}
\usepackage{braket}
\renewcommand{\baselinestretch}{1.05}
\topmargin0.0cm
\headheight0.0cm
\headsep0.0cm
\oddsidemargin0.0cm
\textheight23.0cm
\textwidth16.5cm
\footskip1.0cm
%\usepackage[russian]{babel} %proof'ы становятся доказательствами, если включить
\usepackage{amsmath}%
\usepackage{amsfonts}%
\usepackage{amssymb}%
\usepackage{graphicx}
%\documentclass[a4paper,12pt]{article}
%вот эти три отвечают каким-то образом за могучий русский язык
\usepackage[T2A]{fontenc}
%\usepackage[utf8]{inputenc}
\usepackage[cp1251]{inputenc}
%\usepackage{braket}
\usepackage{showlabels}

%\usepackage{enumitem}
%\setlist[enumerate]{label*=\arabic*.}

\usepackage{amsthm} %for theorem styles
\usepackage{hyperref} %for reference
\usepackage{enumerate}
%\usepackage{mbboard}

\usepackage[all,cmtip]{xy}

%\usepackage[small,nohug,heads=littlevee]{diagrams}
%\diagramstyle[labelstyle=\scriptstyle]


\DeclareMathOperator{\Ker}{Ker} 
\DeclareMathOperator{\supp}{supp}
\DeclareMathOperator{\Imm}{Im} 
\DeclareMathOperator{\cl}{cl}
%\DeclareMathOperator{\Sp}{Sp} %линейная оболочка
\DeclareMathOperator{\Spz}{\overline{Sp}}
\DeclareMathOperator{\Ree}{Re}
\DeclareMathOperator{\Int}{int}
\DeclareMathOperator{\Dom}{dom}
\DeclareMathOperator{\Sp}{span}
\DeclareMathOperator{\icr}{icr}
\DeclareMathOperator{\co}{co}
\DeclareMathOperator{\re}{Re}
\DeclareMathOperator{\id}{id}
\DeclareMathOperator{\Aut}{Aut}
\DeclareMathOperator{\Alt}{Alt}
\DeclareMathOperator{\Inner}{Inn}
\DeclareMathOperator{\lcm}{lcm}
\DeclareMathOperator{\res}{res}
\DeclareMathOperator{\card}{card}
\DeclareMathOperator{\trace}{trace}
\DeclareMathOperator{\Min}{Min}
\DeclareMathOperator{\Gal}{Gal}
\DeclareMathOperator{\charr}{char}
\DeclareMathOperator{\su}{su}
\DeclareMathOperator{\sll}{sl}
\DeclareMathOperator{\Ad}{Ad}
\DeclareMathOperator{\ad}{ad}
\DeclareMathOperator{\Tor}{Tor}
\DeclareMathOperator{\Tr}{Tr}
\DeclareMathOperator{\tr}{tr}
\DeclareMathOperator{\trdeg}{trdeg}
\DeclareMathOperator{\sgn}{sgn}
\DeclareMathOperator{\End}{End}
\DeclareMathOperator{\Ext}{Ext}
\DeclareMathOperator{\Ann}{Ann}
\DeclareMathOperator{\Mat}{Mat}
\DeclareMathOperator{\diag}{diag}
\DeclareMathOperator{\GL}{GL}
\DeclareMathOperator{\gl}{gl}
\DeclareMathOperator{\rank}{rank}
\DeclareMathOperator{\Ric}{Ric}
\DeclareMathOperator{\so}{so}
\DeclareMathOperator{\Hess}{Hess}
\DeclareMathOperator{\sff}{\mathrm{I\!I}}
\DeclareMathOperator{\pr}{pr}
\DeclareMathOperator{\Supp}{Supp}
\DeclareMathOperator{\Spec}{Spec}
\DeclareMathOperator{\codim}{codim}
\DeclareMathOperator{\Hom}{Hom}
\DeclareMathOperator{\Cl}{Cl}
\DeclareMathOperator{\SO}{SO}

%%%%%%%%%%%%%%
%% МАКРОСЫ
%%%%%%%%%%%%%%
\newcommand{\dif}{\left.\frac{d}{dt}\right|_{t=0}}
\newcommand{\difs}{\left.\frac{d}{ds}\right|_{s=0}}
\newcommand{\g}{\mathfrak g}
\newcommand{\h}{\mathfrak h}
\newcommand{\hh}{\mathcal{H}}
\newcommand{\frb}{\mathfrak b}
\newcommand{\pprime}{{\prime\prime}}

\newcommand{\Mod}[1]{\ \left(\mathrm{mod}\ #1\right)}
%------------------------------------------------------------
\newtheorem{myth}{Theorem}
\newtheorem{theorem}{Theorem}
\theoremstyle{definition}
\newtheorem{example}{Example}
\theoremstyle{definition}
\newtheorem{myexample}{Example}
\newtheorem{corollary}{Corollary}[theorem]
\newtheorem{mycor}{Corollary}[myth]
\theoremstyle{definition}\newtheorem{definition}{Definition}
%\theoremstyle{definition}\newtheorem{definition}{Определение}
\theoremstyle{definition} \newtheorem{cntex}{Counterexample}
\theoremstyle{definition}\newtheorem*{acknowledgement}{Acknowledgement}
\newtheorem{lemma}{Lemma}
\newtheorem{problem}{Problem}
\newtheorem{proposition}{Proposition}
\newtheorem{corp}{Corollary}[proposition]
\theoremstyle{definition}\newtheorem{remark}{Remark}
\theoremstyle{definition}\newtheorem{quest}{Question}
\numberwithin{equation}{section}
\newtheorem{exercise}{Exercise}
\newtheorem{fur}{Further Search}
\newtheorem{ths}{Thoughts Aloud}
\newtheorem*{statement}{Statement}
\newtheorem*{idea}{Idea}
%--------------------------------------------------------
\begin{document}
\tableofcontents
\section{Preliminaries}
\subsection{The main principles of classical lattice models}
Let $\Omega$ be a finite set (the set of \emph{microstates}), let $\hh : \Omega \rightarrow \mathbb R$ be a \emph{hamiltonian}, a specifically chosen random variable. Let $\mathcal M(\Omega)$ be the space of probability measures on $\Omega$. In this theory, the expectation of a random variable $f : \Omega \rightarrow \mathbb R$ with respect to a measure $\mu$ (if not implicitly understood; it's also called the \emph{thermal average}) is denoted as
\[
\left\langle f \right\rangle_{\mu} := \mathbb E_{\mu} f = \int_{\Omega} f d\mu.
\]
One of the first questions in statistical mechanics is devoted to the choice of the right measure $\mu$. The choice is governed using Shannon's entropy $S : \mathcal M(\Omega) \rightarrow \mathbb R$, defined as $S(\mu) := -\int_{\Omega}x\log x d\mu(x)$ (there's a way to understand why $S$ has this form; see \cite{friedli}). \emph{The maximum entropy principle says}: for a given model of statistical mechanics, choose $\mu$ that maximizes $S$. For example, if there's no other information avaiable about the system, then the measure that maximizes $S$ is the uniform distribution. There are other two typical situations:
\begin{enumerate}[(a)]
\item If we know that $\left\langle \mathcal H\right\rangle = U$ for some fixed $U$ (the \emph{internal energy} of the system), then the measure that maximizes $S$ is 
\[
\mu(\omega) = \frac{e^{-\beta H(\omega)}}{Z}, \ \ Z:= \sum_{\omega \in \Omega} e^{-\beta \mathcal H(\omega)} 
\]
which is the \emph{Gibbs measure}. This corresponds to the situation when we know the system exchanges its energy with some external thermal reservoir. The result can be obtained using Lagrange multipliers; the parameter $\beta$, called the \emph{inverse temperature}, is uniquely determined by $U$ (and vice versa: $U$ is uniquely determined by $\beta$). In the theory, $\beta = (kT)^{-1}$, where $k$ is the Boltzmann constant and $T$ is the temperature of the system.
\item If we know additionally that the system exchanges its particles $\mathcal N$ with the external environment, and that the expected value of the particles if $\left\langle \mathcal N \right\rangle = N$, then we obtain a \emph{grand canonical Gibbs distribution} via a similar procedure. It's given by
\[
\mu(\omega) := \frac{e^{-\beta(\hh(\omega)-\mu N)}}{Z}, \ \ Z := \sum_{N} e^{\beta \mu N} \sum_{\omega \in \Omega} e^{-\beta \hh (\omega)}.
\]
The parameter $\mu$ is identified with the so-called \emph{chemical potential}.
\end{enumerate}

\subsection{The thermodynamic limit}

\subsection{Quantum lattices}
\subsubsection{A general set-up}
I will follow closely the treatment in \cite{israel}. Again, we have a lattice $X \subset \mathbb Z^n$, but to each site $i \in X$ we attach a copy of a finite-dimensional Hilbert space $H_i$. To a finite $X$ we attach the tensor product $H_X := \otimes_{i \in X} H_i$. %For example, in spin-$1/2$ systems each Hilbert space is two-dimensional, with an orthonormal basis corresponding to spins up and down.

For infinite lattices, the author of \cite{israel} suggests proceeding as follows. Let $\mathfrak{A}_X := \End(H_X)$, and for any two finite subsets $X \subseteq Y \subset \mathbb Z^n$, let $\imath : \mathfrak{A}_X \rightarrow \mathfrak{A}_Y$ be the inclusion that sends $A$ to $A \otimes 1$ (where $1$ is viewed as an endomorphism of $\mathfrak{A}_{Y\setminus X}$). For an infinite $\Lambda \subseteq \mathbb Z^n$, the family of all its finite subsets with the inclusions form a direct system. Let $\mathfrak A_{\Lambda} := \varinjlim \mathfrak{A}_X$ be the direct limit taken in the category of $C^*$-algebras over all finite subsets of $\Lambda$. In the literature, this algebra is known as an AF (approximately finite-dimensional) $C^*$-algebra. The first reference in this theory goes back to Bratteli \cite{bratteli}. See the next subsection for an elaboration on the inductive limit.

Further, for a finite $\Lambda \subset \mathbb Z^n$ and a Hamiltonian $\mathcal H_{\Lambda}$, the partition function is defined as
\[
Z = \tr_{\Lambda} e^{-\beta \mathcal H_{\Lambda}}
\]
and the expectation of an observable $A \in \mathfrak A_{\Lambda}$ is 
\[
\left\langle A \right\rangle_{\Lambda} := Z^{-1} \tr_{\Lambda} (A e^{-\mathcal H_{\Lambda}}).
\]
The trace in these formulas is normalized: it's $1/d$ of the usual trace, where $d$ is the dimension of the Hilbert space at one site. An interesting consequence of such normalization is that $\tr$ extends than to a norm-one linear functional on the whole $\mathfrak A:= \mathfrak A_{\mathbb Z^n}$ (see \cite{israel}). The Hamiltonian they choose is given by
\[
\mathcal H_{\Lambda} = \sum_{X \subseteq \Lambda} \Phi(X),
\]
where $\Phi$ is a so-called \emph{interaction}: it's a function from the non-empty finite subsets of $\mathbb Z^n$ to self-adjoint operators on them, such that $\Phi(X+i) = \Phi(X)$ for any $i \in \mathbb Z^n$ (i.e., it's translational invariant).

The pressure for a finite region $\Lambda$ in the quantum lattice system is given by
\[
P_{\Lambda}(\Phi) := |\Lambda|^{-1} \ln \tr e^{-H_{\Lambda}}.
\]
One can show that the limit in the sense of van Hove of $P_{\Lambda}$ does exist in the quantum setting as well (\cite{israel}).

%\begin{fur}
%Not yet have I delved into the properties of AF $C^*$-algebras. How is the norm defined there at least? Not yet clear.
%\end{fur}
%
%\begin{fur}
%It's said in \cite{bratteli} that one might refer to \cite{haag} for limits of lattices in case the number of states is infinite.
%\end{fur}
\subsubsection{The inductive limit in more detail}
For two finite subsets $X \subseteq Y \subset \mathbb Z^n$, the inclusion $\imath : \mathfrak A_{X} \rightarrow \mathfrak A_{Y}$ that sends $A$ to $A \otimes 1$ is injective; therefore, whenever $X \subseteq Y$, we can view $\mathfrak A_{X}$ as a subalgebra of $\mathfrak A_{Y}$. Hence we can take a union of all such subalgebras coming from finite subsets of $\Lambda \subseteq \mathbb Z^n$, and then, to be safe and ensure it's a Banach space, take the closure. So, one can identify (see a proposition below for a rigorous proof)
\[
\mathfrak A_{\Lambda} = \varinjlim \mathfrak A_{X} = \cl \left(\bigcup_{X \subset \Lambda, \ |X| < \infty} \mathfrak A_{X}\right)/_{u \sim u \otimes 1}
\]
From this point of view, it's easy to understand what the norm is. For $A$ from the dense subspace (the union itself), we just set $\|A\|_{\Lambda} := \|A\|_{X}$ if $A \in \mathfrak A_{X}$. The norm extends to the closure by the very process of completeness: for $A \in \mathfrak A_{\Lambda}$, we choose a sequence $A_n \in \mathfrak A_{X_n}$ and then set $\|A\|_{\Lambda} := \lim_{n \rightarrow \infty} \|A_n\|$.

From Appendix on $C^*$ algebras, we see that Gelfand-Naymark theorem ensures there is a Hilbert space $H$ such that $\mathfrak A_{\Lambda} \cong \End(H)$.
\begin{fur}
I can elaborate on the construction of this Hilbert space. It's more or less constructive and relies on finding pure states. In particular, it would be interesting to see how this $H$ is related to the infinite tensor product $\otimes_{i \in \Lambda} H_i$: what exactly goes wrong?
\end{fur}
\begin{proposition}
In the above set-up, we indeed have $\varinjlim \mathfrak A_{X} = \cl \left(\bigcup_{X \subset \Lambda, \ |X| < \infty} \mathfrak A_{X}\right)/_{u \sim u \otimes 1}$ (isometrically and preserving the $\ast$-structure).
\end{proposition}
\begin{proof}
Denote $\mathfrak A^\prime := \left(\bigcup_{X \subset \Lambda, \ |X|<\infty} \mathfrak A_X\right)/_{u \sim u \otimes 1}$.
So, we choose morphisms in the category of unital $C^*$-algebras as bounded unital $\ast$-homorphisms with norm less then or equal to one\footnote{Otherwise I don't think there's a way to prove that the map induced on the diagram of the injective limit is a bounded operator}. To prove the statement, all we need to show is that for a unital $C^*$-algebra $A$ and a bunch of morphisms $\alpha_X : \mathfrak A_X \rightarrow \mathfrak A_Y$ where $X \subseteq Y$ and such that $\alpha_X(u) = \alpha_Y(u \otimes 1)$ (but remember that $u$ is identified with $u \otimes 1$ in the union), there's a unique morphism $\alpha : \cl \mathfrak A^\prime \rightarrow A$. In the language of diagrams, this is saying that
\[
\xymatrix{
& & A & & \\
& &  & & \\
& & \cl \mathfrak A^\prime\ar@{-->}[uu]_{\exists ! \alpha} & &\\
\mathfrak A_X\ar[rrrr]^{u \mapsto u \otimes 1} \ar[rru]^{\imath_X} \ar[rruuu]^{\alpha_X} & & & & \mathfrak A_Y\ar[lluuu]_{\alpha_Y} \ar[llu]_{\imath_Y}
}
\]
Once $\alpha$ is defined on $\mathfrak A^\prime$ with all the mentioned properties, it automatically extends to the closure. So, for $u \in \mathfrak A_X$ we set $\alpha(u) := \alpha_X(u)$. This is well defined, for $u$ is identified with $u \otimes 1$ in the union. We get automatically that $\alpha$ is a unital $\ast$-homorphism since all $\alpha_X$'s are. It's norm is bounded by $1$, for $\|\alpha(u)\| \leq \|\alpha_X\|\|u\| \leq \|u\|$. Thus $\alpha$ is a morphism in the corresponding category.
\end{proof}

\subsection{Relation between classical and quantum lattices}
I follow \cite{israel} with some minor modifications more appealing to my taste. Let $\Omega_0$ be a finite set of microstates at one site, and let $H_0$ be a Hilbert space of dimension equal to $|\Omega|$ (which is assigned to one site as well). Let $C(\Omega)$ be the space of observables on $\Omega$. Choose an orthonormal basis $e_{\mu}$ of $H_0$ labeled my microstates $\mu \in \Omega_0$. Then we have an injection $\imath : C(\Omega_0) \rightarrow \End(H_0)$ given by
\[
[\imath(f)](e_{\mu}) := e_{f(\mu)}.
\]
In other words, the classical observables are embedded into the quantum observables as diagonal matrices.

\subsection{Continuous spins: general principles}\label{ss:gen_princ}
I follow closely Section 6.10 of \cite{friedli}. In case the space of states $\Omega_0$ at a single site is non-compact, the existence of Gibbs measures is no longer guaranteed. For $\Omega_0$ a topological space, one defines the following ingredients. Let $\mathcal B_0$ be the Borel $\sigma$-algebra on $\Omega_0$. For a finite lattice $\Lambda \subset \mathbb Z^n$, we supply the space of states with the $\sigma$-algebra $\mathcal B_{\Lambda} := \bigotimes_{i \in \Lambda} \mathcal B_0$. The natural projections $\pi_{\Lambda} : \Omega \rightarrow \Omega_{\Lambda}$ allow us to define a $\sigma$-algebra on $\Omega$ with base in $\Lambda$:
\[
\sigma_{\Lambda} := \pi^{-1}_{\Lambda}(\mathcal B_{\Lambda}).
\]
If $S \subseteq \mathbb Z^n$ is a possibly infinite lattice, then we supply it with the $\sigma$-algebra
\[
\sigma_{S} := \sigma(\bigcup_{\Lambda \subset S, \ \Lambda \text{ finite}} \sigma_{\Lambda})
\]
(by the last equality I mean the smallest $\sigma$-algebra generated by the union).
\section{The Main Stage}
\subsection{Statement of the problem and ideas}
The current statement, I guess, is the following:

\begin{statement}
Consider $\mathbb Z^n$ where to each node we attach an infinite-dimensional separable Hilbert space $H$. Let $\Lambda \subseteq \mathbb Z^n$ be an infinite sublattice. Consider the limit
\[
\mathfrak A_{\Lambda} := \varinjlim_{X \subset \Lambda, \ \ X\ \text{finite}} \mathfrak A_{X}.
\]
If it turns out that $\mathfrak A_{\Lambda}$ is a $C^*$-algebra, I'd like to do the following: find an ideal $J \triangleleft \mathfrak A_{\Lambda}$ such that the Hilbert space $H$ promised by the Gelfand-Naymark theorem is separable; i.e., $\mathfrak A_{\Lambda}/J \cong \End(H)$ for $H$ separable. It would be also nice to keep embeddings $\mathfrak A_{X} \rightarrow \mathfrak A_{\Lambda}$ for finite sublattices $X \subset \Lambda$.
\end{statement}

\begin{fur}
Given a unital $C^*$-algebra $A$, under which conditions on $A$ the Hilbert space given by Gelfand-Naymark theorem is separable?
\end{fur}

\begin{idea}
To keep everything physically meaningful, I think that $J$ can be tried out as the ideal generated by operators with all but finitely many eigen-states concentrated in a finite sublattice of $\Lambda$. Here, I need to refresh my mind with regards to eigen-states.
\end{idea}

\subsection{Tests on quantum Ising model}
In Ising model, to each node of $\mathbb Z^2$ we attach a $2$-dimensional Hilbert space $H$ with some a priori chosen orthonormal basis $e_1,e_2$. It corresponds to the states \emph{spin up} and \emph{spin down}.
\subsubsection{The issue with the infinite tensor product}
Let's consider first the algebraic tensor product $H_{\infty}$ of all Hilbert spaces attached to all sites. It's spanned by simple tensors of the form
\[
e_{\lambda(1)} \otimes e_{\lambda(2)} \otimes e_{\lambda(3)} \otimes \cdots 
\]
where $\lambda : \mathbb N \rightarrow \{e_1,e_2\}$ is a function. There are as my such simple tensors as functions $\lambda$; the cardinal number is equal to $|2^{\mathbb N}| = |\mathbb R|$, i.e., there are uncountably many of them\footnote{More details on why it's uncountable. Any number $a \in \mathbb R$ can be represented as a power series $a = \sum_{i=-\infty}^\infty c_i 2^{i}$, where $c_i \in \{0,1\}$ and only finitely many $c_i$ for $i > 0$ might be non-zero. Restrict ourselves to $a = \sum_{i=1}^{\infty}c_i 2^{-i}$. Then we have a 1-1 correspondence between functions $\lambda : \mathbb N \rightarrow \{0,1\}$ and such numbers. }. It's natural to declare such simple tensors an orthonormal basis of $H_{\infty}$. But then, it's a result of metric spaces theory that if there are uncountably many points such that the distance between any of two is bounded by a positive constant (that doesn't depend on the points), then the space is not separable. This is the case with $H_{\infty}$. Since it's not separable, its completion $\cl H_{\infty}$ can't be separable as well. By the way, the same idea is used when one proves $l_{\infty}$ is not separable.

Just out of curiosity, the same cardinal number occurs when all Hilbert spaces are infinite-dimensional but separable. In this case, we deal with all functions $\lambda : \mathbb N \rightarrow \mathbb N$; the cardinal number of them is again ${|\mathbb N}^{\mathbb N}| = |\mathbb R|$.

%\subsubsection{The limit of algebras of observables in combination with Gelfand-Naymark theorem}
\subsubsection{The limit and G-F theorem}

For an infinite sublattice $\Lambda \subseteq \mathbb Z^n$, the injective limit $\varinjlim_{X \subset \Lambda, \ X \ \text{finite}} \mathfrak A_{X}$ can be thought of as the completion of the union of those subalgebras. 

The following example explains how the union works.
\begin{example}The algebra $\mathfrak A_{1}$ attached to a single site can be identified with the algebra of $2\times 2$ matrices over $\mathbb C$. For two nodes, the algebra $\mathfrak A_{2}$ can be identified with matrices of dimension $4 \times 4$. The embedding $\mathfrak A_{1} \rightarrow \mathfrak A_{2}$ then does the following:
\[
A:=\begin{pmatrix} a_{11} & a_{12}\\ a_{21} & a_{22}\end{pmatrix}
\mapsto A \otimes 1 = \begin{pmatrix} A & 0 \\ 0 & A\end{pmatrix}.
\]
\end{example}

Therefore, for an inifnite sublattice $\Lambda$, the algebra $\mathfrak A_{\Lambda}$ can be identified with the space of matrices of infinite size such that only finitely many entries of each of them are non-zero. \emph{Note} the difference with the attempt to take an infinite tensor product of Hilbert spaces: these infinite matrices naturally act upon the space 
\[
T:=\bigoplus_{n=1}^\infty \, \bigotimes_{i \in A \subset \Lambda, \ |A| = n} H_i 
\]
It's easy to see that this space has an infinite countable basis. This implies that, whatever norm we put on $T$, the space will not be complete (that's a standard result from functional analysis: in a Banach space, a vector space basis is at least uncountable).

\textbf{Way 1 (just a fantasy)}. Let $\mathfrak A_{\Lambda}^\prime$ be the union of all $\mathfrak A_X$ for $X \subset \Lambda$ and $X$ finite. We can substitute the norm on $\mathfrak A_{\Lambda}^\prime$ with the Hilbert-Schmidt norm (see appendix), and then complete $\mathfrak A_{\Lambda}^\prime$ with respect to it. The elements $A$ of the resulting space can be represented as infinite matrices $(a_{nk})_{k,n = 1}^{\infty}$ such that $\sum_{n,k} |a_{nk}|^2 < \infty$. We can act with these on a completion of $T$. However, it's not a $C^*$-algebra; it is a Banach algebra though. There might be something in this approach.

\textbf{Way 2.} We can complete with respect to the operator norm. I can't prove this, but the evidence is that we obtain the space of compact operators on $\cl T$. That's very good. If our Hamiltonian is normalized in such a way that in the limit it gives a bounded operator, then we can employ the Hilbert-Schmidt theorem and find an orthonormal basis in $\cl T$ of eigen-values of the limiting Hamiltonian. 

%\begin{fur}
%I see so far two ways one can go here, I need to delve into both of them. One way leads to Hilbert-Schmidt operators (if we had a different norm at the begininning), the other (assuming the operator norm), I guess, leads to compact operators. I think we need to study both these cases. Reference \cite{conway} tells how to realize the first way (page 268).
%\end{fur}

%
%Therefore, the union is made of infinite matrices with the property that each contains a submatrix of a finite dimension outside of which the matrix entries equal to zero.
%
%\begin{fur}
%I haven't understood yet how the completion of such a space looks like. In the process of looking this up and at the same time I'm pondering over this.
%\end{fur}

\subsection{Thoughts on $C^*$-algebras approach}
So the idea was the following: since it's not sometimes clear what a limiting Hilbert space should look like, we can take the limit of the corresponding algebras of observables and then, by Gelfand-Naymark theorem, find an underlying Hilbert space, hopefully a separable one. But the dream will not come true:
\begin{statement}
Even in quantum Ising model, the Gelfand-Naymark representation from the proof of the theorem (see Appendix) yields a non-separable Hilbert space when corresponds to an infinite lattice.
\end{statement}
\begin{proof}[Evidence]
Let's have a more careful look at how the representation is constructed. Let $A$ be a unital $C^*$-algebra, let's say. If $A$ is taken as the $C^*$-algebra corresponding to an infinite lattice in quantum Ising model, then $A$ is an AF-algebra (approximately finite). In particular, it is separable and infinite-dimensional. Now, to construct the representation, for every non-zero $z \in A$ we pick a representation $\pi_z$ such that $\|\pi_z(z)\xi_z\| = \|z\|$, where $\xi_z$ is the cyclic vector of $\pi_z$, and then we take the direct sum of those. Clearly, the sum is uncountable, for as a set the algebra $A$ is uncountable, so a basis of the resulting Hilbert space cannot be countable.
\end{proof}
So, the proof of Gelfand-Naymark theorem, even though more or less constructive, does not yield a way to construct a separable Hilbert space. The algorithm might be polished, I guess. Which $z \in A$ we might restrict to? Which are sufficient? The problem is that the choice of $z$'s is \emph{set-theoretic}, it's not \emph{functionally-analytic}. 

The $C^*$-algebras approach also has a downside that we lose unbounded observables. For example, in free fermions on infinite chain, if we don't normalize the Hamiltonian, the limit results in an unbounded operator, so the limiting $C^*$-algebra doesn't capture this. We've tried to use different normalizations, but this yielded either the zero operator or the identity, something trivial. One might tweak the eigen-values so that their absolute values are less than $1$; but then, what's the meaning of the limiting observable?

\emph{To sum up, the downside:} the limiting $C^*$-algebra loses both separability of the Hilbert space and does not contain unbounded observables like the total energy of the lattice.

I'd like to mention Segal's article \cite{segal}. From what I understood from his article, it's not necessary to find a faithful representation of the whole limiting $C^*$-algebra. There's the following result, \emph{the upside of the $C^*$-algebras approach}, which is a corollary of a more general statement that can be found in the article:
\begin{statement}
Let $A$ be a $C^*$-algebra and $u \in A$ be self-adjoint.Then for any $\alpha$ from the spectrum of $u$, there exists an irreducible representation $\phi$ of $A$ and a non-zero element $x$ of the space on which $A$ is represented such that $\phi(u)x = \alpha x$.
\end{statement}
The result is great in a sense that even the continuous spectrum of a self-adjoint element can be realized (at least partly) as a point spectrum. This reminds of the rigged Hilbert space approach that also gives a way to realize the continuum spectrum as a point one (through generalized eigen-values).


\begin{fur}
Segal in \cite{segal} mentions that it's actually not an issue that unbounded operators don't land in the limiting $C^*$-algebras, for they can be treated in terms of bounded operators. What did he mean?
\end{fur}
\section{Classical lattice models}
\section{Ising model}
\subsection{A general description of the IRF version}
There are two versions of the Ising model: the IRF (interaction-round-a-face) model and the vertex model. In the first one, the energy is assigned to vertices; in the second one, the energy is assigned to the bonds between the sites. 

Let $\Lambda \subseteq \mathbb Z^n$ be a subset of the integer lattice of dimension $n$. We associate with the lattice the space of microstates $\Omega_{\Lambda} := \{-1,+1\}^{\Lambda}$. Therefore, to each node $i \in \Lambda$ there corresponds a \emph{spin} $\omega_i = \pm 1$. For a finite $\Lambda$, the hamiltonian of the model is given by
\[
\mathcal H = \sum_{i,j \in \Lambda, \ i \sim j} \omega_i \omega_j - h \sum_{i \in \Lambda} \omega_i,
\]
where $h \in \mathbb R$ is some real number that corresponds to the external magnetic field, and $i \sim j$ means the nodes $i$ and $j$ are neighbors on the lattice. We also supply the model with the Gibbs measure defined previously.

\subsection{Transfer matrices in IRF model (not finished)}
To describe the transfer matrices, I restrict myself to a finite cubic lattice $\Lambda\subset \mathbb Z^2$ with periodic boundary conditions. Then we can assign energy to each face of the lattice:
\[
\epsilon(\text{face},\omega) := \sum_{i,j \in \text{face}, \ i \sim j} \omega_i \omega_j - h \sum_{i \in \text{face}} \omega_i.
\]
So the Hamiltonian breaks up into the sum of energies over all faces in $\Lambda$: 
\[
H(\omega) = \sum_{F \in \{\text{faces of }\Lambda\}} \epsilon(F,\omega).
\]
A \emph{Boltzmann weight} is the quantity $R(F,\omega):=\exp(-\beta \epsilon(F,\omega))$ assigned to a face $F$. The partition function can be rewritten as
\[
Z = \sum_{\omega \in \Omega}\prod_{F \in \text{faces}} R(F,\omega).
\]

\subsection{The vertex model and its transfer matrix}
I follow closely \cite{chari}. Let $\Lambda$ be an $n \times m$ cubic lattice in $\mathbb Z^2$ with periodic boundary conditions. The states are assigned to the bonds between vertices rather than to the vertices themselves in this model. Let $\Omega_0 = \{1,\ldots,n\}$ be the set of possible states of a single bond. For a picture of kind
\[
\xymatrix{
& &\\
& \bullet \ar@{-}[r]^{k} \ar@{-}[d]^{l} \ar@{-}[u]^{j} \ar@{-}[l]^{i} &\\
& &
}
\]
let $\varepsilon_{ij}^{kl}$ denote the energy assigned to the site in this setting. We assume that it doesn't depend on the position of the site but only on the states of the bonds around the sit. The Hamiltonian $\mathcal H$ of this model for a particular choice of the state of the lattice is then the sum of $\varepsilon_{ij}^{kl}$ over all vertices. The partition function is given by $Z = \sum_{\omega \in \Omega} \exp(-\beta H(\omega))$. A \emph{Boltzmann weight} is the quantity
\[
R_{ij}^{kl} := \exp(-\beta \varepsilon_{ij}^{kl}).
\]
\begin{proposition}
Let $V$ be an $m$-dimensional vector space. There exists an endomorphism $T \in \End(V \otimes V^{m})$, which is called a \emph{transfer matrix}, such that the partition function of the model is given by
\[
Z = \tr_{V^{\otimes m}} (\tr_{V} T)^{n}
\]
where the trace is the usual one (the sum of diagonal elements).
\end{proposition}
\begin{proof}
Consider a row in the cubic lattice, for a moment assuming that the boundary conditions on the ends (the states $i_1$ and $i_1^\prime$) may not be the same
\[
\xymatrix{
& & & & & & \\
& \bullet \ar@{-}[l]^{i_1}\ar@{-}[u]^{k_1}\ar@{-}[d]^{l_1} \ar@{-}[r]^{r_1} & \bullet \ar@{-}[u]^{k_2}\ar@{-}[d]^{l_2} \ar@{-}[rr]^{\cdots}&  & \bullet \ar@{-}[u]^{k_{m-1}}\ar@{-}[d]^{l_{m-1}} \ar@{-}[r]^{r_{m-1}} & \bullet\ar@{-}[u]^{k_m}\ar@{-}[d]^{l_m} \ar@{-}[r]^{i_1^\prime} & \\
& & & & & &
}
\]
Let us fix the end states $i_1$, $i_1^\prime$, $k_1,\ldots,k_m$ and $l_1,\ldots,l_m$. The contribution to $Z$ when only $r_i$'s are running over $\Omega_0$ is given by
\[
T^{i_1^\prime l_1 \cdots l_m}_{i_1 k_1 \cdots k_m} := \sum_{r_1,\ldots,r_{m-1}} R_{i_1k_1}^{r_1l_1}\cdots R_{r_{m-1}k_{m}}^{i_1^\prime l_m}.
\]
Let $V$ be an $m$-dimensional vector space spanned by some $e_1,\ldots,e_m$. Define an endomorphism $T \in \End(V \otimes V^{\otimes m})$ by setting on the basis elements
\[
T(e_{i_1}\otimes e_{k_1} \otimes \cdots \otimes e_{k_m}) = \sum_{i_1^\prime,l_1,\ldots,l_m} T^{i_1^\prime l_1 \cdots l_m}_{i_1 k_1 \cdots k_m} e_{i_1^\prime} \otimes e_{l_1} \otimes \cdots \otimes e_{l_m}.
\]
If wee unfreeze the endpoints with states $i_1$ and $i_1^\prime$ and let them run over $\Omega_0$, then we see that the contribution to $Z$ of the whole row (with still fixed states on the vertical bonds and now $i_1 = i_1^\prime$) is given by $\tr_V(T)_{k_1\ldots k_m}^{l_1\ldots l_m}$. Now, if the row was the first one and we consider the next one to it, and let $l_1,\ldots,l_m$ run over $\Omega_0$, then the contribution to $Z$ is
\[
\sum_{l_1,\ldots,l_m} \tr_V(T)_{k_1\ldots k_m}^{l_1\ldots l_m} \tr_V(T)_{l_1\ldots l_m}^{j_1\ldots j_m} = [(\tr_V(T))^2]_{k_1\ldots k_m}^{j_1\ldots j_m}
\]
(the last equality was not obvious to me due to a mess with indices, but it can be checked easily). Continuing in this fashion, the contribution to $Z$ with fixed states of the vertical bonds on the ends is given by  $[(\tr_V(T))^n]_{k_1\ldots k_m}^{l_1\ldots l_m}$. Now, applying the periodic condition $k_j = l_j$ and summing over all possible states of the ends, we finally find that $Z = \tr_{V^{\otimes m}} [\tr_V(T)]^n$.
\end{proof}
I think I can say that a transfer matrix is just a batch of all possible microstates of a row ingeniously packed into a linear endomorphism.
\section{Gaussian free field model}
I follow closely Chapter 8 from \cite{friedli}. In Gaussian free field model, the space of states at a single site is chosen to be $\Omega_0 := \mathbb R$. Accordingly, the space of states on a region $\Lambda \subseteq \mathbb Z^n$ is given by $\Omega_{\Lambda} := \mathbb R^{\Lambda}$. The Hamiltonian of the model is on the lattice $\Lambda$ is chosen to be
\[
\hh := \frac{\beta}{4n} \sum_{i \sim j, \ \{i,j\}\cap \Lambda \neq \emptyset} (\omega_i - \omega_j)^2 + \frac{m^2}{2} \sum_{i \in \mathbb Z^n} \omega_i^2,
\]
where $\beta$ is the inverse temperature, $\omega_i \in \Omega_0$ is the assigned spin at site $i \in \mathbb Z^n$, and $m$ is the mass.

A couple of comments on the choice of the Hamiltonian:
\begin{enumerate}[1)]
\item The factor $(\omega_i-\omega_j)^2$ tells us that the interaction favors the agreement of neighboring spins;
\item Since the space of states at single site is non-compact, we penalize large values of spin by adding the factor $m^2/2 \cdot \omega_i^2$ for each one;
\item Notice the condition under the first summation. It tells us that we also take into the account the boundary of $\Lambda$ (there might be different boundary conditions though).
\end{enumerate}
Fix a finite lattice $\Lambda \subset \mathbb Z^n$ and a state $\eta \in \Omega$ (it serves as a boundary condition for $\Lambda$). For a state $\omega_{\Lambda} \in \Omega_{\Lambda}$, by $\mathcal H(\omega_{\Lambda})$ we mean that we plug into the Hamiltionian the state that equals $\omega_{\Lambda}$ on $\Lambda$ and $\eta$ on the complement of $\Lambda$.

In Subsection \ref{ss:gen_princ} we specified a way of choosing $\sigma$-algebras on the spaces of states. Let $\sigma_{\mathbb Z^n}$ be such $\sigma$-algebra on the whole $\mathbb Z^n$. For $A \in \sigma_{\mathbb Z^n}$, the Gibbs measure in this model is defined as
\[
\mu(A) := \int_A \frac{e^{-\hh(\omega_{\Lambda})}}{Z} \prod_{i \in \Lambda} d\omega_i,
\]
where $d\omega_i$ is the Lebesgue measure on $\mathbb R$ assigned to the site $i \in \mathbb Z^n$ and $Z$ is the obviously chosen partition function. 

There's a way to define Gibbs measures for infinite $\Lambda$ as well (explained in \cite{friedli}, I postpone its description here for a moment). The case of massless GFF is drastically different from the case of massive GFF. For instance, Theorem 8.19 in \cite{friedli} says that there are no infinite-volume Gibbs measures in $n=1$ and $n=2$ cases. Nevertheless, Theorem 8.21 in the same reference tells us that there are infintely many infinite-volume Gibbs measures when $n \geq 3$. In the massive case, the GFF model has infinitely many infinite-volume Gibbs measures for any $n$ (see Theorem 8.28 in \cite{friedli}).
\subsection{$\text{O}(N)$-symmetric model}
I follow Chapter 9 from \cite{friedli}. In $\text{O}(N)$-model, we take $\Omega_0 := S^{N-1}$, so the spins might have an arbitrary direction. For a finite lattice $\Lambda \subseteq \mathbb Z^n$, the Hamiltonian (in the absence of a magnetic field) is usually written as
\[
\hh = - \beta \sum_{i\sim j, \ \ \{i,j\}\cap \Lambda \neq \emptyset} \left\langle \omega_i,\omega_j \right\rangle,
\]
where $\omega_i \in \Omega_0$ is a spin at site $i$, and the brackets denote the standard inner product in $\mathbb R^N$. For different $N$'s, we obtain some familiar models: for $N=1$ we have the Ising model; for $N=2$ we get the $XY$-model; and for $N=3$ we obtain the Heisenberg model.

The definition of finite-volume Gibbs measures is similar to the case of GFF model. At each site $i$, we have Lebesgue measure $d\omega_i$ on $S^{N-1}$. We fix a boundary condition, which is the choice of a state $\eta \in \Omega$, and then for measurable sets $A$ we set
\[
\mu(A) := \int_A \frac{e^{-\hh(\omega_{\Lambda})}}{Z} \prod_{i \in \Lambda} d\omega_i,
\]
where $Z$ is the obvious partition function and $\omega_{\Lambda} \in \Omega_{\Lambda}$; by $\hh(\omega_{\Lambda})$ I mean that we plug in a state equal to $\omega_{\Lambda}$ on $\Lambda$ and $\eta$ outside of $\Lambda$.

One might be interested in the following questions with regards to $\text{O}(N)$-models:
\begin{enumerate}[1)]
\item Is there an orientational long-range order? In my understanding, the mathematical formalism of this question is whether the correlations $\mathbb E_{\mu}\left\langle \omega_i, \omega_j \right\rangle$ converge to zero as $\|i-j\| \rightarrow \infty$;
\item Is there a spontaneous magnetization? The formalism in my understanding is: for any infinite-volume Gibbs measure $\mu$, is it true that $\lim_{n \rightarrow \infty} \left\langle \|m_{B(n)}\| \right\rangle_{\mu} \neq 0$? Here $B(n)$ is a cube of size $n$ and $m_{B(n)} := \frac{1}{|B(n)|} \sum_{i \in B(n)} \omega_i$ is the \emph{magnetization density}.

%the expectations $\left\langle \omega_i \right\rangle_\mu$ are equal to the same number for every $i$?
\end{enumerate}
The answers to both questions are negative for $N \geq 2$ and $n=1,2$. This is due to the following theorem, which can be also stated for a more general Hamiltonian:
%One of the questions for $\text{O}(N)$-models is about the \emph{orientational long-range order}. As I understand, it's formalized in the following way: the correlations $\mathbb E_{\mu}\left\langle \omega_i, \omega_j \right\rangle$ converge to zero as $\|i-j\| \rightarrow \infty$. The following theorem is claimed to provide a negative answer in some cases:

%I think that this can be formalized in the followi way that in the Van Hove limit $B(k) \Uparrow \mathbb Z^n$, where $B(k)$ is, say, a cube of size $k$, the corresponding Gibbs measures converge weakly to a measure $\mu$ such that $\left\langle \omega_i\right\rangle_{\mu}$ is the same non-zero number for all $i \in \mathbb Z^n$. The negative answer to this question in some settings is given by the following theorem (for a slightly more general Hamiltonian see \cite{friedli}, Chapter 9):

\begin{theorem}
(Mermin-Wagner) For $N \geq 2$ and $n=1,2$, all infinite-volume Gibbs measures are invariant under the action of the rotation group.
\end{theorem}

Maybe, I will write why the answers are negative a bit later. 

%Let's see why this theorem yields negative answers to the above questions. Assume $N \geq 2$ and $n=1$ or $n=2$.
%\begin{enumerate}[1)]
%\item Consider the long-range orientability problem. Then... 
%\item Now consider the spontaneous magnetization problem. The theorem actually says that each $\omega_i$ is distributed uniformly: if $A$ is measurable and $T \in \SO(N)$, then $\mu(\omega_i \in A) = (T\mu)(\omega_i \in A) = \mu(\omega_i \in T^{-1}(A))$. This implies that $\left\langle \omega_i \right\rangle_\mu = 0$.
%\end{enumerate}
%It turns out that for $n=1,2$ and $N \geq 2$, the expectation values of the spins are zero: $\left\langle \omega_i \right\rangle_{\mu} = 0$, even at a low temperature and for any measure $\mu$ (see \cite{friedli}).
\section{Free Bosons}
	\subsection{Truncated Free Bosons}
	We have here one site with Hilbert space $\mathcal{H}_M$ of dimension $M$,  and basis $\{|0\rangle, |1\rangle, \cdots, |M-1\rangle\}$.
	We should here say $|0\rangle_M$, $|1\rangle_M$, etc, but will omit the extra subindex for clarity.
	We define creation $a^\dagger$ and destruction operators\footnote{Have checked that they are indeed adjoint to each other even in the truncated case.} $a$, (they should really be $a_M$ and $a^\dagger_M$ but will omit the subindex)
	such that $a|0\rangle=0$ (the zero of the Hilbert space), $a|n\rangle=\sqrt{n}|n-1\rangle$ for all $0<n<M$, 
	$a^\dagger|n\rangle=\sqrt{n+1}|n+1\rangle$ for all $0\le n < M - 1$, $a^\dagger|M-1\rangle=0$.
	Let $H:=H_M:=M^{-1}a^\dagger a$ be the ``Hamiltonian'' operator for free bosons.
	Then $H$ is self-adjoint, every element of the basis is an eigenvector of $H$, and with eigenvalues\footnote{Indeed, $H|0\rangle = 0$ and for $M-1 \geq m > 0$ we have $H|m \rangle = \frac{\sqrt{m}\sqrt{m}}{M}$. } $\{0, 1/M, 2/M, \cdots, (M-1)/M\}$.
	
	Consider the the algebra $\mathcal{U}_M$ generated by $a$ and $a^\dagger$ and the identity $I$. $H$ belongs to this algebra. It's a $C^*$-algebra (even a von Neumann algebra): we included in the list of generators all conjugates and we supplied the algebra with the operator norm, which always satisfies the $C^*$-identity. It's a von Neumann algebra because any finite-dimensional $C^*$-algebra is a von Neumann algebra.
	
Regarding the dimension, it's clear to me that the algebra is finite-dimensional and that the dimension can be bounded from below by $3M-2$. I haven't figured out yet the exact dimension.

	\subsection{Infinite Free Bosons}
		We have here  one site with separable Hilbert space $\mathcal{H}_\infty$ of infinite dimension,  and basis $\{|0\rangle, |1\rangle, \cdots\}$; 
	we define creation $a^\dagger$ and destruction operators $a$,
	such that $a|0\rangle=0$, $a|n\rangle=\sqrt{n}|n-1\rangle$ for all $0<n$, 
	$a^\dagger|n\rangle=\sqrt{n+1}|n+1\rangle$ for all $0\le n$. Notice that both $a$ and $a^\dagger$ are unbounded but their domains are dense in $\mathcal H_{\infty}$.
	
	\begin{statement}
	The approach with energy density does not work. Precisely, the following is true: for every $M \geq 1$, extend $H_M = \frac{1}{M} a_M^\dagger a_M$ to act on $\mathcal{H}_\infty$ by setting it equal to zero on the orthogonal complement of $\mathcal H_{M} \subset \mathcal H_{\infty}$. Then there is a well defined limit in the strong topology: $H_{\infty}|\psi \rangle = \lim_{M \rightarrow \infty} H_{M}|\psi \rangle$. However, $H_{\infty} = 0$.
	\end{statement}
	\begin{proof}
	Indeed, for any basis ket $|k\rangle$, we have $\lim_{M \rightarrow} H_M|k\rangle = \lim_{k \rightarrow \infty}(k/M)|k\rangle = 0$, so the sequence $H_M$ converges strongly to zero.
	\end{proof}
	\begin{statement}
	Redefine $H_M$ as $H_M := a_M^\dagger a_M$ and extend by zero onto the orthogonal complement of $\mathcal H_{M}$, so $H_M |k \rangle = k |k\rangle$ for $k \leq M-1$. Then $H_M$ converges strongly to some unbounded self-adjoint (hence closed) operator $H_{\infty}$ on a dense subspace of $\mathcal H_{\infty}$.
	\end{statement}
	\begin{proof}
The domain of $H_{\infty}$ would be the dense subspace
\[
\Dom H_{\infty} = \{ |\psi\rangle = \sum_{k=0}^\infty \psi_k |k \rangle \in \mathcal H_{\infty} \ | \ \sum_{k=0}^\infty k |\psi_k| < \infty \}.
\]
So, for any $|\psi\rangle$ such that $\sum_{k=0}^\infty k |\psi_k| < \infty$, we can safely define\footnote{It's a general fact from the theory of Banach spaces that a series $\sum_{k=1}^\infty x_k$ converges iff the series $\sum_{k=1}^\infty \|x_k\|$ converges.}
\[
H_{\infty}|\psi \rangle :=\lim_{M \rightarrow \infty} H_{M}|\psi \rangle = \sum_{k=0}^\infty k\psi_k |k \rangle.
\]
The operator is clearly unbounded, for $\|H_{\infty}|k\rangle\| \rightarrow \infty$.

Now let's find the domain of $H_{\infty}^\dagger$. By definition of the adjoint of an unbounded operator, $|\phi \rangle \in \Dom (H_{\infty}^\dagger)$ if and only if there exists $|\theta\rangle \in \mathcal H_{\infty}$ such that for every $|\psi \rangle \in \Dom(H_{\infty})$ we have $\left\langle \phi | H_{\infty} | \psi \right\rangle = \langle \theta | \psi \rangle$. In components, this equality means that $\phi_k^* \psi_k k = \theta_k^* \psi_k$, hence $|\phi\rangle$ must reside in $\Dom(H_\infty)$. Since $H_{\infty}$ is obviously symmetric and $\Dom(H_{\infty}) = \Dom(H_{\infty}^\dagger)$, it is self-adjoint. From the general theory of unbounded operators we know that the adjoint is always closed, hence any $H_{\infty} = H_{\infty}^\dagger$ is closed.
	\end{proof}
	
	
	%For every $M\ge 1$ we extend $H_M$ to act  Since the norm of $H_M$ on $\mathcal H_{M}$ is equal to $(M-1)/M$, such is the norm of the extension as well.
	The limit (in operator norm, weak, strong?) of the sequence of operators $H_M$ exits (?), is unique, and let's call it $H_\infty$.
	Then $H_\infty$ is self adjoint (is it really?) and its spectrum is $\sigma(H_\infty) = [0, 1)$ (does it include 1?).
    Moreover, every $1/n$ for $n>=1$ is an eigenvalue of $H_\infty$, and the corresponding eigenvectors form a basis of $\mathcal{H}_\infty$.
    
	For $P\ge 1$, let  $v_P,\,w_P$ be both vectors in $\mathcal{H}_P$, and consider their extension to any $M\ge P$ by the same name. 
	then $lim_{M\rightarrow\infty} \langle v_P | H_M | w_P\rangle$ exists, and is equal to
	$\langle v_P | H_\infty | w_P\rangle$, where $v_P,\,w_P$ are considered vectors of $\mathcal{H}_\infty$.
	
Consider the algebra $\mathcal{U}_\infty$ generated by $a$ and $a^\dagger$ and the identity $I$. Does $H_\infty$ belong to this algebra?
What's the dimension of this algebra? Is this a C* algebra? Is it a von Neumann algebra? Etc.
	Is this algebra a limit in some sense from the sequence of algebras $\mathcal{U}_M$.
	
	\section{Free Fermions}
	\subsection{Finite Chain}
	Consider one site with Hilbert space of dimension two,  and basis $\{|0\rangle, |1\rangle\}$; physically we say
	that the state $|0\rangle$ is empty, and the state $|1\rangle$ occupied;
	we define creation $c^\dagger$ and destruction operators $c$,
	such that $c|0\rangle=0$, $c|1\rangle=|0\rangle$, $c^\dagger|0\rangle=|1\rangle$, $c^\dagger|1\rangle=0$.
	For $N$ sites, we extend the definitions by making $N$ tensor products, so that the total dimension is $2^N$.
	(Even though we use the word \emph{site} or \emph{sites}, we use here momentum space for simplicity, 
	so that the $H$ below will be already in ``diagonal form.'')
	Given a $N-$tensor product state $|v\rangle\equiv|v_0\rangle\otimes|v_1\rangle\otimes\cdots\otimes|v_{N-1}\rangle$, 
	the $v_m$ are either $0$ or $1$. Let the parity up to 
	$m$ operator $P_m$ be the diagonal operator such that  $P_m|v\rangle = p(|v\rangle, m)|v\rangle$, where
	$p(|v\rangle, m)$ is the number $1$ or $-1$, and is given by 
	\begin{equation}
	p(|v\rangle, m)\equiv(-1)^{\sum_{0\le m'<m} v_{m'}}.
	\end{equation}
	In other words, $P_m |v \rangle$ is $|v\rangle$ if the number of $|1\rangle$ strictly preceding $m$th position is even; $-|v\rangle$ if odd.
	
	We write $d_m$ to mean $I\otimes I\otimes\cdots \otimes c \otimes I \otimes \cdots\otimes I$, where $c$ is at location $m$,
	and $I$ is the one site identity. We write $c_m (N) \equiv d_m P_m$, where $P_m$ is the parity operator up to $m$, as defined before.
	Note that $c_m (N)$ acts on the full space $\{|0\rangle, |1\rangle\}^{\otimes N}$. \emph{I drop now the $(N)$ from the $c_m$.}
	
	\begin{example}
Let $|w\rangle=|0\rangle\otimes|1\rangle\otimes|1\rangle\otimes|0\rangle\cdots$, and remember that we count locations from $0$.
	Then $c_2|w\rangle=-|0\rangle\otimes|1\rangle\otimes|0\rangle\otimes|0\rangle\otimes\cdots$, with a -1 because the sum of 1s before location 2 is odd.
	On the other hand $c^\dagger_3|w\rangle = |0\rangle\otimes|1\rangle\otimes|1\rangle\otimes|1\rangle\otimes\cdots$, with a +1 because
	the sum of 1s before location 3 is even. Obviously  $c_2^\dagger|w\rangle = c_3|w\rangle = 0$ as we can't destroy if there's no particle,
	and we can't create if there's already a particle, because these are fermions and accept only one particle per location.
	\end{example}
	 
	We write $c^\dagger_m c_m$ to mean the tensor product operator with $c^\dagger$ and $c$ at location $m$, 
	identities elsewhere, \emph{and with appropriate parities,} so it's $P_m d^\dagger_m  d_m P_m$. (Careful that $P_m$ and $d_m$ or $d^\dagger_m$ do not commute.)
	Let 
	\begin{equation}
	H_N=N^{-1}\sum_{m=0}^{m=N-1} e_m c^\dagger_m c_m
	\end{equation} be the ``Hamiltonian'' operator for free fermions, where $e_m = -2\cos(2\pi m/N)$ for $0\le m < N$ integers;
	this is called the \emph{dispersion} relation for free periodic fermions.
	Then $H_N$ is a self-adjoint (symmetric) matrix of rank $2^N$.
	
	\subsubsection{Eigenvalues and Eigenvectors of the Hamiltonian}
	All this is just matrix diagonalization, but because of the form of the $2^N$ matrix $H$, we can find its
	eigenvalues and eigenvectors in a compact way, that we know how to describe.
	
	Let $S = \{ f: \{0, 1, \cdots, N - 1\} \rightarrow \{0, 1\}\}$, that can be thought of as binary numbers with $N$ digits.
	There are $2^N$ functions $f$ in $S$ and for each, the number
	$N^{-1}\sum_{m=0}^{m < N} f(m) e_m$ is an eigenvalue of $H_N$ with eigenvector
	\[
	\bigotimes_{m\in I(f)}c^\dagger_{m}|0\rangle,
	\]
	 where $I(f) = \{m\in\{0, 1, \cdots, N - 1\}; f(m) = 1\}$,  
	and $|0\rangle = |0\rangle\otimes|0\rangle\cdots|0\rangle$ the fully ``empty'' state.
 	 In particular, the lowest eigenvector of $H_N$ is
 	 \[
 	 \bigotimes_{m\in I_{min}}c^\dagger_{m}|0\rangle,
 	 \]
 	 with eigenvalue $N^{-1}\sum_{m \in I_{min}} e_m$, where $I_{min} = \{m; 0\le m \le N/(8\pi)\}\cup\{m; 3N/(8\pi)\le m < N\}$.
 	
 	 Exercise: Express the largest eigenvector and eigenvalue to make sure you understand the construction.
 	 
 	 \subsubsection{Fermionic Density}
 	 Let $D_N=N^{-1}\sum_{m=0}^{m < N} c^\dagger_m c_m$ be the density operator. Then $D_N$ is self-adjoint, diagonal, and
 	 with eigenvalues $\{0, 1/N, 2/N, \cdots, 1\}$. Moreover, $D_N$ and $H_N$ commute (physically, $H_N$ ``conserves'' the particle density
 	 or the number of particles), and the eigenvector of $H_N$ characterized by $f\in S$, is also an eigenvector of $D_N$
 	 with eigenvalue equal to the number of elements of $I(f)$ divided by $N$.
 	 
 	 Consider the algebra $\mathcal{U}_M$ generated by $c_m$ (with parity and in the $2^N$ space) and $c^\dagger_m$ and the identity $I$.
 	 $H_N$ and $D_N$ belong to this algebra.
 	 What's the dimension of this algebra? Is this a C* algebra? Is it a von Neumann algebra? Etc.
 	 
 	 \subsection{Infinite Chain}
 	 We have here a separable Hilbert space $\mathcal{H}_\infty$ of infinite dimension,  and basis $\{|0\rangle, |1\rangle, \cdots\}$.
 	 	For every $M\ge 1$ we extend $H_M$ to act on $\mathcal{H}_\infty$ (does this extension make sense?).
 	 		The limit (in operator norm, weak, strong?) of the sequence of operators $H_M$ exits (?), is unique, and let's call it $H_\infty$.
 	 	Then $H_\infty$ is self adjoint (is it really?) and its spectrum is $\sigma(H_\infty) = [-2/\pi, 2/\pi]$.
 	 	
 	 	What are the eigenvalues (if any) and eigenvectors of $H_\infty$?
 	 	
 	 	Let $D_\infty$ be the corresponding limit of the $D_N$ operators. Show it's self-adjoint with spectrum $\sigma(D_\infty) = [0, 1]$.
 	 	What are the eigenvalues (if any) and eigenvectors of $D_\infty$?
 	 	
 	 Consider the algebra $\mathcal{U}_\infty$ generated by $c_i$ and $c^\dagger_i$ and the identity $I$ with some completion.
 	 How does one ``complete'' or ``close'' this algebra?
 	 Does $H_\infty$ belong to this algebra? What about $D_\infty$?
 	 What's the dimension of this algebra? Is this a C* algebra? Is it a von Neumann algebra? Etc.
 	 Is this algebra a limit in some sense from the sequence of algebras $\mathcal{U}_N$.
	\section{Pure Gauge in 1D}
TBW
	
	\section{Pure Gauge in 2D}
	TBW
	\section{Gauge and Matter in 1D}
	TBW
	
	

%\documentclass{article}
%\newcommand{\Mod}[1]{\ \left(\mathrm{mod}\ #1\right)}
%\usepackage{braket}

%\begin{document}
	
	\section{The meaning of Gauge}
	The Quantum Ising model has a \emph{global} symmetry $\tilde{S}_i = -S_i$, that leaves
	the form of the Hamiltonian invariant in the new variables, and so it's the same system.
	By \emph{global} I mean independent of $i$.
	The Quantum Ising Gauge model (see Hunter L.'s write up) has a \emph{local} symmetry.
	
	\section{Gauge and Matter in 1D}
	\subsection{Definitions of the full space}
	We consider again a 1D chain, put bosons of dimension $n$ on the links, with group $Z(n)$ and
	define $E$, $A$, $V$ and $U$ as before. We won't use $A$ and $V$ in the Hamiltonian.
	This time we add fermions on the sites, where at site $l$, the operator $\psi_\ell^\dagger$
	creates a fermion, and the operator $\psi_\ell$ destroys a fermion at that site.
	The fermion operators \emph{anti-commute} on \emph{different} sites.
	Then let
		\begin{equation}\label{eq:1dQEDLatticeHamiltonian}
	H=-\frac{n}{2\pi} \sum_\ell (\psi_\ell^\dagger U_{\ell,\ell+1}\psi_{\ell+1}+H.c.) + 2m\sqrt{\frac{n}{2\pi}}\sum_\ell (-1)^\ell \psi_\ell^\dagger\psi_\ell+  \sum_{\ell} E_{\ell,\ell+1}^2,		
	\end{equation}
	where $n$ is an odd non-negative integer, and $m$ is a real number. We use the following version of the chain with a free end on the left that will determine the states of all the other links under the Gauss law:
	\begin{equation}\label{eq:lchain}
	\xymatrix{
	\ar@{-}[r]^{-1} & \bigcirc^0 \ar@{-}[r]^{0} &  \bigcirc^1 \ar@{-}[r]^{1} & \cdots & \ar@{-}[l]_{N-2} \bigcirc^{N-1}
	}
	\end{equation}
	
%\noindent \textbf{A discussion on the boundary conditions}. I attempted to use the condition
	%\[
	%\xymatrix{
	%\ar@{-}[r]^{-1} & \bigcirc^0 \ar@{-}[r]^{0} &  \bigcirc^1 \ar@{-}[r]^{1} & \cdots & \ar@{-}[l]_{N-2} \bigcirc^{N-1}
	%}
	%\]
	%where the numbers indicate the corresponding indices of links and sites. So, we specify the state on the leftmost link, which is $b_{-1}$, and this determines the states of all the other links in the physical space. However, there is nothing in the Hamiltonian that might change $b_{-1}$, it always stays the same, but all the other $b_i$'s might change (the first term of the Hamiltonian to make sense requires sites on each side of the link). This means that we will never land in the physical space. We might, however, be pleased with something like
		%\[
	%\xymatrix{
	%\square\ar@{-}[r]^{-1} & \bigcirc^0 \ar@{-}[r]^{0} &  \bigcirc^1 \ar@{-}[r]^{1} & \cdots & \ar@{-}[l]_{N-2} \bigcirc^{N-1}
	%}
	%\]
	%with the meaning of $\square$ to be an abyss that swallows any fermion that hops into it. For now, I will attempt the condition
			%\[
	%\xymatrix{
%\bigcirc^0 \ar@{-}[r]^{0} &  \bigcirc^1 \ar@{-}[r]^{1} & \cdots & \ar@{-}[l]_{N-2} \bigcirc^{N-1},
	%}
	%\]
	%so I don't have any boundary links.
	
	The full space on $N$ sites has dimension $(2n)^{N}$, the $2^N$ comes from the fermions (electrons and positrons), 
	and the $n^N$ from the bosons (phonons). The actual physical space we work with is specified by the kernel of the Gauss law (see below).
	
	A basis element at site $l$ is denoted as $|n_l, b_l\rangle$, where $n_l \in \{0,1\}$ indicates the presence of a fermion at site $l$ and $b_l \in \{0,\ldots,n-1\}$ indicates the state of the boson to the right of the fermion. Such basis is called \emph{computational}.
	
	Recall how the operators act. Let $|n_0 \, n_1\, \ldots \, n_{N-1}\, b_{-1}\, b_0 \, \ldots \, b_{N-2}\rangle$ be an arbitrary basic state. Then
	\begin{enumerate}[(a)]
	%\item $\psi^\dagger_\ell |0,k\rangle = |1,k\rangle$, $\psi^\dagger_l |1,k\rangle = 0$;
	\item \[\begin{split}
	&\psi^\dagger_\ell |n_0 \, n_1\, \ldots \, n_{N-1}\, b_{-1}\, b_0 \, \ldots \, b_{N-2}\rangle =  \\
&=(-1)^{n_0 + n_1 + \cdots + n_{\ell-1}}(1-n_\ell)|n_0 \, n_1\, \ldots\, n_\ell=1\, \ldots \, n_{N-1}\, b_{-1}\, b_0 \, \ldots \, b_{N-2}\rangle;
	\end{split}
	\]
	\item %$\psi_l|0,k\rangle = 0$, $\psi_l|1,k\rangle = |0,k\rangle$;
	\[\begin{split}
	&\psi_\ell |n_0 \, n_1\, \ldots \, n_{N-1}\, b_{-1}\, b_0 \, \ldots \, b_{N-2}\rangle =  \\
&=(-1)^{n_0 + n_1 + \cdots + n_{\ell-1}}n_\ell|n_0 \, n_1\, \ldots\, n_\ell=0\, \ldots \, n_{N-1}\, b_{-1}\, b_0 \, \ldots \, b_{N-2}\rangle;
	\end{split}
	\]
	\item %$\psi^\dagger_l \psi_l |0, s\rangle = 0$ and $\psi^\dagger_l \psi_l |1,s\rangle = |1,s\rangle$;
	\[
	\psi^\dagger_{\ell} \psi_{\ell} |n_0 \, n_1\, \ldots \, n_{N-1}\, b_{-1}\, b_0 \, \ldots \, b_{N-2}\rangle = n_\ell |n_0 \, n_1\, \ldots \, n_{N-1}\, b_{-1}\, b_0 \, \ldots \, b_{N-2}\rangle;
	\]
	\item For $n = 2s+1$ odd, %$E_{l,l+1}|c,k\rangle = (k-s)|c,k\rangle$. So, $E_{l,l+1}^2|c,k\rangle = (k-s)^2 |c,k\rangle$;
	\[
	E_{\ell,\ell+1} |n_0 \, n_1\, \ldots \, n_{N-1}\, b_{-1}\, b_0 \, \ldots \, b_{N-2}\rangle = (b_{\ell}-s) |n_0 \, n_1\, \ldots \, n_{N-1}\, b_{-1}\, b_0 \, \ldots \, b_{N-2}\rangle;
	\]
	\[
	E_{\ell,\ell+1}^2 |n_0 \, n_1\, \ldots \, n_{N-1}\, b_{-1}\, b_0 \, \ldots \, b_{N-2}\rangle = (b_\ell -s)^2 |n_0 \, n_1\, \ldots \, n_{N-1}\, b_{-1}\, b_0 \, \ldots \, b_{N-2}\rangle;
	\]
	\item \[\begin{split}
	U_{\ell,\ell+1} &|n_0 \, n_1\, \ldots \, n_{N-1}\, b_{-1}\, b_0 \, \ldots \, b_{N-2}\rangle =\\= &|n_0 \, n_1\, \ldots \, n_{N-1}\, b_{-1}\, b_0 \, \ldots \, b_{\ell}+1\, \ldots\, b_{N-2}\rangle
	\end{split}
	\]%$U_{l,l+1} |c,k\rangle = |c,k+1\rangle$;
	\item %Let $|n_0\, n_1\, \ldots \, n_{N-1} \, b_{-1}\, b_0\, \ldots\, b_{N-1}\rangle$ be a state. Then
	\begin{equation}\label{eq:ham_fir_t}\begin{split}
	\psi_\ell^\dagger U_{\ell,+1} \psi_{\ell+1} &|n_0\, n_1\, \ldots \, n_{N-1} \, b_{-1}\, b_0\, \ldots\, b_{N-2}\rangle=\\ = n_{\ell+1} (1-n_\ell) &|n_0\, n_1\, \ldots \, n_{\ell}=1 \, n_{\ell+1} = 0\,\ldots \, n_{N-1} \, b_{-1}\, b_0\, \ldots\,b_\ell + 1\,\cdots \, b_{N-2}\rangle.
	\end{split}
	\end{equation}
	So, when we have a fermion at site $l+1$ and no fermion at site $l$, the operator $\psi_l^\dagger U_{l,l+1} \psi_{l+1}$ shifts the fermion to site $l$ and increments the state of the boson on the corresponding lift. Similarly, its Hermitian conjugate acts as
		\[\begin{split}
	\psi_{\ell+1}^\dagger U_{\ell,\ell+1}^\dagger \psi_{\ell} &|n_0\, n_1\, \ldots \, n_{N-1} \, b_{-1}\, b_0\, \ldots\, b_{N-2}\rangle=\\ = n_{\ell} (1-n_{\ell+1}) &|n_0\, n_1\, \ldots \, n_{\ell} = 0\, n_{\ell+1} = 1\, \ldots\, n_{N-1} \, b_{-1}\, b_0\, \ldots\,b_\ell -1 \,\cdots \, b_{N-2}\rangle.
	\end{split}
	\]
	\end{enumerate}
	%\item Now let's observe how the Hamiltonian acts on states. 
	%\[
	%\xymatrix{
	%\bigcirc \ar@{-}[rr]& & \bigcirc \\
	%& \bigcirc\ar@{-}[ru] \ar@{-}[lu] &
	%}
	%%\xymatrix{
	%%\bigcirc \ar@{-}[r] &	\bigcirc \ar@{-}[r] &	\bigcirc 
	%%\ar@{-}@/^1pc/[ll]
	%%}
	%\]
	%Let $|n_0,n_1,n_2,b_0,b_1,b_2\rangle$ be a state. Let's see what parts of $H$ do to this state. Due to the rotational symmetry, there are actually only four distinct situations (all that matters is the number of fermions). So, 
	%\[\begin{split}
%&-\frac{n}{\pi}\sum_{l=0}^2 (\psi^\dagger_l U_{l,l+1} \psi_{l+1}) |n_0,n_1,n_2,b_0,b_1,b_2\rangle =\\ &= \begin{cases}
%0, \ \ \text{no fermions or exactly three}\\
%\text{Set } n_{l} = 0, \ n_{l-1} = 1, \ b_{l-1} \mapsto b_{l-1} + 1 \ \text{if there is a fermion at site }l \text{ and no fermion at site }l-1
%\end{cases}
%\end{split}
	%\]
	%\end{enumerate}
	%For $2\tilde{m}\sum_{l} (-1)^l \psi_l^\dagger \psi_l$, we have
	%\[
	%(2\tilde{m}\sum_{l} (-1)^l \psi_l^\dagger \psi_l) |n_0,n_1,n_2,b_0,b_1,b_2\rangle  = 2 \tilde{m} \cdot F \cdot |n_0,n_1,n_2,b_0,b_1,b_2\rangle,
	%\]
	%where 
	%\[
	%F = \begin{cases}
	%0, \ \ \text{there are two fermions and one of them is at site }1;\\
	%(-1)^l, \ \ \text{there is only one fermion and located at } l;\\
	%2, \ \ \text{the fermions are at sites }0\text{ and }2;\\
	%
	%\end{cases}
	%\]
	%Lastly,
	%\[
	%\sum_l E^2_{l,l+1} |n_0,n_1,n_2,b_0,b_1,b_2\rangle = 2|n_0,n_1,n_2,b_0,b_1,b_2\rangle.
	%\]
	%Now let's see what happens for some specific states. If there's only one fermion, then
	%\[
	%H|1,0,0; b_0,b_1,b_2\rangle = |0,0,1; b_0,b_1,b_2+1\rangle + (2\tilde{m}+2)|1,0,0; b_0,b_1,b_2\rangle 
	%\]
	%and the rest is rotationally symmetric.
	%
	
	From the letter: in QED fermions (matter) occupy even sites, and anti-fermions (anti-matter) occupy odd sites. The same for QCD, except that fermions in QED are electrons, and fermions in QCD are quarks.
	
	\subsection{Gauss Law and Physical Space}
	Let the Gauss law be
		\begin{equation}\label{eq:1dQED_Gauss_law}
	G_\ell := \psi_\ell^\dagger \psi_\ell + \frac{1}{2}((-1)^\ell -1)-(E_{\ell,\ell+1}-E_{\ell-1,\ell}).
	\end{equation}
	We restrict the full space to the physical space of states $|\psi\rangle$ that satisfy
	$G_\ell|\psi\rangle= 0 \Mod{n}$. 
	The physical space has dimension $n 2^N$; see Hunter L's explanation for why this is the case.
	
	I wanted to expand a bit on the Gauss law. So, all pieces that comprise $G_\ell$ are diagonal, hence $G_\ell$ is diagonal itself and its kernel is spanned by some of the basic kets. If $|n_0\, n_1\, \ldots \, n_{N-1} \, b_{-1}\, b_0\, \ldots\, b_{N-2}\rangle$ is such, then the action of $G_\ell$ upon it is given by
	\[\begin{split}
	G_\ell|n_0\, n_1\, \ldots \, n_{N-1} \, b_{-1}\, b_0\, \ldots\, b_{N-2}\rangle =\\= (n_{\ell} + \frac{1}{2}((-1)^\ell-1) + b_{\ell-1}-b_\ell) |n_0\, n_1\, \ldots \, n_{N-1} \, b_{-1}\, b_0\, \ldots\, b_{N-2}\rangle,
	\end{split}
	\]
We see that $b_\ell$ is determined by $b_{\ell-1}$ via
	\[
	b_\ell = b_{\ell-1} + n_{\ell} + \frac{1}{2}((-1)^\ell-1).
	\]
	So the physical space is 
	\[
	\Sp ( |n_0\, n_1\, \ldots \, n_{N-1} \, b_{-1}\, b_0\, \ldots\, b_{N-2}\rangle \ | \ b_{\ell} =b_{\ell-1} + n_{\ell} + \frac{1}{2}((-1)^\ell-1) \ \text{for every} \ \ell).
	\]
	Recursively solving the equation for $b_{\ell}$, we find that
	\[
	b_{\ell} = b_{-1} + n_0 + n_1 + \cdots + n_{\ell} - \left[\frac{\ell+1}{2}\right],
	\]
	where $[ \, ]$ denotes the integer part.
	
	\begin{proposition}
Each term in the Hamiltonian defined as in \eqref{eq:1dQEDLatticeHamiltonian} on the chain \eqref{eq:lchain} with a free left link leaves the physical space invariant.
	\end{proposition}
	\begin{proof}
	The first term that shifts a fermion to the left acts according to the formula \ref{eq:ham_fir_t}. If $n_\ell = 0$ and $n_{\ell+1} = 1$, then the term acts non-trivially and the state on $\ell$th link is updated according to
	\[
	b_\ell + 1 = b_{-1} + n_0 + n_1 + \cdots + (n_\ell+1) -\left[\frac{\ell+1}{2}\right],
	\]
	so we simply add $1$ to both sides. For $k \geq 1$, the equation of states on the other links are also preserved:
	\[
	b_{\ell+k} = b_{-1} + n_0 + \cdots + (n_{\ell}-1) + (n_{\ell+1}+1) + \cdots + n_{\ell +k} -\left[\frac{\ell+1}{2}\right].
	\]
	Similarly for Hermitian conjugate term (now subtract $1$ from both sides in the equation for $b_{\ell}$).
	The other terms in the Hamiltonian are diagonal, so they obviously preserve the physical space. Thus the Hamiltonian itself does preserves the physical space.
	\end{proof}
	
	\subsection{Limits}
	Ercolessi et al \cite{ercolessi} numerically take both limits \emph{first} the $N\rightarrow\infty$ limit
   and then the $n\rightarrow\infty$ limit and find a ``phase transition'' at some value of $\tilde{m}$. Order of the limits here FIXME TODO.
	For now, perhaps is enough to take $N\rightarrow\infty$ for $n$ finite, and prove that the limiting (topological, C*, Von Neuman??)
	algebra exists, and the limiting $H$ exits.
	To define a phase transition I need an ``order parameter'' which will be something like a magnetization in a spin model.
	Let $W=N^{-1}\sum_l E_{l, l+1}$ be the ``Wilson loop'' operator. Let $|gs\rangle$ be the ground state of $H$.
	Let $w_1(\tilde{m}, n, N) = \langle gs | W |gs\rangle$, and let $w(\tilde{m}, n)=\lim_{N\rightarrow\infty} w_1(\tilde{m}, n, N)$.
	 This quantity $w$ as a function of $\tilde{m}$ will have a first derivative everywhere,
	except for a value of $\tilde{m}$, where the left and right derivatives both exist but are different.
	We say that the model goes through a phase transition at that value of $\tilde{m}$. Obviously, that value will be dependent on $n$,
	and we are interested in $n\rightarrow\infty$.
	
	\subsubsection{Staggered configurations}
 We define two \emph{staggered} states: $|\st_1(b_{-1})\rangle$ and $|\st_2(b_{-1})\rangle$ in the following way: for the first one, all even sites are occupied and odd are empty; for the second one, all even sites are empty and all odd sites are occupied. The states on the links are determined by $b_{-1}$, which we include in the definition. 
\begin{statement}
For $|\st_2(b_{-1})\rangle$, $
b_{\ell} = b_{-1}$; for $|\st_1(b_{-1})\rangle$, $
b_{\ell} = \begin{cases}
b_{-1} + 1, \ \ell = 2k;\\
b_{-1}, \ \ell = 2k+1.
\end{cases}$. Pictorially,
\[
|\st_2(b_{-1})\rangle =	\xymatrix{
	\ar@{-}[r]^{b_{-1}} & \bigcirc \ar@{-}[r]^{b_{-1}} &  \bigotimes \ar@{-}[r]^{b_{-1}} & \cdots & \ar@{-}[l]_{b_{-1}} ?
	}
\]
\[
|\st_1(b_{-1})\rangle =	\xymatrix{
	\ar@{-}[r]^{b_{-1}} & \bigotimes \ar@{-}[r]^{b_{-1}+1} &  \bigcirc \ar@{-}[r]^{b_{-1}} & \cdots & \ar@{-}[l]_{b_{-1}+?} ?
	}
\]
where $\otimes$ means that the site is occupied. This is a simple computation using aforementioned formulas derived from the Gauss law.
\end{statement}

\subsubsection{Wilson loop for $m \gg 0$}
Split the Hamiltonian as $H = A + 2m\sqrt{n/2\pi} D$, where $A$ is the first term of $H$ and $D = \sum_l (-1)^l \psi_l^\dagger \psi_l$.
\begin{statement}
We have 
\[
\lim_{m \rightarrow \infty}m^{-1} \langle \gs | H| \gs\rangle = \lim_{m \rightarrow \infty} m^{-1}\langle \st_2 | H| \st_2 \rangle =  2\sqrt{\frac{n}{2\pi}} \langle \st_2 | D| \st_2\rangle = -2\left[\frac{N}{2}\right]\sqrt{\frac{n}{2\pi}}.
\]
As a consequence, if $\lambda_1$ is the eigen-value corresponding to $|\gs\rangle$, then $\lambda_1 = O(m)$. More precisely, 
\[
\lim_{m \rightarrow \infty} \frac{\lambda_1}{m} = -2\left[\frac{N}{2}\right]\sqrt{\frac{n}{2\pi}}.
\]
This implies that for $m \gg 0$ and $N$ even (or large), the lowest energy level can be approximated as 
\[
\lambda_1(m) \approx - Nm \sqrt{\frac{n}{2\pi}}.
\]
\end{statement}
\begin{proof}
Notice that the eigen-vectors of $H$ and $m^{-1}H$ are the same (and their natural order is preserved). From the definition of $\gs_2$, we see that
\[
m^{-1}\langle \st_2 | H| \st_2 \rangle \geq m^{-1}\langle \gs | H| \gs \rangle = \min_{\|\psi\|=1} \langle \psi |H|\psi\rangle \geq m^{-1}\min_{\|\psi\|=1} A\psi + 2\sqrt{\frac{n}{2\pi}} \langle \st_2 | D| \st_2\rangle 
\]
Note that 
\[
\lim_{m\rightarrow \infty} m^{-1}\langle \st_2 | H|\st_2 \rangle = 2\sqrt{\frac{n}{2\pi}} \langle \st_2 | D| \st_2\rangle
\]
Hence, taking the limit in the above inequality, we obtain the statement. With regards to $\lambda_1$, we write $\langle \gs | H| \gs \rangle = \lambda_1\langle \gs |\gs\rangle = \lambda_1$.
\end{proof}
\begin{quest}
I want to prove rigorously that $\lim_{m \rightarrow \infty} |\gs \rangle = |\st_2 \rangle$. This statement is corroborated with numerical simulations.
\end{quest}
%\begin{wrong}
%I needed to prove the following: let $A \in \Mat(n\times n,\mathbb C)$ be a matrix (one might assume $A = A^\dagger$ if needed), $D \in \Mat(n\times n,\mathbb C)$ be diagonal and let $A(m) := A + mD$, $m \in \mathbb C$. Let $\lambda_1(m),\ldots,\lambda_n(m)$ be the list of eigen-values of $A(m)$. Then $\lambda_i(m) = O(m)$ or $\lambda_i(m) \equiv \text{const}$. This is not at all true. I checked numerically in Matlab that $\lambda_i$ may not be even a polynomial and may contain any power of $m$, even fractional.
%\end{wrong}
%\begin{statement}
%Assume that we proved $\lim_{m \rightarrow \infty} m^{-1} |\gs\rangle = |\st_2\rangle$. As a consequence, we obtain $\lim_{m \rightarrow \infty}\omega_1(m,n,N) = 
%\end{statement}

	
	\subsection{Idea: an equivalent model without links}
	Consider QED 1+1 with the Hamiltonian specified in Equation \ref{eq:1dQEDLatticeHamiltonian}, which reads 
\[
	H=-\frac{n}{2\pi} \sum_\ell (\psi_\ell^\dagger U_{\ell,\ell+1}\psi_{\ell+1}+H.c.) + 2m\sqrt{\frac{n}{2\pi}}\sum_\ell (-1)^\ell \psi_\ell^\dagger\psi_\ell+  \sum_{\ell} E_{\ell,\ell+1}^2
\]
Specifying the state of a boson on one chosen link assigns the states of bosons to all the other links. This suggests that we can consider a family of Hamiltonians parametrized by the state of the marked boson. View the chain as before:
	\[
	\xymatrix{
	\ar@{-}[r]^{-1} & \bigcirc^0 \ar@{-}[r]^{0} &  \bigcirc^1 \ar@{-}[r]^{1} & \cdots & \ar@{-}[l]_{N-2} \bigcirc^{N-1}
	}
	\]
	Let $q \in \mathbb Z_n$ be a state on the leftmost link. Recall that the physical space (determined by the Gauss law) is given by
	\[
	P = \Sp ( |n_0\, n_1\, \ldots \, n_{N-1} \, b_{-1}\, b_0\, \ldots\, b_{N-2}\rangle \ | \ b_{\ell} =b_{\ell-1} + n_{\ell} + \frac{1}{2}((-1)^\ell-1) \ \text{for every} \ \ell).
	\]
	We can split $P$ into the direct sum according to what value of $b_{-1}$ is:
	\[
	P = P_0 \oplus P_1 \oplus \cdots \oplus P_{n-1},
	\]
	where
	\[
	P_j = \Sp( |n_0\, n_1\, \ldots \, n_{N-1} \, b_{-1}\, b_0\, \ldots\, b_{N-2}\rangle \in P \ | \ b_{-1} = j).
	\]
	Now, let's observe what is the restriction of $H$ to each subspace $P_j$. Denote $H_j := H|_{P_j}$.
\begin{idea}
Does there exist a nice basis in each $P_j$ that allows us to write the action of $H_j$ as if there were no links?
\end{idea}

\subsubsection{Order of basis}
I show on an example what the order I choose. For $N=2$ and $n=3$, writing a basic ket as $|n_0\, n_1\, b_{-1}\, b_0\rangle$, we have
\[
|0\, 0\, 0\, 2\rangle \prec |0\, 0\, 1\, 0\rangle \prec |0\, 0\, 2\, 1\rangle \prec 
\prec |1\, 0\, 0\, 0\rangle \prec |1\, 0\, 1\, 1\rangle \prec |1\, 0\, 2\, 2\rangle \prec
\]\[
 |0\, 1\, 0\, 2\rangle \prec |0\, 1\, 1\, 0\rangle \prec |0\,1\, 2\, 1\rangle \prec
|1\, 1\, 0\, 0\rangle \prec |1\, 1\, 1\, 1\rangle \prec |1\, 1\, 2\, 2\rangle.
\]
So the rule is: first, split the basis into groups $F(n_0,\ldots,n_{N-1})$, each of which depends only on the states of the sites. We declare that
\[
F(n_0,\ldots,n_{N-1}) \prec F(\tilde{n}_0,\ldots,\tilde{n}_{N-1})
\]
if and only if, as a binary number,
\[
\underline{n_{N-1}\cdots n_{0}} \leq \underline{\tilde{n}_{N-1}\cdots \tilde{n}_{0}}.
\]
Inside of each group, we order the kets by $b_{-1}$ (the other $b_{\ell}$'s are completely determined by the states on the links and $b_{-1}$).

\subsubsection{Some observations}
\begin{statement}
Let $A_{\ell} := \psi_\ell^\dagger U_{\ell,\ell+1} \psi_{\ell+1}$. Fix the states $\underline{n}_0,\ldots,\hat{n}_{\ell},\hat{n}_{\ell+1},\ldots,\underline{n}_{N-1},\underline{b}_{-1},\underline{b}_0,\ldots,\hat{b}_{\ell},\ldots,\underline{b}_{N-2}$ (the hat means that we skip that site or link and I use the underline to mean that the state is fixed).
\[\begin{split}
V:=V(\underline{n}_0,\ldots,\hat{n}_{\ell},\hat{n}_{\ell+1},\ldots,\underline{n}_{N-1},\underline{b}_{-1},\underline{b}_0,\ldots,\hat{b}_{\ell},\ldots,\underline{b}_{N-2}) :=\\= \Sp \{ |n_0\, n_1\, \ldots\, n_{N-1}\, b_{-1}\, b_0\, \ldots\, b_{N-2} \in P \ | \ n_i = \underline{n}_i, \ i \neq \ell, \ell+1; \ b_i = \underline{b}_{i}, \ i \neq \ell; \ n_{\ell} \neq n_{\ell+1}  \}.
\end{split}
\]
Basically, I define a subspace in which everything is fixed except that $b_{\ell}$ can vary and $n_{\ell}$ with $n_{\ell+1}$ cannot be equal. Notice that the dimension of this subspace is equal to $2\cdot n$. We might say about the restriction of $A_{\ell}$ to $V$ the following:
\begin{enumerate}[1)]
\item As a matrix, the restriction of $A_{\ell}$ to $V$ might be represented as
\[
A_{\ell}|_V = \begin{pmatrix}
0 & 0 \\
I & 0
\end{pmatrix}
\]
where $I$ is the identity $n \times n$ matrix.
%where $S$ is $n \times n$ matrix given by
%\[
%S = \begin{pmatrix}
%0 & \cdots & 0 &0 & 1\\
%1 & & & &\\
%& 1 & & &\\
%& & \ddots && \\
%& & & 1 & 0
%\end{pmatrix}
%\]
\item We have
\[
(A_{\ell} + A_{\ell}^\dagger)^2|_V = AA^\dagger + A^\dagger A= I,
\]
where $I$ is the identity matrix. This is a simple computation.
\end{enumerate}
\end{statement}
%\end{document}

%\documentclass{article}
%\newcommand{\Mod}[1]{\ \left(\mathrm{mod}\ #1\right)}
%\usepackage{braket}

%\begin{document}

 	 \section{Additional comments}
 	 \subsection{Commutators and Anti-commutators}
 	 Let $[A, B] \equiv AB - BA$.
 	 In the bosonic truncated example, prove (well, ``prove'' is too strong for this straightforward result, I mean, just note)
 	 that $[a, a^\dagger] \neq I$. This is way more general: No operators $A$ and $B$ over a \emph{finite} dimensional
 	 Hilbert space can satisfy $[A, B] = I$. (Hint: Use the trace and the circular property of the trace.)
 	 For the non-truncated (non-truncated above, it's always truncated below) infinite dimensional bosonic example,
 	 note that $[a, a^\dagger] = I$.
 	 Physics jargon: This commutator is the
 	 starting point for the  ``Heisenberg uncertainty principle'' and some models require it out of principle, hence its importance.
 
 	 Let $\{A, B\}=AB + BA$. This is physics notation and unfortunately clashes with the set $\{\}$ notation. Some use $[A, B]_+$ but I won't here,
 	 at least for now.
 	 In the fermionic examples, note that $\{c^\dagger_i, c_j\} = \delta_{i,j}I$,
 	 and that $\{c_i, c_j\} = 0$ for finite and infinite number of sites $N$.
 	 Remember that for $N$ sites, $c^\dagger_i$ actually acts on a $2^N$ dimensional space, and \emph{includes} the ``fermionic sign,'' that is the parity
 	 operator.  (I suppose it'd be too tedious to write $d_iP_i\equiv c_i$.)
 
 	 \subsection{Quantum Field Theory}
 	 You wanted me to explain what QFT means. This is going to be opinionated, so feel free to disagree.
 	 I'll give you two ``definitions'': one good that I like and one bad that can be found in some (bad) literature,
 	 where people try to ``patch'' it to make it sensible without success. As I'm biased toward the lattice, I'd prefer to work on the lattice definition.
 	 A QFT is the double infinity: that is, a model where we have infinite sites with an already infinite (separable) Hilbert space on each single site.
 	 For example, take the (already infinite and separable) bosonic example above with one site as defined, and add $N$ sites by tensor product, and take $N$ to infinity. You'll already get the desired commutation $[a_i^\dagger,a_j]=\delta_{i,j}I$, and $[a_i, a_j]=0$.
 	 Now for the bad definition: Take the bosonic example above with one site, and instead of adding sites, put a \emph{continuum} variable to each
 	 $a$ and $a^\dagger$ and make it $a(x)$, and define an algebra there. You want the following property: $[a(x)^\dagger, a(y)]=\delta(x-y)I$, where you
 	 have to define that ``Dirac delta $\delta$'' in some sort of Schwartz space of tempered distributions. You'll see that there's no way
 	 to have operators like that, I mean, that obey that commutation relation in the continuum. People have worked to savage this definition, but nothing has worked in the continuum
 	 if there are interactions, that is, beyond the simplest quadratic Hamiltonian you can write.
 
 	 You can also ``define'' QFT in the Lagrangian formalism, with both good and bad definitions: that is, with lattice and continuum definitions.
 	 On the lattice, you'll have a countable and infinite number of integrals that you need to make sense of.
 	 On the continuum, you'll have a non-countable number of integrals to deal with, and make sense of it.
 	 According to A. Jaffe and E. Witten, (the guys that wrote the text for the Yang-Mills millenium problem), Balaban was the one that has had
 	 more success with proving existence of YM theories, (existence in lower dimensions than 3+1), and Balaban has worked with the good definition
 	 (the lattice one), albeit in the Lagrangian or analysis or partition function formalism, however you want to call it.
 
 	 \subsection{Hamiltonian versus Lagrangian}
 	 We'll use only the Hamiltonian approach and thus mostly algebra (plus topology) as opposed to analysis. But here I write this, just in
 	 case, we find that some things are better seen with analysis.
 	 How are the Hamiltonian (algebraic) and Lagrangian (analysis, partition function) approaches connected?
 	 Answer: By the transfer matrix method; see Hunter L.'s write up on this topic.
 	 Are the Hamiltonian and Lagrangian formulations of a given model equivalent?
 	 Short Answer: Yes. Longer answer: We would have to define ``equivalent'' something like ``yields the same physics''; we would also have to have a definition
 	 of the model in question in at least one approach (Hamiltonian or Lagrangian) and convert to the other using the Transfer Matrix.
 	 With all that said, the answer is yes.
 
 	 \subsection{Mapping from First Quantization to Second Quantization}
 	 We won't work in First Quantization but I can write this mapping for a general $H$ or at least for some examples if you are interested in this info.
 	 FIXME. TODO.
 
	\section{Pure Gauge in 1D}
    OK, here I'll borrow from Hunter L.'s 6.2. Lattice $Z(n)$-QED Model in 1-d.
	I want to do this thing first WITHOUT fermions so just pure gauge; and this may be too trivial given that we're in one dimension.
	As in the Hamiltonian formalism we don't include physical time this is pure QED in 1+1 (if you read the literature), but it needs
	to be qualified, actually \emph{doubly} qualified: on the lattice and with a $Z(n)$ group instead of $U(1)$ (we'll go to U(1), soon, though).

	But let's start with all finite, so we have two finite integers: $n$ in $Z(n)$ so our one-site Hilbert space has size $n$, and then we have
	$N$: the number of sites. Instead of putting our Hilbert space on the sites, we'll put them on the links.
	We assume periodic boundary conditions (PBC) (for now I think it's probably better than open boundary conditions), and so we
	have also $N$ links: there's a link connecting the last site with the first, and our lattice looks more like a ring than a line.
	Now, I need to define some operators. I will define operators on each link, and then think of them as acting on the $n^N$ Hilbert space
	by tensor product. If I look at the Hamiltonian of $Z(n)$ lattice QED in 1D in Eq.~(6.2) of Hunter's write up, I see that only the last
	term $E^2_{i,i+1}$ remains. So I have to define this so-called ``electric field'' operator. Here the subindex $i,i+1$ refers to the lattice link that connects sites
	$i$ with $i+1$; Let's rather write that as $E^2_{i,x}$ where $i,x$ says: the link that starts at site $i$ and goes in the forward $x$ direction.
	But for simplicity and because we're in 1D, I'll drop the $x$ from the subindex.

	Scaling the Hamiltonian will be very important here. While in the previous examples we had only one term, and so only an overall
	factor that would make the spectrum be in $[0,1]$ as opposed to $[0,\infty]$ with straightforward conversion between the two
	definitions, in the following examples, we'll have more than one term in $H$, and therefore, we would get unrelated results
	if the \emph{relative} scaling of the various Hamiltonian terms is different. I will then choose $E_i$ to be the one with \emph{tilde} in
	Hunter's writing. Moreover, I restrict this thing to $n$ odd. So here we go: Let $\{|0\rangle, |1\rangle,\cdots,|n-1\rangle\}$ with $n$ odd,
	be the basis of the Hilbert space on link $i,x$, we then define
	the non-negative integer $s$ such that $n=2s+1$, and the operator
	$E_i|k\rangle=(k-s)|k\rangle$, which is of course diagonal, with integer eigenvalues $-s$ to $s$ in steps of 1.
	We extend this definition by tensor product
	to the space of $N$ sites and dimension $n^N$. For example if $N=3$, and $|v\rangle=|k_0\rangle|k_1\rangle|k_2\rangle$,
	then $E_1|v\rangle=(k_1-s)|v\rangle$, and so $E_i$ is a square matrix with $n^N$ rows. Note that in this tensor product extension, there isn't a
	fermionic sign or Parity operator, because the $E$ operators are bosons and commute on different sites.

	We of course need a Hamiltonian here; Because we don't have fermions, just pure gauge, I'll define it as
	\begin{equation}
	H=\sum_{i=0}^{i<N} E_i^2,
	\end{equation} so that it's just another square matrix with $n^N$ rows, and note that I omitted the identities in the tensor product,
	because I would have to write $I\otimes I\otimes\cdots E_i^2 \otimes I \otimes I$ with $E_i^2$ at location $i$, instead of just $E_i^2$.
	Before finding eigenvalues and eigenvectors of $H$, we need to deal with the gauge symmetry.

	\subsection{Gauge Symmetry}
	Unlike in non-gauge models, here we need to weed out states that don't follow Gauss Law. So, we'll define the Gauss law operator, and
	consider its kernel as a (Hilbert) subspace of the $n^N$ dimensional Hilbert space considered before.
	Because this \emph{is} a subspace, things should be fine, and we should reconsider all operators defined before to be constrained to that space.
	This Hilbert subspace is going to be called ``physical'' as opposed to the original space that contains physical and unphysical states.
	We will then forget forever the original big space; I say big, because obviously the physical space will be smaller than the original.
	Let $G$ acting on $N$ sites be
	\begin{equation}
	G = \sum_{i=0}^{i=N-1} E_{i+1} - E_i,
	\end{equation}
	with $E_N\equiv E_0$.
	Physical jargon justification: Because we don't have fermions (electric charge in QED) then the divergence of the electric field
	over a closed loop should give zero. And because on a one dimensional lattice, the only loop seems to be the one going from beginning to end, we
	get the sum over all sites of the difference.

	The operator $G$ so defined is exactly the zero operator, so we don't have to restrict anything. Oops.

	\subsection{Algebra of Operators}
	Let me define another operator called $U_i|k\rangle=\ket{k+1 \Mod{n}}$ and I also define $U^\dagger_i|k\rangle=\ket{k - 1 \Mod{n}}$, so
	that these operators wrap around. Obviously this operator is extended to $N$ sites by tensor products, and is a bosonic operator,
	which means it commutes on different sites. On the same site, $U^\dagger U = U U^\dagger = I$ so that it is unitary.
	But note that if $n$ is finite, then (on the same site) $[E, U]\neq U$ due to the ``border'' basis states;
	note that it's not equal but ``almost'' equal, and by that I mean equal on states of the basis other than the border ones.
	Now, as for the physical jargon, we need an operator $A$, such that $[E, A] = iI$ where $i=\sqrt{-1}$ (same site here),
	 so that we need the ``counterpart'' magnetic field $A$
	to the electric field $E$. We already know (see previous bosonic example, cross ref. here FIXME TODO) that this $A$ doesn't
	exist if $n$ is finite. But it exists when $n$ is infinite, as we will see.
	But before that let me define the exponentiation of $E$ as $V=\exp(i \beta E)$ (one site only, and extend by tensor products), and
	I need to think about the $\beta$ variable, let's set it to one for now $\beta\equiv 1$.

	It may be better to work on just one site with $N=1$ for this. Prove that the $V$ above exists and is unique for $n$ finite.
	Prove that if $n$ is finite, then there exists a unique $A$ such that $U=\exp(i \alpha A)$, where I need to adjust the constant $\alpha$ which
	may depend on $n$, and let just set it to 1 $\alpha\equiv1$ for now. This is all on one fixed site.
	This is just saying that the exponential and logarithm of certain matrices exist.
	I guess I could put the exercise: Find the generators of the algebra that includes $U, V, E, A$,  and $H$ as defined above, \emph{and} the identity $I$.
	What's the dimension of this algebra. Note that $H$ is symmetric, and find its eigenvalues.

	Now we take the $n$ to infinity but keep $N$ finite. We can even take $N=1$ for simplicity.
	Prove that  $[E, A] = iI$ and that equivalently $[E, U]= U$ so that we have the QED ``actual'' commutations.
	I think we can say that $U$ is a representation of the infinite group $U(1)$, so we have even that.
	I guess we'd like to show that computations done for $n$ finite converge to this $n$ infinity case that is separable, and does
	represent QED in one spacial dimension (and is equivalent to QED 1+1 in Lagrangian formalism, where the last 1 is the time dimension).

	What happens if we take both limits $n$ to infinity but also $N$ to infinity? We may want to postpone this question for latter, not sure.

	I'm not sure if going to 2D now, which is the smallest dimension that has plaquettes... or I should perhaps introduce fermions to this 1D case.
	So the order of the below sections may change.


	\section{Pure Gauge in 2D}
	TBW
	\section{Gauge and Matter in 1D}
	TBW

%\end{document}

\section{Questions}
\subsection{For a discussion}

\subsection{Just for myself}

In one article found this: the totality of numbers associated with a given observable is called its \emph{spectrum}. And indeed, if we regard the algebra of continuous functions as a Banach algebra, then the spectrum of anything is its image; in quantum mechanics, we indeed look at the eigenvalues, which from Banach algebras point of view form their spectra.

In Segal's article I saw that a product of two observables may not be an observable, so he proposed an algebraic structure where a multiplication is not assumed but only powers of elements.

Also from one article:
More recent studies have indicated that all self-adjoint operators may not be adequate as the model for the algebra of observables of every physical system. The $C*$-algebras are a long step from this special model, but still not into the chaos of abstract structures consistent with the general features of physical systems. 

From Kadison:
Given a state $f$ and an observable $A$, the value $f(A)$ is the expectation of the observable $A$ when these two have physical meanings.

\subsection{Notes from conversations}
\subsubsection{07/08}
We might want:
\begin{enumerate}
\item Prove that the limits for the Wilson loop taken in different orders yield different results;
\item Is there a gap between the ground state and the first excited state in QED 1+1?
\item Look at the papers of people who tried to prove these statements in other models.
\end{enumerate}
In free fermions, the gap is $O(N^{-1})$. \emph{Spontaneous symmetry breaking} means that the ground state is \emph{degenerate} (i.e., the corresponding eigen-space is of dimension bigger than 1). One can indeed encounter this phenomenon in nature. There are arguments from thermodynamics (not rigorous) about phase transitions (the most challenging problems).
\section{Appendix}
\subsection{$C^*$-algebras}
I think a few results from this theory are worth mentioning in the notes. Recall the definition of a $C^*$-algebra itself:
\begin{definition}
A (unital) $C^*$-algebra $A$ is a Banach space over $\mathbb C$ that is also a (unital) algebra such that the multiplication is a bounded bilinear map of norm $1$. It's also supplied with an involution $*:A\rightarrow A$ such that $\|a^*a\| = \|a\|^2$ for all $a \in A$. 
\end{definition}
A straightforward\footnote{From the boundedness of the multiplication we obtain $\|a\|^2 = \|a^*a\| \leq \|a^*\|\|a\|$, hence $\|a\| \leq \|a\|^*$. Switching to $a \mapsto a^*$, get $\|a\| = \|a\|^*$. For the unit, $\|1\| = \|1\|^2$ since $1^* = 1$.} consequence from the axioms is that $\|a\| = \|a^*\|$ and in the unital case $\|1\| = 1$.

The results of interest, I think, are the following:
\begin{theorem}
(Gelfand-Naymark, 1st) We have the following:
\begin{enumerate}[(i)]
\item Any possibly non-unital $C^*$-algebra is isomorphic to the space $C_0(\Omega)$ of continuous functions vanishing\footnote{Precisely, a function $\varphi : \Omega \rightarrow \mathbb C$ is vanishing at infinity if for any $\varepsilon >0$ there exists a compact subset $\Delta \subseteq \Omega$ such that $|\phi(t)|< \varepsilon$ when $t \notin \Delta$.} at $\infty$ on some locally compact Hausdorff topological space $\Omega$;
\item Any unital $C^*$-algebra is isomorphic to $C(\Omega)$ for $\Omega$ a compact Hausdorff space.
\end{enumerate}
\end{theorem}

There's an accurate description of both the isomorphism and the space $\Omega$: the space $\Omega$ is the Gelfand spectrum of $A$, and the isomorphism is the Gelfand representation.

Here are the definitions. Let $A$ be a Banach algebra\footnote{I.e. a Banach space that's also an algebra such that the multiplication is a bounded bilinear map with norm $1$.}. Its \emph{Gelfand spectrum} is the space $\Omega \subset A^*$ of all characters\footnote{A character is a non-zero linear functional that preserves the multiplication. It turns out that any character $\chi : A \rightarrow \mathbb C$ has norm $\|\chi\| \leq 1$, so they're automatically continuous, for if $\chi(a) = 1$ for some $a \in A$ with $\|a\| <1$, then let $b:=\sum_{k=1}^\infty a^k$. It follows that $a + ab = b$, hence $1 + \chi(b) = \chi(b)$, which is absurd. } of $A$  that is endowed with $\omega^*$-topology induced from $A^*$. The isomorphism then is the Gelfand representation $\Gamma_A : A \rightarrow C_0(\Omega)$ that sends an element $a \in A$ to a continuous map (vanishing at $\infty$) $\hat a$ such that\footnote{The continuity of $\hat a$ follows right from the definition of $\omega^*$-topology. It indeed vanishes at $\infty$, for $\hat a$ is a continuous map on the compactification $\Omega^\prime := \Omega \cup \{0\}$ such that $\hat{a}(0) = 0$ (look at the formula defining $\hat{a}$), hence for any $\varepsilon > 0$ there's $U_{\varepsilon} \ni 0$ such that $|\hat{a}(\chi)| < \varepsilon$ when $t \in U_{\varepsilon}$. But $U_{\varepsilon}^c$ is compact, as a closed subset of the compact space $\Omega^\prime$, so $\hat a$ vanishes at $\infty$.}
\begin{proposition}
The Gelfand spectrum $\Omega$ of a Banach algebra $A$ is a locally compact Hausdorff space; if $A$ is in addition unital, then $\Omega$ is Hausdorff and compact.
\end{proposition}
\begin{proof}
Obviously, $\Omega$ is Hausdorff since $\omega^*$-topology is Hausdorff. Consider $\Omega^\prime := \Omega \cup \{0\}$. If $\chi_\alpha \rightarrow f$ in $\omega^*$-topology, then obviously $f$ preserves multiplication (but it might become zero), so $\Omega^\prime$ is a closed subset of the unit ball in $A^*$. Now it's the consequence of Banach-Alaoglu theorem that $\Omega^\prime$ is compact. Being a subset of a compact Hausdorff space, we see that $\Omega$ is locally compact and Hausdorff. In case $A$ is unital, $0$ is an isolated point of $\Omega^\prime$, for if there was a net $\chi_\alpha \in \Omega$ such that $\chi_\alpha \rightarrow 0$, then $1 = \chi_\alpha(1) \rightarrow 0$, which is a contradiction, and thus $\Omega$ is compact.
\end{proof}

In general, for possibly non-commutative $C^*$-algebras we have 
\begin{theorem}
(Gelfand-Naymark, 2nd) Any $C^*$-algebra is $*$-isometrically isomorphic to $\End(H)$ for some Hilbert space $H$.
\end{theorem}
\begin{fur}
Might understand the construction of this operator algebra in detail. I've seen that the proof is constructive. That's the matter of finding pure states and defining the corresponding irreducible representations. The direct sum of those will yield the result.
\end{fur}

\begin{thebibliography}{10}
\bibitem{antoine} Antoine, J-P. "Quantum mechanics beyond Hilbert space." Irreversibility and Causality Semigroups and Rigged Hilbert Spaces. Springer, Berlin, Heidelberg, 1998. 1-33.
\bibitem{arveson} Arveson, William. An invitation to C*-algebras. Vol. 39. Springer Science \& Business Media, 2012.
\bibitem{bratteli} Bratteli, Ola. Inductive limits of finite-dimensional $C^*$-algebras. Transactions of the American Mathematical Society 171 (1972): 195-234.\label{bratteli}
\bibitem{chari} Chari, Vyjayanthi, and Andrew N. Pressley. A guide to quantum groups. Cambridge university press, 1995.
\bibitem{conway} Conway, John B. A course in functional analysis. Vol. 96. Springer, 2019.
\bibitem{gelfand} Gelfand, Izrail Moiseevich, and Georgii Evgenevich Shilov. Generalized functions, Vol. 4: applications of harmonic analysis. Academic Press, 1964.
\bibitem{friedli} S. Friedli and Y. Velenik. Statistical Mechanics of Lattice Systems: A Concrete Mathematical Introduction. Cambridge University Press, 2017.
\bibitem{haag_alg} Haag, Rudolf, and Daniel Kastler. "An algebraic approach to quantum field theory." Journal of Mathematical Physics 5.7 (1964): 848-861.
\bibitem{haag} Haag, Rudolf, R. V. Kadison, and Daniel Kastler. "Nets of $C^*$-algebras and classification of states." Communications in Mathematical Physics 16.2 (1970): 81-104.
\bibitem{israel} Israel, Robert B. Convexity in the theory of lattice gases. Vol. 64. Princeton University Press, 2015.
\bibitem{segal} Segal, Irving E. "Irreducible representations of operator algebras." Bulletin of the American Mathematical Society 53.2 (1947): 73-88.
\end{thebibliography}
\end{document}