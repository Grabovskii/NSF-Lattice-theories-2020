%\documentclass{article}
%\newcommand{\Mod}[1]{\ \left(\mathrm{mod}\ #1\right)}
%\usepackage{braket}

%\begin{document}
	
	\section{The meaning of Gauge}
	The Quantum Ising model has a \emph{global} symmetry $\tilde{S}_i = -S_i$, that leaves
	the form of the Hamiltonian invariant in the new variables, and so it's the same system.
	By \emph{global} I mean independent of $i$.
	The Quantum Ising Gauge model (see Hunter L.'s write up) has a \emph{local} symmetry.
	
	\section{Gauge and Matter in 1D}
	\subsection{Definitions of the full space}
	We consider again a 1D chain, put bosons of dimension $n$ on the links, with group $Z(n)$ and
	define $E$, $A$, $V$ and $U$ as before. We won't use $A$ and $V$ in the Hamiltonian.
	This time we add fermions on the sites, where at site $l$, the operator $\psi_\ell^\dagger$
	creates a fermion, and the operator $\psi_\ell$ destroys a fermion at that site.
	The fermion operators \emph{anti-commute} on \emph{different} sites.
	Then let
		\begin{equation}\label{eq:1dQEDLatticeHamiltonian}
	H=-\frac{n}{2\pi} \sum_\ell (\psi_\ell^\dagger U_{\ell,\ell+1}\psi_{\ell+1}+H.c.) + 2m\sqrt{\frac{n}{2\pi}}\sum_\ell (-1)^\ell \psi_\ell^\dagger\psi_\ell+  \sum_{\ell} E_{\ell,\ell+1}^2,		
	\end{equation}
	where $n$ is an odd non-negative integer, and $m$ is a real number. We use the following version of the chain with a free end on the left that will determine the states of all the other links under the Gauss law:
	\begin{equation}\label{eq:lchain}
	\xymatrix{
	\ar@{-}[r]^{-1} & \bigcirc^0 \ar@{-}[r]^{0} &  \bigcirc^1 \ar@{-}[r]^{1} & \cdots & \ar@{-}[l]_{N-2} \bigcirc^{N-1}
	}
	\end{equation}
	
%\noindent \textbf{A discussion on the boundary conditions}. I attempted to use the condition
	%\[
	%\xymatrix{
	%\ar@{-}[r]^{-1} & \bigcirc^0 \ar@{-}[r]^{0} &  \bigcirc^1 \ar@{-}[r]^{1} & \cdots & \ar@{-}[l]_{N-2} \bigcirc^{N-1}
	%}
	%\]
	%where the numbers indicate the corresponding indices of links and sites. So, we specify the state on the leftmost link, which is $b_{-1}$, and this determines the states of all the other links in the physical space. However, there is nothing in the Hamiltonian that might change $b_{-1}$, it always stays the same, but all the other $b_i$'s might change (the first term of the Hamiltonian to make sense requires sites on each side of the link). This means that we will never land in the physical space. We might, however, be pleased with something like
		%\[
	%\xymatrix{
	%\square\ar@{-}[r]^{-1} & \bigcirc^0 \ar@{-}[r]^{0} &  \bigcirc^1 \ar@{-}[r]^{1} & \cdots & \ar@{-}[l]_{N-2} \bigcirc^{N-1}
	%}
	%\]
	%with the meaning of $\square$ to be an abyss that swallows any fermion that hops into it. For now, I will attempt the condition
			%\[
	%\xymatrix{
%\bigcirc^0 \ar@{-}[r]^{0} &  \bigcirc^1 \ar@{-}[r]^{1} & \cdots & \ar@{-}[l]_{N-2} \bigcirc^{N-1},
	%}
	%\]
	%so I don't have any boundary links.
	
	The full space on $N$ sites has dimension $(2n)^{N}$, the $2^N$ comes from the fermions (electrons and positrons), 
	and the $n^N$ from the bosons (phonons). The actual physical space we work with is specified by the kernel of the Gauss law (see below).
	
	A basis element at site $l$ is denoted as $|n_l, b_l\rangle$, where $n_l \in \{0,1\}$ indicates the presence of a fermion at site $l$ and $b_l \in \{0,\ldots,n-1\}$ indicates the state of the boson to the right of the fermion. Such basis is called \emph{computational}.
	
	Recall how the operators act. Let $|n_0 \, n_1\, \ldots \, n_{N-1}\, b_{-1}\, b_0 \, \ldots \, b_{N-2}\rangle$ be an arbitrary basic state. Then
	\begin{enumerate}[(a)]
	%\item $\psi^\dagger_\ell |0,k\rangle = |1,k\rangle$, $\psi^\dagger_l |1,k\rangle = 0$;
	\item \[\begin{split}
	&\psi^\dagger_\ell |n_0 \, n_1\, \ldots \, n_{N-1}\, b_{-1}\, b_0 \, \ldots \, b_{N-2}\rangle =  \\
&=(-1)^{n_0 + n_1 + \cdots + n_{\ell-1}}(1-n_\ell)|n_0 \, n_1\, \ldots\, n_\ell=1\, \ldots \, n_{N-1}\, b_{-1}\, b_0 \, \ldots \, b_{N-2}\rangle;
	\end{split}
	\]
	\item %$\psi_l|0,k\rangle = 0$, $\psi_l|1,k\rangle = |0,k\rangle$;
	\[\begin{split}
	&\psi_\ell |n_0 \, n_1\, \ldots \, n_{N-1}\, b_{-1}\, b_0 \, \ldots \, b_{N-2}\rangle =  \\
&=(-1)^{n_0 + n_1 + \cdots + n_{\ell-1}}n_\ell|n_0 \, n_1\, \ldots\, n_\ell=0\, \ldots \, n_{N-1}\, b_{-1}\, b_0 \, \ldots \, b_{N-2}\rangle;
	\end{split}
	\]
	\item %$\psi^\dagger_l \psi_l |0, s\rangle = 0$ and $\psi^\dagger_l \psi_l |1,s\rangle = |1,s\rangle$;
	\[
	\psi^\dagger_{\ell} \psi_{\ell} |n_0 \, n_1\, \ldots \, n_{N-1}\, b_{-1}\, b_0 \, \ldots \, b_{N-2}\rangle = n_\ell |n_0 \, n_1\, \ldots \, n_{N-1}\, b_{-1}\, b_0 \, \ldots \, b_{N-2}\rangle;
	\]
	\item For $n = 2s+1$ odd, %$E_{l,l+1}|c,k\rangle = (k-s)|c,k\rangle$. So, $E_{l,l+1}^2|c,k\rangle = (k-s)^2 |c,k\rangle$;
	\[
	E_{\ell,\ell+1} |n_0 \, n_1\, \ldots \, n_{N-1}\, b_{-1}\, b_0 \, \ldots \, b_{N-2}\rangle = (b_{\ell}-s) |n_0 \, n_1\, \ldots \, n_{N-1}\, b_{-1}\, b_0 \, \ldots \, b_{N-2}\rangle;
	\]
	\[
	E_{\ell,\ell+1}^2 |n_0 \, n_1\, \ldots \, n_{N-1}\, b_{-1}\, b_0 \, \ldots \, b_{N-2}\rangle = (b_\ell -s)^2 |n_0 \, n_1\, \ldots \, n_{N-1}\, b_{-1}\, b_0 \, \ldots \, b_{N-2}\rangle;
	\]
	\item \[\begin{split}
	U_{\ell,\ell+1} &|n_0 \, n_1\, \ldots \, n_{N-1}\, b_{-1}\, b_0 \, \ldots \, b_{N-2}\rangle =\\= &|n_0 \, n_1\, \ldots \, n_{N-1}\, b_{-1}\, b_0 \, \ldots \, b_{\ell}+1\, \ldots\, b_{N-2}\rangle
	\end{split}
	\]%$U_{l,l+1} |c,k\rangle = |c,k+1\rangle$;
	\item %Let $|n_0\, n_1\, \ldots \, n_{N-1} \, b_{-1}\, b_0\, \ldots\, b_{N-1}\rangle$ be a state. Then
	\begin{equation}\label{eq:ham_fir_t}\begin{split}
	\psi_\ell^\dagger U_{\ell,+1} \psi_{\ell+1} &|n_0\, n_1\, \ldots \, n_{N-1} \, b_{-1}\, b_0\, \ldots\, b_{N-2}\rangle=\\ = n_{\ell+1} (1-n_\ell) &|n_0\, n_1\, \ldots \, n_{\ell}=1 \, n_{\ell+1} = 0\,\ldots \, n_{N-1} \, b_{-1}\, b_0\, \ldots\,b_\ell + 1\,\cdots \, b_{N-2}\rangle.
	\end{split}
	\end{equation}
	So, when we have a fermion at site $l+1$ and no fermion at site $l$, the operator $\psi_l^\dagger U_{l,l+1} \psi_{l+1}$ shifts the fermion to site $l$ and increments the state of the boson on the corresponding lift. Similarly, its Hermitian conjugate acts as
		\[\begin{split}
	\psi_{\ell+1}^\dagger U_{\ell,\ell+1}^\dagger \psi_{\ell} &|n_0\, n_1\, \ldots \, n_{N-1} \, b_{-1}\, b_0\, \ldots\, b_{N-2}\rangle=\\ = n_{\ell} (1-n_{\ell+1}) &|n_0\, n_1\, \ldots \, n_{\ell} = 0\, n_{\ell+1} = 1\, \ldots\, n_{N-1} \, b_{-1}\, b_0\, \ldots\,b_\ell -1 \,\cdots \, b_{N-2}\rangle.
	\end{split}
	\]
	\end{enumerate}
	%\item Now let's observe how the Hamiltonian acts on states. 
	%\[
	%\xymatrix{
	%\bigcirc \ar@{-}[rr]& & \bigcirc \\
	%& \bigcirc\ar@{-}[ru] \ar@{-}[lu] &
	%}
	%%\xymatrix{
	%%\bigcirc \ar@{-}[r] &	\bigcirc \ar@{-}[r] &	\bigcirc 
	%%\ar@{-}@/^1pc/[ll]
	%%}
	%\]
	%Let $|n_0,n_1,n_2,b_0,b_1,b_2\rangle$ be a state. Let's see what parts of $H$ do to this state. Due to the rotational symmetry, there are actually only four distinct situations (all that matters is the number of fermions). So, 
	%\[\begin{split}
%&-\frac{n}{\pi}\sum_{l=0}^2 (\psi^\dagger_l U_{l,l+1} \psi_{l+1}) |n_0,n_1,n_2,b_0,b_1,b_2\rangle =\\ &= \begin{cases}
%0, \ \ \text{no fermions or exactly three}\\
%\text{Set } n_{l} = 0, \ n_{l-1} = 1, \ b_{l-1} \mapsto b_{l-1} + 1 \ \text{if there is a fermion at site }l \text{ and no fermion at site }l-1
%\end{cases}
%\end{split}
	%\]
	%\end{enumerate}
	%For $2\tilde{m}\sum_{l} (-1)^l \psi_l^\dagger \psi_l$, we have
	%\[
	%(2\tilde{m}\sum_{l} (-1)^l \psi_l^\dagger \psi_l) |n_0,n_1,n_2,b_0,b_1,b_2\rangle  = 2 \tilde{m} \cdot F \cdot |n_0,n_1,n_2,b_0,b_1,b_2\rangle,
	%\]
	%where 
	%\[
	%F = \begin{cases}
	%0, \ \ \text{there are two fermions and one of them is at site }1;\\
	%(-1)^l, \ \ \text{there is only one fermion and located at } l;\\
	%2, \ \ \text{the fermions are at sites }0\text{ and }2;\\
	%
	%\end{cases}
	%\]
	%Lastly,
	%\[
	%\sum_l E^2_{l,l+1} |n_0,n_1,n_2,b_0,b_1,b_2\rangle = 2|n_0,n_1,n_2,b_0,b_1,b_2\rangle.
	%\]
	%Now let's see what happens for some specific states. If there's only one fermion, then
	%\[
	%H|1,0,0; b_0,b_1,b_2\rangle = |0,0,1; b_0,b_1,b_2+1\rangle + (2\tilde{m}+2)|1,0,0; b_0,b_1,b_2\rangle 
	%\]
	%and the rest is rotationally symmetric.
	%
	
	From the letter: in QED fermions (matter) occupy even sites, and anti-fermions (anti-matter) occupy odd sites. The same for QCD, except that fermions in QED are electrons, and fermions in QCD are quarks.
	
	\subsection{Gauss Law and Physical Space}
	Let the Gauss law be
		\begin{equation}\label{eq:1dQED_Gauss_law}
	G_\ell := \psi_\ell^\dagger \psi_\ell + \frac{1}{2}((-1)^\ell -1)-(E_{\ell,\ell+1}-E_{\ell-1,\ell}).
	\end{equation}
	We restrict the full space to the physical space of states $|\psi\rangle$ that satisfy
	$G_\ell|\psi\rangle= 0 \Mod{n}$. 
	The physical space has dimension $n 2^N$; see Hunter L's explanation for why this is the case.
	
	I wanted to expand a bit on the Gauss law. So, all pieces that comprise $G_\ell$ are diagonal, hence $G_\ell$ is diagonal itself and its kernel is spanned by some of the basic kets. If $|n_0\, n_1\, \ldots \, n_{N-1} \, b_{-1}\, b_0\, \ldots\, b_{N-2}\rangle$ is such, then the action of $G_\ell$ upon it is given by
	\[\begin{split}
	G_\ell|n_0\, n_1\, \ldots \, n_{N-1} \, b_{-1}\, b_0\, \ldots\, b_{N-2}\rangle =\\= (n_{\ell} + \frac{1}{2}((-1)^\ell-1) + b_{\ell-1}-b_\ell) |n_0\, n_1\, \ldots \, n_{N-1} \, b_{-1}\, b_0\, \ldots\, b_{N-2}\rangle,
	\end{split}
	\]
We see that $b_\ell$ is determined by $b_{\ell-1}$ via
	\[
	b_\ell = b_{\ell-1} + n_{\ell} + \frac{1}{2}((-1)^\ell-1).
	\]
	So the physical space is 
	\[
	\Sp ( |n_0\, n_1\, \ldots \, n_{N-1} \, b_{-1}\, b_0\, \ldots\, b_{N-2}\rangle \ | \ b_{\ell} =b_{\ell-1} + n_{\ell} + \frac{1}{2}((-1)^\ell-1) \ \text{for every} \ \ell).
	\]
	Recursively solving the equation for $b_{\ell}$, we find that
	\[
	b_{\ell} = b_{-1} + n_0 + n_1 + \cdots + n_{\ell} - \left[\frac{\ell+1}{2}\right],
	\]
	where $[ \, ]$ denotes the integer part.
	
	\begin{proposition}
Each term in the Hamiltonian defined as in \eqref{eq:1dQEDLatticeHamiltonian} on the chain \eqref{eq:lchain} with a free left link leaves the physical space invariant.
	\end{proposition}
	\begin{proof}
	The first term that shifts a fermion to the left acts according to the formula \ref{eq:ham_fir_t}. If $n_\ell = 0$ and $n_{\ell+1} = 1$, then the term acts non-trivially and the state on $\ell$th link is updated according to
	\[
	b_\ell + 1 = b_{-1} + n_0 + n_1 + \cdots + (n_\ell+1) -\left[\frac{\ell+1}{2}\right],
	\]
	so we simply add $1$ to both sides. For $k \geq 1$, the equation of states on the other links are also preserved:
	\[
	b_{\ell+k} = b_{-1} + n_0 + \cdots + (n_{\ell}-1) + (n_{\ell+1}+1) + \cdots + n_{\ell +k} -\left[\frac{\ell+1}{2}\right].
	\]
	Similarly for Hermitian conjugate term (now subtract $1$ from both sides in the equation for $b_{\ell}$).
	The other terms in the Hamiltonian are diagonal, so they obviously preserve the physical space. Thus the Hamiltonian itself does preserves the physical space.
	\end{proof}
	
	\subsection{Limits}
	Ercolessi et al \cite{ercolessi} numerically take both limits \emph{first} the $N\rightarrow\infty$ limit
   and then the $n\rightarrow\infty$ limit and find a ``phase transition'' at some value of $\tilde{m}$. Order of the limits here FIXME TODO.
	For now, perhaps is enough to take $N\rightarrow\infty$ for $n$ finite, and prove that the limiting (topological, C*, Von Neuman??)
	algebra exists, and the limiting $H$ exits.
	To define a phase transition I need an ``order parameter'' which will be something like a magnetization in a spin model.
	Let $W=N^{-1}\sum_l E_{l, l+1}$ be the ``Wilson loop'' operator. Let $|gs\rangle$ be the ground state of $H$.
	Let $w_1(\tilde{m}, n, N) = \langle gs | W |gs\rangle$, and let $w(\tilde{m}, n)=\lim_{N\rightarrow\infty} w_1(\tilde{m}, n, N)$.
	 This quantity $w$ as a function of $\tilde{m}$ will have a first derivative everywhere,
	except for a value of $\tilde{m}$, where the left and right derivatives both exist but are different.
	We say that the model goes through a phase transition at that value of $\tilde{m}$. Obviously, that value will be dependent on $n$,
	and we are interested in $n\rightarrow\infty$.
	
	\subsection{Staggered configurations}
 We define two \emph{staggered} states: $|\st_1(b_{-1})\rangle$ and $|\st_2(b_{-1})\rangle$ in the following way: for the first one, all even sites are occupied and odd are empty; for the second one, all even sites are empty and all odd sites are occupied. The states on the links are determined by $b_{-1}$, which we include in the definition. 
\begin{statement}
For $|\st_2(b_{-1})\rangle$, $
b_{\ell} = b_{-1}$; for $|\st_1(b_{-1})\rangle$, $
b_{\ell} = \begin{cases}
b_{-1} + 1, \ \ell = 2k;\\
b_{-1}, \ \ell = 2k+1.
\end{cases}$. Pictorially,
\[
|\st_2(b_{-1})\rangle =	\xymatrix{
	\ar@{-}[r]^{b_{-1}} & \bigcirc \ar@{-}[r]^{b_{-1}} &  \bigotimes \ar@{-}[r]^{b_{-1}} & \cdots & \ar@{-}[l]_{b_{-1}} ?
	}
\]
\[
|\st_1(b_{-1})\rangle =	\xymatrix{
	\ar@{-}[r]^{b_{-1}} & \bigotimes \ar@{-}[r]^{b_{-1}+1} &  \bigcirc \ar@{-}[r]^{b_{-1}} & \cdots & \ar@{-}[l]_{b_{-1}+?} ?
	}
\]
where $\otimes$ means that the site is occupied. This is a simple computation using aforementioned formulas derived from the Gauss law.
\end{statement}

\subsection{Wilson loop for $m \gg 0$}
Split the Hamiltonian as $H = A + 2m\sqrt{n/2\pi} D$, where $A$ is the sum of the first term and the third terms of $H$ and $D = \sum_l (-1)^l \psi_l^\dagger \psi_l$.
\subsubsection{Dependence on $m$}
\begin{proposition}\label{p:dep_m}
Let $\lambda_1$ be the eigen-value of $H$ that corresponds to a ground state. Then $\lambda_1$ is a concave $C^1$-function of the mass. The ground state, as a function, can be chosen so that it's $C^1$.
\end{proposition}
\begin{proof}
Concavity is a consequence of Proposition \ref{p:conv_le}; differentiability of $\lambda_1$ is a consequence of Theorem \ref{thm:pert}.
\end{proof}

\noindent \textbf{Caution:} one cannot choose ground states continuously (with respect to the mass). It's easy to see for odd $N$ (say, $N=3$): take limits $m\rightarrow \pm \infty$ and see that the number of fermions is not preserved. If a continuous choice was possible, it would be preserved. Also, a theorem that guarantees continuity of eigen-vectors requires holomorphicity of $H$. The Hamiltonian is indeed holomorphic, but for complex values of $m$ it's not Hermitian, so that theorem is not applicable.

\subsubsection{Estimate for $\lambda_1$}
\begin{proposition}\label{p:lambda_est_pos}
Let $\lambda_1$ be the eigen-value of $H$ that corresponds to a ground state. Then
\[
\lim_{m \rightarrow \infty} \frac{\lambda_1}{m} = -2\left[\frac{N}{2}\right]\sqrt{\frac{n}{2\pi}}.
\]
Also, for $m > 0$,
\[
\lambda_1(m) < -2m\left[\frac{N}{2}\right]\sqrt{\frac{n}{2\pi}}
\]
(note: the inequality is strict!). The limit implies that for $m \gg 0$ and $N$ even (or large), the lowest energy level can be approximated as 
\[
\lambda_1(m) \approx - Nm \sqrt{\frac{n}{2\pi}}.
\]
\end{proposition}
\begin{proof}
Let $m > 0$. From the definition of $\gs_2$, we see that
\[
m^{-1}\langle \st_2(s) | H| \st_2(s) \rangle \geq m^{-1}\langle \gs | H| \gs \rangle = m^{-1}\min_{\|\psi\|=1} \langle \psi |H|\psi\rangle \geq m^{-1}\min_{\|\psi\|=1} \langle \psi |A |\psi\rangle + 2\sqrt{\frac{n}{2\pi}} \langle \st_2(s) | D| \st_2(s)\rangle 
\]
%Note that 
%\[
%\lim_{m\rightarrow \infty} m^{-1}\langle \st_2(s) | H|\st_2(s) \rangle = 2\sqrt{\frac{n}{2\pi}} \langle \st_2(s) | D| \st_2(s)\rangle
%\]
%and
Note that
\[
\langle \st_2(s) | H|\st_2(s) \rangle = 2m\sqrt{\frac{n}{2\pi}} \langle \st_2(s) | D + \sum_l E_{l,l+1}^2 | \st_2(s)\rangle = -2m\left[\frac{N}{2}\right]\sqrt{\frac{n}{2\pi}}
\]
so, taking the limit in the above series of inequalities, we obtain the result.
%Hence, taking the limit in the above inequality, we obtain the statement. With regards to $\lambda_1$, we write $\langle \gs | H| \gs \rangle = \lambda_1\langle \gs |\gs\rangle = \lambda_1$.
With regards to the strict inequality for $\lambda_1$, $|\st_2(s)\rangle$ is clearly not an eigen-value of $H$ when $m$ is finite, so $\langle\st_2(s)|H|\st_2(s)\rangle$ cannot equal $\lambda_1$ (see the footnote\footnote{Let $|\psi_k\rangle$ constitute an orthonormal basis of eigen-vectors of $H$ (they can be chosen to have real components with respect to the standard basis for $H$ is hermitian and symmetric). Then $|\st_2(s)\rangle = \sum_{k} \alpha_k |\psi_k\rangle$ and $\sum_{k}\alpha_k^2 = 1$ (all $\alpha_k$'s are real). On the other hand, applying $\langle \st_2 |H$ to both sides yields $\lambda_1 = \sum_{k} \alpha_k^2 \lambda_k$. Since $|\st_2(s)\rangle$ is not an eigen-vector of $H$, there's $\lambda_j > \lambda_1$ such that $\alpha_j \neq 0$. We want to show that the above equality is not possible. Indeed, let $m$ be the index of the last $\lambda_i$ that is equal to $\lambda_1$. We can rewrite the equation as $\lambda_1 = \sum_{k=m+1}^n \frac{\alpha_k^2}{1-\sum_{i=1}^m \alpha_i^2} \lambda_k$, which is again a convex combination, so write $\lambda_1 = \sum_{k=m+1}^n \beta_k\lambda_k$. We know that $\lambda_1 < \lambda_{k}$ for every $k> m$. But then $\beta_k\lambda_1 \leq \beta_k\lambda_k$ for $k > m$, and for $j$ the inequality is strict: $\beta_j\lambda_1 < \beta_j\lambda_j$; summing up, we get $\sum_{k=m+1}^n \beta_k\lambda_1 < \sum_{k=m+1}^n \beta_k \lambda_k$, which contradicts the equality. Thus $\lambda_1 \neq \langle \st_2(s)|H|\st_2(s)\rangle$.

} for a detailed theoretical explanation).
\end{proof}
\noindent \textbf{Remark.} Notice that L'Hopitals rule gives us an estimate for the derivative $\lambda_1^\prime$ with respect to $m$:
\[
\lim_{m \rightarrow \infty} \lambda_1^\prime(m) = -2\left[\frac{N}{2}\right] \sqrt{\frac{n}{2\pi}}.
\]

\subsubsection{Invariance properties of $H$}
Recall that the physical space (determined by the Gauss law) is given by
	\[
	P = \Sp ( |n_0\, n_1\, \ldots \, n_{N-1} \, b_{-1}\, b_0\, \ldots\, b_{N-2}\rangle \ | \ b_{\ell} =b_{\ell-1} + n_{\ell} + \frac{1}{2}((-1)^\ell-1) \ \text{for every} \ \ell).
	\]
	We can split $P$ into the direct sum according to what value of $b_{-1}$ is:
	\[
	P = P_0 \oplus P_1 \oplus \cdots \oplus P_{n-1},
	\]
	where
	\[
	P_j = \Sp( |n_0\, n_1\, \ldots \, n_{N-1} \, b_{-1}\, b_0\, \ldots\, b_{N-2}\rangle \in P \ | \ b_{-1} = j).
	\]

Also, let $N:= \sum_{l} \psi_l^\dagger \psi_l$ be the operator that counts fermions.

The following lemma is an obvious but very useful observation:

\begin{lemma}\emph{(Invariance properties of $H$)} The following hold:
\begin{enumerate}[(i)]
\item The restriction of $H$ to each subspace $P_j$. Denote $H_j := H|_{P_j}$.
\item $[N,H] = 0$. In particular, for a chosen $|\gs\rangle$, the basic kets $|\psi\rangle$ for which $\langle \psi | \gs \rangle \neq 0$ all have the same number of fermions.
\end{enumerate}
\end{lemma}

\subsubsection{Ground states for large mass}

\begin{lemma} \label{l:gs_conv_pos}
The following hold:
\begin{enumerate}
\item For any basic ket $|\psi\rangle$ such that $\langle \psi |D|\psi\rangle > -[N/2]$ (i.e., $|\psi\rangle \neq |\st_2(i)\rangle$ for any $i$), we have $\lim_{m \rightarrow \infty} \langle \gs | \psi \rangle = 0$;
\item For\footnote{By this I mean that there exists $m_0 > 0$ such that for all $m \geq m_0$ the statement is true.} $m \gg 0$ we have $\left\langle \gs | \st_2(s) \right\rangle \neq 0$. In particular, if for every large enough mass we choose $|\gs\rangle$ so that $\langle \gs | \st_2(s) \rangle > 0$ (which is possible), then $\lim_{m \rightarrow \infty} |\gs\rangle = |\st_2(s)\rangle$. 
\item As a consequence of the above, for $m \gg 0$, $|\gs\rangle \in P_s$, and the number of fermions in each non-zero component of $|\gs\rangle$ is equal to $[N/2]$.
\end{enumerate}
\end{lemma}
\begin{proof}
\begin{enumerate}
\item Indeed, we see that
\[
\langle \psi | \gs \rangle = \frac{1}{\lambda_1} \langle \psi | H|\gs\rangle = \frac{1}{\lambda_1} \langle \psi |A|\gs\rangle + \frac{2m}{\lambda_1} \sqrt{\frac{n}{2\pi}} \langle \psi | D | \gs \rangle = \frac{1}{\lambda_1} \langle \psi |A|\gs\rangle + \frac{2m}{\lambda_1} \sqrt{\frac{n}{2\pi}} \cdot a \langle \psi | \gs \rangle
\]
where $a$ is an eigen-value of $D$ (which is not equal to $-[N/2]$ by assumption). The above can be rewritten as
\[
\langle \psi | \gs\rangle = \frac{1}{\lambda_1} \frac{1}{1- \frac{2ma}{\lambda_1}\sqrt{\frac{n}{2\pi}} } \langle \psi | A | gs\rangle.
\]
We see that 
\[
1- \frac{2ma}{\lambda_1}\sqrt{\frac{n}{2\pi}} \rightarrow 1 + a [N/2] \neq 0
\]
therefore
\[
\lim_{m \rightarrow \infty}\langle \psi | \gs \rangle = 0.
\]
\item Since each $P_i$ is invariant under $H$, we find $j$ such that $|\gs \rangle \in P_j$, so we can write it as $|\gs\rangle = \alpha |\st_2(j)\rangle + |v\rangle$, where $|v\rangle \perp |\st_2(j)\rangle$. The above implies together with $\rangle \gs | \gs \langle = 1$ implies
\[
\lim_{m \rightarrow \infty} (1 - |\alpha|^2) = 0
\]
or $\lim_{m \rightarrow \infty} |\langle \gs |\st_2(j)\rangle| = 1$. If we made the choice of ground states so that $\alpha > 0$, then $\lim_{m\rightarrow \infty} \alpha = 1$, so $|\gs\rangle \rightarrow |\st_2(j)\rangle$.

Let's prove that $j = s$. Write $|\gs\rangle = \alpha |\st_2(j)\rangle + \beta |v\rangle$ and $|\gs^\prime\rangle = \alpha |\st_2(s)\rangle + \beta |v^\prime \rangle$. The vector $|v^\prime \rangle$ is totally the same as $|v\rangle$ except that we substitute everywhere the bosonic state $b_{-1} = j$ with the state $b_{-1} = s$. The constants $\alpha$ and $\beta$ can be chosen real, $\alpha > 0$, and $\alpha^2+\beta^2 = 1$. Then we compare the expectation values of the Hamiltonian:
\[
\langle \gs | H|\gs \rangle  = \alpha^2 (-2m[N/2] \sqrt{n/2\pi} + (N-1)(j-s)^2) + O(\beta),
\]
\[
\langle \gs^\prime | H|\gs^\prime \rangle  = \alpha^2 (-2m[N/2] \sqrt{n/2\pi}) + O(\beta),
\]
hence their difference is
\[
\langle \gs | H|\gs \rangle - \langle \gs^\prime | H|\gs^\prime \rangle = \alpha^2(N-1)(j-s)^2 + O(\beta).
\]
It is supposed to be always negative. However, we can make $O(\beta)$ as small as we want, whereas $\alpha^2$ tends to $1$ and is multiplied by something stubbornly positive. Therefore, the difference can stay negative iff $j = s$.
\end{enumerate}
\end{proof}

\begin{conj}
It is true for any $m > 0$ that $\langle \gs|\st_2(s)\rangle \neq 0$. Numerical experiments suggest even more: $\langle \gs | \st_2(s)\rangle > \langle \gs | \psi \rangle$ for any other basic ket $\psi$.
\end{conj}

%\begin{proposition}
%We have $|\gs\rangle \in P_s$ for \emph{any} $m$, and $N|\gs\rangle = [N/2] |\gs\rangle$.
%\end{proposition}
%\begin{proof}
%This is pure general topology. For any fixed ground state at a particular mass, we can find a $C^1$-curve $m \mapsto |\gs\rangle$ that passes through it. The image of the ground state curve lies in the disjoint union $P_1 \cap S^{\dim P} \cup \cdots \cup P_n \cap S^{\dim P}$. Since $|\gs\rangle$ is in particular continuous in $m$, the image of $\mathbb R$ lies only in one of the components of the union. Since for $m \gg 0$ it happens in $P_s\cap S^{\dim P}$, it's there for \emph{any} $m$.
%\end{proof}

\begin{proposition}
For $m \gg 0$, a ground state is non-degenerate.
\end{proposition}
\begin{proof}
Assume on the contrary that it is degenerate for $m \gg 0$. This means there's a sequence of values of mass $m_k \rightarrow \infty$ and sequences of ground states $|\gs_k\rangle$ and $|\gs^\prime_k\rangle$ such that $\langle\gs_k^\prime|\gs_k\rangle = 0$. However, passing to the limit in the slots of the Hermitian product yields $\langle \st_2(s) | \st_2(s) \rangle \neq 0$. This is a contradiction, and thus the ground state is unique for large $m$.
\end{proof}
%\begin{proof}[Heuristic evidence]
%I believe that $|\gs\rangle$, if chosen so that $\langle \gs | \st_2(s)\rangle \in [0,1]$, is a continuous function of $m$. The eigen-value $\lambda_1$ is continuous as well. But recall that $P = P_1 \oplus \cdots \oplus P_n$. The sets $S^{\dim P} \cap P_i$ (an intersection of the unit sphere with a subspace) are disjoint, and since $|\gs\rangle$ is continuous, it maps the whole real line $\mathbb R$ into only one such intersection, i.e. into $S^{\dim P} \cap P_s$. So $b_{-1} = s$ always has to be the bosonic state for $|\gs\rangle$.
%\end{proof}

\begin{proposition}
Let $\Sigma = N^{-1}\sum \langle E_{l,l+1} \rangle$ be the electric field operator. Then\footnote{No matter how we choose the ground states for every value of mass since we're under the Hermitian inner product.} $\lim_{m \rightarrow \infty} \Sigma = 0$.
\end{proposition}
\begin{proof}
Indeed, by Lemma \ref{l:gs_conv_pos} we can assume that $\lim_{m \rightarrow \infty} |\gs\rangle = |\st_2(s)\rangle$. Then 
\[
\langle \gs | E_{l,l+1} | \gs \rangle \rightarrow \langle \st_2(s) | E_{l,l+1}|\st_2(s)\rangle = 0.\qedhere
\]
\end{proof}

\subsection{Wilson loop for $m \ll 0$}
%\begin{proposition}
%We have
%\[
%\lim_{m \rightarrow -\infty} \frac{\lambda_1}{m} = 2\left[\frac{N}{2}\right]\sqrt{\frac{n}{2\pi}}.
%\]
%\end{proposition}
%\begin{proof}
%As in Proposition \ref{p:lambda_est_pos}, we always have an inequality by the very definition of $|\gs\rangle$, but now, since the coefficient in front of $D$ is negative, we refer to $\st_1$:
%\[
%\langle \st_1 | H|\st_1 \rangle \geq \langle \gs | H|\gs\rangle \geq \min_{\|\psi\| = 1} A\psi + 2m \sqrt{\frac{n}{2\pi}} \langle \st_1 |D|\st_1\rangle.
%\]
%Multiplying by $m^{-1}$ for $m$ negative flips the inequality:
%\[
%m^{-1}\langle \st_1 | H|\st_1 \rangle \leq m^{-1}\langle \gs | H|\gs\rangle \leq m^{-1}\min_{\|\psi\| = 1} A\psi + 2 \sqrt{\frac{n}{2\pi}} \langle \st_1 |D|\st_1\rangle.
%\]
%Obviously,
%\[
%\lim_{m\rightarrow -\infty} m^{-1}\langle \st_1 | H|\st_1\rangle = 2 \sqrt{\frac{n}{2\pi}} \langle \st_1 |D|\st_1\rangle;
%\]
%therefore, passing to the limit as $m \rightarrow -\infty$ sandwiches $m^{-1}\langle \gs | H|\gs\rangle$ and yields
%\[
%\lim_{m \rightarrow -\infty} m^{-1}\langle \gs |H|\gs \rangle = 2 \sqrt{\frac{n}{2\pi}} \langle \st_1 |D|\st_1\rangle = 2\left[\frac{N}{2}\right]\sqrt{\frac{n}{2\pi}}.
%\]
%In particular, 
%\[
%\lim_{m \rightarrow -\infty} \frac{\lambda_1}{m} = 2\left[\frac{N}{2}\right]\sqrt{\frac{n}{2\pi}}.
%\]
%\end{proof}
%\begin{lemma}\label{l:gs_dec1}
%If $\langle \st_1(s)|\gs \rangle > 0$ and $\lambda_1 \neq -\langle \st_1(s)|\gs\rangle^{-2} + 2[N/2]$, the ground state admits a decomposition
%\[
%|\gs \rangle = \alpha \st_1(s)\rangle + |v \rangle
%\]
%where $|v\rangle$ is orthogonal to $|\st_1(i)\rangle$ for all $i$.
%\end{lemma}
%\begin{proof}
%We use the same idea as in \ref{l:gs_dec} except that now the coefficients $c_j$ are taken with respect to $|\st_1(j)\rangle$:
%\[\begin{split}
%c_j& = \langle\st_1(j) | H | \st_1(j) \rangle =\langle\st_1(j) | D+E | \st_1(j) \rangle=\\&= (j+1-s)^2[N/2] + (j-s)^2[(N-1)/2] + [N/2].
%\end{split}
%\]
%The resulting objective function is the same. When $\lambda_1 \neq -\langle \st_1(s)|\gs\rangle^{-2} + 2[N/2]$, we obtain a unique solution for the components of $|\gs\rangle$ with respect to $|\st_1(i)\rangle$'s.
%\end{proof}
%
%\begin{proposition}
%Assume that the choice of the ground state for every negative mass is made in such a way that $\langle \gs | \st_1(s)\rangle > 0$. Then the limit exists and is given by $\lim_{m \rightarrow -\infty} |\gs\rangle = |\st_1(s)\rangle$.
%\end{proposition}
%\begin{corp}
%Let $\Sigma$ be the electric field operator. Then
%\[
%\lim_{m \rightarrow -\infty} \Sigma = \frac{1}{N}\left[\frac{N}{2}\right].
%\]
%\end{corp}
%\begin{proof}
%The proposition tells us that
%\[
%\lim_{m \rightarrow -\infty} \left\langle \gs |W|\gs \right\rangle = \langle \st_1(s) |N^{-1}\sum_l E_{l,l+1} | \st_1(s)\rangle = \frac{1}{N}\left[\frac{N}{2}\right].\qedhere
%\]
%\end{proof}

\subsection{Numerical records}
For $n = 9$, $N = 6$, the space given by $3$ fermions and $b_{-1} = 4$ is $C_6^3 = 20$ dimensional. For $m = 0$ checked that all $20$ vectors are in the decomposition of $|\gs\rangle$. For $m=1$, I can say there are $19$ vectors. For $m = 10$, there are $12$ vectors.
	
	\subsection{Idea: an equivalent model without links}
	Consider QED 1+1 with the Hamiltonian specified in Equation \ref{eq:1dQEDLatticeHamiltonian}, which reads 
\[
	H=-\frac{n}{2\pi} \sum_\ell (\psi_\ell^\dagger U_{\ell,\ell+1}\psi_{\ell+1}+H.c.) + 2m\sqrt{\frac{n}{2\pi}}\sum_\ell (-1)^\ell \psi_\ell^\dagger\psi_\ell+  \sum_{\ell} E_{\ell,\ell+1}^2
\]
Specifying the state of a boson on one chosen link assigns the states of bosons to all the other links. This suggests that we can consider a family of Hamiltonians parametrized by the state of the marked boson. View the chain as before:
	\[
	\xymatrix{
	\ar@{-}[r]^{-1} & \bigcirc^0 \ar@{-}[r]^{0} &  \bigcirc^1 \ar@{-}[r]^{1} & \cdots & \ar@{-}[l]_{N-2} \bigcirc^{N-1}
	}
	\]
	Let $q \in \mathbb Z_n$ be a state on the leftmost link. Recall that the physical space (determined by the Gauss law) is given by
	\[
	P = \Sp ( |n_0\, n_1\, \ldots \, n_{N-1} \, b_{-1}\, b_0\, \ldots\, b_{N-2}\rangle \ | \ b_{\ell} =b_{\ell-1} + n_{\ell} + \frac{1}{2}((-1)^\ell-1) \ \text{for every} \ \ell).
	\]
	We can split $P$ into the direct sum according to what value of $b_{-1}$ is:
	\[
	P = P_0 \oplus P_1 \oplus \cdots \oplus P_{n-1},
	\]
	where
	\[
	P_j = \Sp( |n_0\, n_1\, \ldots \, n_{N-1} \, b_{-1}\, b_0\, \ldots\, b_{N-2}\rangle \in P \ | \ b_{-1} = j).
	\]
	Now, let's observe what is the restriction of $H$ to each subspace $P_j$. Denote $H_j := H|_{P_j}$.
\begin{idea}
Does there exist a nice basis in each $P_j$ that allows us to write the action of $H_j$ as if there were no links?
\end{idea}

\subsubsection{Order of basis}
I show on an example what the order I choose. For $N=2$ and $n=3$, writing a basic ket as $|n_0\, n_1\, b_{-1}\, b_0\rangle$, we have
\[
|0\, 0\, 0\, 2\rangle \prec |0\, 0\, 1\, 0\rangle \prec |0\, 0\, 2\, 1\rangle \prec 
\prec |1\, 0\, 0\, 0\rangle \prec |1\, 0\, 1\, 1\rangle \prec |1\, 0\, 2\, 2\rangle \prec
\]\[
 |0\, 1\, 0\, 2\rangle \prec |0\, 1\, 1\, 0\rangle \prec |0\,1\, 2\, 1\rangle \prec
|1\, 1\, 0\, 0\rangle \prec |1\, 1\, 1\, 1\rangle \prec |1\, 1\, 2\, 2\rangle.
\]
So the rule is: first, split the basis into groups $F(n_0,\ldots,n_{N-1})$, each of which depends only on the states of the sites. We declare that
\[
F(n_0,\ldots,n_{N-1}) \prec F(\tilde{n}_0,\ldots,\tilde{n}_{N-1})
\]
if and only if, as a binary number,
\[
\underline{n_{N-1}\cdots n_{0}} \leq \underline{\tilde{n}_{N-1}\cdots \tilde{n}_{0}}.
\]
Inside of each group, we order the kets by $b_{-1}$ (the other $b_{\ell}$'s are completely determined by the states on the links and $b_{-1}$).

%\subsubsection{Some observations}
%\begin{statement}
%Let $A_{\ell} := \psi_\ell^\dagger U_{\ell,\ell+1} \psi_{\ell+1}$. Fix the states $\underline{n}_0,\ldots,\hat{n}_{\ell},\hat{n}_{\ell+1},\ldots,\underline{n}_{N-1},\underline{b}_{-1},\underline{b}_0,\ldots,\hat{b}_{\ell},\ldots,\underline{b}_{N-2}$ (the hat means that we skip that site or link and I use the underline to mean that the state is fixed).
%\[\begin{split}
%V:=V(\underline{n}_0,\ldots,\hat{n}_{\ell},\hat{n}_{\ell+1},\ldots,\underline{n}_{N-1},\underline{b}_{-1},\underline{b}_0,\ldots,\hat{b}_{\ell},\ldots,\underline{b}_{N-2}) :=\\= \Sp \{ |n_0\, n_1\, \ldots\, n_{N-1}\, b_{-1}\, b_0\, \ldots\, b_{N-2} \in P \ | \ n_i = \underline{n}_i, \ i \neq \ell, \ell+1; \ b_i = \underline{b}_{i}, \ i \neq \ell; \ n_{\ell} \neq n_{\ell+1}  \}.
%\end{split}
%\]
%Basically, I define a subspace in which everything is fixed except that $b_{\ell}$ can vary and $n_{\ell}$ with $n_{\ell+1}$ cannot be equal. Notice that the dimension of this subspace is equal to $2\cdot n$. We might say about the restriction of $A_{\ell}$ to $V$ the following:
%\begin{enumerate}[1)]
%\item As a matrix, the restriction of $A_{\ell}$ to $V$ might be represented as
%\[
%A_{\ell}|_V = \begin{pmatrix}
%0 & 0 \\
%I & 0
%\end{pmatrix}
%\]
%where $I$ is the identity $n \times n$ matrix.
%%where $S$ is $n \times n$ matrix given by
%%\[
%%S = \begin{pmatrix}
%%0 & \cdots & 0 &0 & 1\\
%%1 & & & &\\
%%& 1 & & &\\
%%& & \ddots && \\
%%& & & 1 & 0
%%\end{pmatrix}
%%\]
%\item We have
%\[
%(A_{\ell} + A_{\ell}^\dagger)^2|_V = AA^\dagger + A^\dagger A= I,
%\]
%where $I$ is the identity matrix. This is a simple computation.
%\end{enumerate}
%\end{statement}
%%\end{document}
