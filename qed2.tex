%\documentclass{article}
%\newcommand{\Mod}[1]{\ \left(\mathrm{mod}\ #1\right)}
%\usepackage{braket}

%\begin{document}
	
	\section{The meaning of Gauge}
	The Quantum Ising model has a \emph{global} symmetry $\tilde{S}_i = -S_i$, that leaves
	the form of the Hamiltonian invariant in the new variables, and so it's the same system.
	By \emph{global} I mean independent of $i$.
	The Quantum Ising Gauge model (see Hunter L.'s write up) has a \emph{local} symmetry.
	
	\section{Gauge and Matter in 1D}
	\subsection{Definitions of the full space}
	We consider again a 1D chain, put bosons of dimension $n$ on the links, with group $Z(n)$ and
	define $E$, $A$, $V$ and $U$ as before. We won't use $A$ and $V$ in the Hamiltonian.
	This time we add fermions on the sites, where at site $l$, the operator $\psi_\ell^\dagger$
	creates a fermion, and the operator $\psi_\ell$ destroys a fermion at that site.
	The fermion operators \emph{anti-commute} on \emph{different} sites.
	Then let
		\begin{equation}\label{eq:1dQEDLatticeHamiltonian}
	H=-\frac{n}{2\pi} \sum_\ell (\psi_\ell^\dagger U_{\ell,\ell+1}\psi_{\ell+1}+H.c.) + 2\tilde{m}\sum_\ell (-1)^\ell \psi_\ell^\dagger\psi_\ell+  \sum_{\ell} E_{\ell,\ell+1}^2,		
	\end{equation}
	where $n$ is an odd non-negative integer, and $m$ is a real number.
	We leave the boundary conditions open for now.
	The full space on $N$ sites has dimension $(2n)^{N}$, the $2^N$ comes from the fermions (electrons and positrons), 
	and the $n^N$ from the bosons (phonons). But this space isn't the physical space, and is too big, as we now discuss.
	
	A basis element at site $l$ is denoted as $|n_l, b_l\rangle$, where $n_l \in \{0,1\}$ indicates the presence of a fermion at site $l$ and $b_l \in \{0,\ldots,n-1\}$ indicates the state of the boson to the right of the fermion. Such basis is called \emph{computational}.
	
	Recall how the operators act at a single site $l$:
	\begin{enumerate}[(a)]
	\item $\psi^\dagger_l |0,k\rangle = |1,k\rangle$, $\psi^\dagger_l |1,k\rangle = 0$;
	\item $\psi_l|0,k\rangle = 0$, $\psi_l|1,k\rangle = |0,k\rangle$;
	\item $\psi^\dagger_l \psi_l |0, s\rangle = 0$ and $\psi^\dagger_l \psi_l |1,s\rangle = |1,s\rangle$;
	\item For $n = 2s+1$ odd, $E_{l,l+1}|c,k\rangle = (k-s)|c,k\rangle$. So, $E_{l,l+1}^2|c,k\rangle = (k-s)^2 |c,k\rangle$;
	\item $U_{l,l+1} |c,k\rangle = |c,k+1\rangle$;
	\item Consider two consecutive sites $l$ and $l+1$ and let $|n_l,b_l\rangle \otimes |n_{l+1},b_{l+1}\rangle$ be the corresponding state. Then 
	\[
	\psi_l^\dagger U_{l,l+1} \psi_{l+1} |n_l,b_l\rangle \otimes |n_{l+1},b_{l+1}\rangle = \begin{cases}
	0, \ n_{l+1} = 0 \ \text{or} \ n_{l} = 1;\\
	|1,b_l+1\rangle \otimes |0,b_{l+1}\rangle \ \text{otherwise};
	\end{cases}
	\]
	So, when we have a fermion at site $l+1$ and no fermion at site $l$, the operator $\psi_l^\dagger U_{l,l+1} \psi_{l+1}$ shifts the fermion to site $l$ and increments the state of the boson on the corresponding lift.
	\item Now let's observe how the Hamiltonian acts on states. For simplicity, consider three sites with periodic boundary conditions, i.e.
	\[
	\xymatrix{
	\bigcirc \ar@{-}[rr]& & \bigcirc \\
	& \bigcirc\ar@{-}[ru] \ar@{-}[lu] &
	}
	%\xymatrix{
	%\bigcirc \ar@{-}[r] &	\bigcirc \ar@{-}[r] &	\bigcirc 
	%\ar@{-}@/^1pc/[ll]
	%}
	\]
	Let $|n_0,n_1,n_2,b_0,b_1,b_2\rangle$ be a state. Let's see what parts of $H$ do to this state. Due to the rotational symmetry, there are actually only four distinct situations (all that matters is the number of fermions). So, 
	\[\begin{split}
&-\frac{n}{\pi}\sum_{l=0}^2 (\psi^\dagger_l U_{l,l+1} \psi_{l+1}) |n_0,n_1,n_2,b_0,b_1,b_2\rangle =\\ &= \begin{cases}
0, \ \ \text{no fermions or exactly three}\\
\text{Set } n_{l} = 0, \ n_{l-1} = 1, \ b_{l-1} \mapsto b_{l-1} + 1 \ \text{if there is a fermion at site }l \text{ and no fermion at site }l-1
\end{cases}
\end{split}
	\]
	\end{enumerate}
	For $2\tilde{m}\sum_{l} (-1)^l \psi_l^\dagger \psi_l$, we have
	\[
	(2\tilde{m}\sum_{l} (-1)^l \psi_l^\dagger \psi_l) |n_0,n_1,n_2,b_0,b_1,b_2\rangle  = 2 \tilde{m} \cdot F \cdot |n_0,n_1,n_2,b_0,b_1,b_2\rangle,
	\]
	where 
	\[
	F = \begin{cases}
	0, \ \ \text{there are two fermions and one of them is at site }1;\\
	(-1)^l, \ \ \text{there is only one fermion and located at } l;\\
	2, \ \ \text{the fermions are at sites }0\text{ and }2;\\
	
	\end{cases}
	\]
	Lastly,
	\[
	\sum_l E^2_{l,l+1} |n_0,n_1,n_2,b_0,b_1,b_2\rangle = 2|n_0,n_1,n_2,b_0,b_1,b_2\rangle.
	\]
	Now let's see what happens for some specific states. If there's only one fermion, then
	\[
	H|1,0,0; b_0,b_1,b_2\rangle = |0,0,1; b_0,b_1,b_2+1\rangle + (2\tilde{m}+2)|1,0,0; b_0,b_1,b_2\rangle 
	\]
	and the rest is rotationally symmetric.
	
	
	From the letter: in QED fermions (matter) occupy even sites, and anti-fermions (anti-matter) occupy odd sites. The same for QCD, except that fermions in QED are electrons, and fermions in QCD are quarks.
	
	\subsection{Gauss Law and Physical Space}
	Let the Gauss law be
		\begin{equation}\label{eq:1dQED_Gauss_law}
	G_\ell := \psi_\ell^\dagger \psi_\ell + \frac{1}{2}((-1)^\ell -1)-(E_{\ell,\ell+1}-E_{\ell-1,\ell}).
	\end{equation}
	We restrict the full space to the physical space of states $|\psi\rangle$ that satisfy
	$G_\ell|\psi\rangle= 0 \Mod{n}$. 
	The physical space has dimension $n 2^N$; see Hunter L's explanation for why this is the case.
	
	\subsection{Limits}
	Ercolessi et al (see the reference in Hunter's writeup) numerically take both limits \emph{first} the $N\rightarrow\infty$ limit
   and then the $n\rightarrow\infty$ limit and find a ``phase transition'' at some value of $\tilde{m}$. Order of the limits here FIXME TODO.
	For now, perhaps is enough to take $N\rightarrow\infty$ for $n$ finite, and prove that the limiting (topological, C*, Von Neuman??)
	algebra exists, and the limiting $H$ exits.
	To define a phase transition I need an ``order parameter'' which will be something like a magnetization in a spin model.
	Let $W=N^{-1}\sum_l E_{l, l+1}^2$ be the ``Wilson loop'' operator. Let $|gs\rangle$ be the ground state of $H$.
	Let $w_1(\tilde{m}, n, N) = \langle gs | W |gs\rangle$, and let $w(\tilde{m}, n)=\lim_{N\rightarrow\infty} w_1(\tilde{m}, n, N)$.
	 This quantity $w$ as a function of $\tilde{m}$ will have a first derivative everywhere,
	except for a value of $\tilde{m}$, where the left and right derivatives both exist but are different.
	We say that the model goes through a phase transition at that value of $\tilde{m}$. Obviously, that value will be dependent on $n$,
	and we are interested in $n\rightarrow\infty$.
	
%\end{document}
