%\documentclass{article}
%\newcommand{\Mod}[1]{\ \left(\mathrm{mod}\ #1\right)}
%\usepackage{braket}

%\begin{document}
	
	\section{The meaning of Gauge}
	The Quantum Ising model has a \emph{global} symmetry $\tilde{S}_i = -S_i$, that leaves
	the form of the Hamiltonian invariant in the new variables, and so it's the same system.
	By \emph{global} I mean independent of $i$.
	The Quantum Ising Gauge model (see Hunter L.'s write up) has a \emph{local} symmetry.
	
	\section{Gauge and Matter in 1D}
	\subsection{Definitions of the full space}
	We consider again a 1D chain, put bosons of dimension $n$ on the links, with group $Z(n)$ and
	define $E$, $A$, $V$ and $U$ as before. We won't use $A$ and $V$ in the Hamiltonian.
	This time we add fermions on the sites, where at site $l$, the operator $\psi_\ell^\dagger$
	creates a fermion, and the operator $\psi_\ell$ destroys a fermion at that site.
	The fermion operators \emph{anti-commute} on \emph{different} sites.
	Then let
		\begin{equation}\label{eq:1dQEDLatticeHamiltonian}
	H=-\frac{n}{2\pi} \sum_\ell (\psi_\ell^\dagger U_{\ell,\ell+1}\psi_{\ell+1}+H.c.) + 2m\sqrt{\frac{n}{2\pi}}\sum_\ell (-1)^\ell \psi_\ell^\dagger\psi_\ell+  \sum_{\ell} E_{\ell,\ell+1}^2,		
	\end{equation}
	where $n$ is an odd non-negative integer, and $m$ is a real number. We use the following version of the chain with a free end on the left that will determine the states of all the other links under the Gauss law:
	\begin{equation}\label{eq:lchain}
	\xymatrix{
	\ar@{-}[r]^{-1} & \bigcirc^0 \ar@{-}[r]^{0} &  \bigcirc^1 \ar@{-}[r]^{1} & \cdots & \ar@{-}[l]_{N-2} \bigcirc^{N-1}
	}
	\end{equation}
	
%\noindent \textbf{A discussion on the boundary conditions}. I attempted to use the condition
	%\[
	%\xymatrix{
	%\ar@{-}[r]^{-1} & \bigcirc^0 \ar@{-}[r]^{0} &  \bigcirc^1 \ar@{-}[r]^{1} & \cdots & \ar@{-}[l]_{N-2} \bigcirc^{N-1}
	%}
	%\]
	%where the numbers indicate the corresponding indices of links and sites. So, we specify the state on the leftmost link, which is $b_{-1}$, and this determines the states of all the other links in the physical space. However, there is nothing in the Hamiltonian that might change $b_{-1}$, it always stays the same, but all the other $b_i$'s might change (the first term of the Hamiltonian to make sense requires sites on each side of the link). This means that we will never land in the physical space. We might, however, be pleased with something like
		%\[
	%\xymatrix{
	%\square\ar@{-}[r]^{-1} & \bigcirc^0 \ar@{-}[r]^{0} &  \bigcirc^1 \ar@{-}[r]^{1} & \cdots & \ar@{-}[l]_{N-2} \bigcirc^{N-1}
	%}
	%\]
	%with the meaning of $\square$ to be an abyss that swallows any fermion that hops into it. For now, I will attempt the condition
			%\[
	%\xymatrix{
%\bigcirc^0 \ar@{-}[r]^{0} &  \bigcirc^1 \ar@{-}[r]^{1} & \cdots & \ar@{-}[l]_{N-2} \bigcirc^{N-1},
	%}
	%\]
	%so I don't have any boundary links.
	
	The full space on $N$ sites has dimension $(2n)^{N}$, the $2^N$ comes from the fermions (electrons and positrons), 
	and the $n^N$ from the bosons (phonons). The actual physical space we work with is specified by the kernel of the Gauss law (see below).
	
	A basis element at site $l$ is denoted as $|n_l, b_l\rangle$, where $n_l \in \{0,1\}$ indicates the presence of a fermion at site $l$ and $b_l \in \{0,\ldots,n-1\}$ indicates the state of the boson to the right of the fermion. Such basis is called \emph{computational}.
	
	Recall how the operators act. Let $|n_0 \, n_1\, \ldots \, n_{N-1}\, b_{-1}\, b_0 \, \ldots \, b_{N-2}\rangle$ be an arbitrary basic state. Then
	\begin{enumerate}[(a)]
	%\item $\psi^\dagger_\ell |0,k\rangle = |1,k\rangle$, $\psi^\dagger_l |1,k\rangle = 0$;
	\item \[\begin{split}
	&\psi^\dagger_\ell |n_0 \, n_1\, \ldots \, n_{N-1}\, b_{-1}\, b_0 \, \ldots \, b_{N-2}\rangle =  \\
&=(-1)^{n_0 + n_1 + \cdots + n_{\ell-1}}(1-n_\ell)|n_0 \, n_1\, \ldots\, n_\ell=1\, \ldots \, n_{N-1}\, b_{-1}\, b_0 \, \ldots \, b_{N-2}\rangle;
	\end{split}
	\]
	\item %$\psi_l|0,k\rangle = 0$, $\psi_l|1,k\rangle = |0,k\rangle$;
	\[\begin{split}
	&\psi_\ell |n_0 \, n_1\, \ldots \, n_{N-1}\, b_{-1}\, b_0 \, \ldots \, b_{N-2}\rangle =  \\
&=(-1)^{n_0 + n_1 + \cdots + n_{\ell-1}}n_\ell|n_0 \, n_1\, \ldots\, n_\ell=0\, \ldots \, n_{N-1}\, b_{-1}\, b_0 \, \ldots \, b_{N-2}\rangle;
	\end{split}
	\]
	\item %$\psi^\dagger_l \psi_l |0, s\rangle = 0$ and $\psi^\dagger_l \psi_l |1,s\rangle = |1,s\rangle$;
	\[
	\psi^\dagger_{\ell} \psi_{\ell} |n_0 \, n_1\, \ldots \, n_{N-1}\, b_{-1}\, b_0 \, \ldots \, b_{N-2}\rangle = n_\ell |n_0 \, n_1\, \ldots \, n_{N-1}\, b_{-1}\, b_0 \, \ldots \, b_{N-2}\rangle;
	\]
	\item For $n = 2s+1$ odd, %$E_{l,l+1}|c,k\rangle = (k-s)|c,k\rangle$. So, $E_{l,l+1}^2|c,k\rangle = (k-s)^2 |c,k\rangle$;
	\[
	E_{\ell,\ell+1} |n_0 \, n_1\, \ldots \, n_{N-1}\, b_{-1}\, b_0 \, \ldots \, b_{N-2}\rangle = (b_{\ell}-s) |n_0 \, n_1\, \ldots \, n_{N-1}\, b_{-1}\, b_0 \, \ldots \, b_{N-2}\rangle;
	\]
	\[
	E_{\ell,\ell+1}^2 |n_0 \, n_1\, \ldots \, n_{N-1}\, b_{-1}\, b_0 \, \ldots \, b_{N-2}\rangle = (b_\ell -s)^2 |n_0 \, n_1\, \ldots \, n_{N-1}\, b_{-1}\, b_0 \, \ldots \, b_{N-2}\rangle;
	\]
	\item \[\begin{split}
	U_{\ell,\ell+1} &|n_0 \, n_1\, \ldots \, n_{N-1}\, b_{-1}\, b_0 \, \ldots \, b_{N-2}\rangle =\\= &|n_0 \, n_1\, \ldots \, n_{N-1}\, b_{-1}\, b_0 \, \ldots \, b_{\ell}+1\, \ldots\, b_{N-2}\rangle
	\end{split}
	\]%$U_{l,l+1} |c,k\rangle = |c,k+1\rangle$;
	\item %Let $|n_0\, n_1\, \ldots \, n_{N-1} \, b_{-1}\, b_0\, \ldots\, b_{N-1}\rangle$ be a state. Then
	\begin{equation}\label{eq:ham_fir_t}\begin{split}
	\psi_\ell^\dagger U_{\ell,+1} \psi_{\ell+1} &|n_0\, n_1\, \ldots \, n_{N-1} \, b_{-1}\, b_0\, \ldots\, b_{N-2}\rangle=\\ = n_{\ell+1} (1-n_\ell) &|n_0\, n_1\, \ldots \, n_{\ell}=1 \, n_{\ell+1} = 0\,\ldots \, n_{N-1} \, b_{-1}\, b_0\, \ldots\,b_\ell + 1\,\cdots \, b_{N-2}\rangle.
	\end{split}
	\end{equation}
	So, when we have a fermion at site $l+1$ and no fermion at site $l$, the operator $\psi_l^\dagger U_{l,l+1} \psi_{l+1}$ shifts the fermion to site $l$ and increments the state of the boson on the corresponding lift. Similarly, its Hermitian conjugate acts as
		\[\begin{split}
	\psi_{\ell+1}^\dagger U_{\ell,\ell+1}^\dagger \psi_{\ell} &|n_0\, n_1\, \ldots \, n_{N-1} \, b_{-1}\, b_0\, \ldots\, b_{N-2}\rangle=\\ = n_{\ell} (1-n_{\ell+1}) &|n_0\, n_1\, \ldots \, n_{\ell} = 0\, n_{\ell+1} = 1\, \ldots\, n_{N-1} \, b_{-1}\, b_0\, \ldots\,b_\ell -1 \,\cdots \, b_{N-2}\rangle.
	\end{split}
	\]
	\end{enumerate}
	%\item Now let's observe how the Hamiltonian acts on states. 
	%\[
	%\xymatrix{
	%\bigcirc \ar@{-}[rr]& & \bigcirc \\
	%& \bigcirc\ar@{-}[ru] \ar@{-}[lu] &
	%}
	%%\xymatrix{
	%%\bigcirc \ar@{-}[r] &	\bigcirc \ar@{-}[r] &	\bigcirc 
	%%\ar@{-}@/^1pc/[ll]
	%%}
	%\]
	%Let $|n_0,n_1,n_2,b_0,b_1,b_2\rangle$ be a state. Let's see what parts of $H$ do to this state. Due to the rotational symmetry, there are actually only four distinct situations (all that matters is the number of fermions). So, 
	%\[\begin{split}
%&-\frac{n}{\pi}\sum_{l=0}^2 (\psi^\dagger_l U_{l,l+1} \psi_{l+1}) |n_0,n_1,n_2,b_0,b_1,b_2\rangle =\\ &= \begin{cases}
%0, \ \ \text{no fermions or exactly three}\\
%\text{Set } n_{l} = 0, \ n_{l-1} = 1, \ b_{l-1} \mapsto b_{l-1} + 1 \ \text{if there is a fermion at site }l \text{ and no fermion at site }l-1
%\end{cases}
%\end{split}
	%\]
	%\end{enumerate}
	%For $2\tilde{m}\sum_{l} (-1)^l \psi_l^\dagger \psi_l$, we have
	%\[
	%(2\tilde{m}\sum_{l} (-1)^l \psi_l^\dagger \psi_l) |n_0,n_1,n_2,b_0,b_1,b_2\rangle  = 2 \tilde{m} \cdot F \cdot |n_0,n_1,n_2,b_0,b_1,b_2\rangle,
	%\]
	%where 
	%\[
	%F = \begin{cases}
	%0, \ \ \text{there are two fermions and one of them is at site }1;\\
	%(-1)^l, \ \ \text{there is only one fermion and located at } l;\\
	%2, \ \ \text{the fermions are at sites }0\text{ and }2;\\
	%
	%\end{cases}
	%\]
	%Lastly,
	%\[
	%\sum_l E^2_{l,l+1} |n_0,n_1,n_2,b_0,b_1,b_2\rangle = 2|n_0,n_1,n_2,b_0,b_1,b_2\rangle.
	%\]
	%Now let's see what happens for some specific states. If there's only one fermion, then
	%\[
	%H|1,0,0; b_0,b_1,b_2\rangle = |0,0,1; b_0,b_1,b_2+1\rangle + (2\tilde{m}+2)|1,0,0; b_0,b_1,b_2\rangle 
	%\]
	%and the rest is rotationally symmetric.
	%
	
	From the letter: in QED fermions (matter) occupy even sites, and anti-fermions (anti-matter) occupy odd sites. The same for QCD, except that fermions in QED are electrons, and fermions in QCD are quarks.
	
	\subsection{Gauss Law and Physical Space}
	Let the Gauss law be
		\begin{equation}\label{eq:1dQED_Gauss_law}
	G_\ell := \psi_\ell^\dagger \psi_\ell + \frac{1}{2}((-1)^\ell -1)-(E_{\ell,\ell+1}-E_{\ell-1,\ell}).
	\end{equation}
	We restrict the full space to the physical space of states $|\psi\rangle$ that satisfy
	$G_\ell|\psi\rangle= 0 \Mod{n}$. 
	The physical space has dimension $n 2^N$; see Hunter L's explanation for why this is the case.
	
	I wanted to expand a bit on the Gauss law. So, all pieces that comprise $G_\ell$ are diagonal, hence $G_\ell$ is diagonal itself and its kernel is spanned by some of the basic kets. If $|n_0\, n_1\, \ldots \, n_{N-1} \, b_{-1}\, b_0\, \ldots\, b_{N-2}\rangle$ is such, then the action of $G_\ell$ upon it is given by
	\[\begin{split}
	G_\ell|n_0\, n_1\, \ldots \, n_{N-1} \, b_{-1}\, b_0\, \ldots\, b_{N-2}\rangle =\\= (n_{\ell} + \frac{1}{2}((-1)^\ell-1) + b_{\ell-1}-b_\ell) |n_0\, n_1\, \ldots \, n_{N-1} \, b_{-1}\, b_0\, \ldots\, b_{N-2}\rangle,
	\end{split}
	\]
We see that $b_\ell$ is determined by $b_{\ell-1}$ via
	\[
	b_\ell = b_{\ell-1} + n_{\ell} + \frac{1}{2}((-1)^\ell-1).
	\]
	So the physical space is 
	\[
	\Sp ( |n_0\, n_1\, \ldots \, n_{N-1} \, b_{-1}\, b_0\, \ldots\, b_{N-2}\rangle \ | \ b_{\ell} =b_{\ell-1} + n_{\ell} + \frac{1}{2}((-1)^\ell-1) \ \text{for every} \ \ell).
	\]
	Recursively solving the equation for $b_{\ell}$, we find that
	\[
	b_{\ell} = b_{-1} + n_0 + n_1 + \cdots + n_{\ell} - \left[\frac{\ell+1}{2}\right],
	\]
	where $[ \, ]$ denotes the integer part.
	
	\begin{proposition}
Each term in the Hamiltonian defined as in \eqref{eq:1dQEDLatticeHamiltonian} on the chain \eqref{eq:lchain} with a free left link leaves the physical space invariant.
	\end{proposition}
	\begin{proof}
	The first term that shifts a fermion to the left acts according to the formula \ref{eq:ham_fir_t}. If $n_\ell = 0$ and $n_{\ell+1} = 1$, then the term acts non-trivially and the state on $\ell$th link is updated according to
	\[
	b_\ell + 1 = b_{-1} + n_0 + n_1 + \cdots + (n_\ell+1) -\left[\frac{\ell+1}{2}\right],
	\]
	so we simply add $1$ to both sides. For $k \geq 1$, the equation of states on the other links are also preserved:
	\[
	b_{\ell+k} = b_{-1} + n_0 + \cdots + (n_{\ell}-1) + (n_{\ell+1}+1) + \cdots + n_{\ell +k} -\left[\frac{\ell+1}{2}\right].
	\]
	Similarly for Hermitian conjugate term (now subtract $1$ from both sides in the equation for $b_{\ell}$).
	The other terms in the Hamiltonian are diagonal, so they obviously preserve the physical space. Thus the Hamiltonian itself does preserves the physical space.
	\end{proof}
	
	\subsection{Limits}
	Ercolessi et al \cite{ercolessi} numerically take both limits \emph{first} the $N\rightarrow\infty$ limit
   and then the $n\rightarrow\infty$ limit and find a ``phase transition'' at some value of $\tilde{m}$. Order of the limits here FIXME TODO.
	For now, perhaps is enough to take $N\rightarrow\infty$ for $n$ finite, and prove that the limiting (topological, C*, Von Neuman??)
	algebra exists, and the limiting $H$ exits.
	To define a phase transition I need an ``order parameter'' which will be something like a magnetization in a spin model.
	Let $W=N^{-1}\sum_l E_{l, l+1}$ be the ``Wilson loop'' operator. Let $|gs\rangle$ be the ground state of $H$.
	Let $w_1(\tilde{m}, n, N) = \langle gs | W |gs\rangle$, and let $w(\tilde{m}, n)=\lim_{N\rightarrow\infty} w_1(\tilde{m}, n, N)$.
	 This quantity $w$ as a function of $\tilde{m}$ will have a first derivative everywhere,
	except for a value of $\tilde{m}$, where the left and right derivatives both exist but are different.
	We say that the model goes through a phase transition at that value of $\tilde{m}$. Obviously, that value will be dependent on $n$,
	and we are interested in $n\rightarrow\infty$.
	
	\subsubsection{Staggered configurations}
 We define two \emph{staggered} states: $|\st_1(b_{-1})\rangle$ and $|\st_2(b_{-1})\rangle$ in the following way: for the first one, all even sites are occupied and odd are empty; for the second one, all even sites are empty and all odd sites are occupied. The states on the links are determined by $b_{-1}$, which we include in the definition. 
\begin{statement}
For $|\st_2(b_{-1})\rangle$, $
b_{\ell} = b_{-1}$; for $|\st_1(b_{-1})\rangle$, $
b_{\ell} = \begin{cases}
b_{-1} + 1, \ \ell = 2k;\\
b_{-1}, \ \ell = 2k+1.
\end{cases}$. Pictorially,
\[
|\st_2(b_{-1})\rangle =	\xymatrix{
	\ar@{-}[r]^{b_{-1}} & \bigcirc \ar@{-}[r]^{b_{-1}} &  \bigotimes \ar@{-}[r]^{b_{-1}} & \cdots & \ar@{-}[l]_{b_{-1}} ?
	}
\]
\[
|\st_1(b_{-1})\rangle =	\xymatrix{
	\ar@{-}[r]^{b_{-1}} & \bigotimes \ar@{-}[r]^{b_{-1}+1} &  \bigcirc \ar@{-}[r]^{b_{-1}} & \cdots & \ar@{-}[l]_{b_{-1}+?} ?
	}
\]
where $\otimes$ means that the site is occupied. This is a simple computation using aforementioned formulas derived from the Gauss law.
\end{statement}

\subsubsection{Wilson loop for $m \gg 0$}
Split the Hamiltonian as $H = A + 2m\sqrt{n/2\pi} D$, where $A$ is the sum of the first term and the third terms of $H$ and $D = \sum_l (-1)^l \psi_l^\dagger \psi_l$.
\begin{proposition}
No matter what $b_{-1}$ is, we have
\[
\lim_{m \rightarrow \infty}m^{-1} \langle \gs | H| \gs\rangle = \lim_{m \rightarrow \infty} m^{-1}\langle \st_2 | H| \st_2 \rangle =  2\sqrt{\frac{n}{2\pi}} \langle \st_2 | D| \st_2\rangle = -2\left[\frac{N}{2}\right]\sqrt{\frac{n}{2\pi}}.
\]
As a consequence, if $\lambda_1$ is the eigen-value corresponding to $|\gs\rangle$, then $\lambda_1 = O(m)$. More precisely, 
\[
\lim_{m \rightarrow \infty} \frac{\lambda_1}{m} = -2\left[\frac{N}{2}\right]\sqrt{\frac{n}{2\pi}}.
\]
This implies that for $m \gg 0$ and $N$ even (or large), the lowest energy level can be approximated as 
\[
\lambda_1(m) \approx - Nm \sqrt{\frac{n}{2\pi}}.
\]
\end{proposition}
\begin{proof}
Notice that the eigen-vectors of $H$ and $m^{-1}H$ are the same (and their natural order is preserved). From the definition of $\gs_2$, we see that
\[
m^{-1}\langle \st_2 | H| \st_2 \rangle \geq m^{-1}\langle \gs | H| \gs \rangle = \min_{\|\psi\|=1} \langle \psi |H|\psi\rangle \geq m^{-1}\min_{\|\psi\|=1} A\psi + 2\sqrt{\frac{n}{2\pi}} \langle \st_2 | D| \st_2\rangle 
\]
Note that 
\[
\lim_{m\rightarrow \infty} m^{-1}\langle \st_2 | H|\st_2 \rangle = 2\sqrt{\frac{n}{2\pi}} \langle \st_2 | D| \st_2\rangle
\]
Hence, taking the limit in the above inequality, we obtain the statement. With regards to $\lambda_1$, we write $\langle \gs | H| \gs \rangle = \lambda_1\langle \gs |\gs\rangle = \lambda_1$.
\end{proof}
%\begin{quest}
%I want to prove rigorously that $\lim_{m \rightarrow \infty} |\gs \rangle = |\st_2 \rangle$. This statement is corroborated with numerical simulations.
%\end{quest}

%\noindent\textbf{Further observations}
%\begin{enumerate}[1] 
%\item I conducted a few numerical tests for $n = 5$ and $N=6$. I found that if $\lambda_2(m)$ is the sixth eigen-value of $H$ (I skip the first five, for they converge to $\lambda_1(m)$ and correspond to energies of bosons), then, at least numerically, 
%\[
%\lambda_2(m) \approx -2\left[ \frac{N}{2}-1\right] \sqrt{\frac{n}{2\pi}}.
%\]
%However, an obvious tweak of this formula (subtract 2 instead of 1) didn't work for $\lambda_3(m)$ (it seems in that case should be an extra factor of $2$; by $\lambda_3(m)$ here I mean the eleventh eigen-value).
%\item I found eigen-values symbolically in Matlab for $n = 3$, $N = 2$. They are all either constants or of order $O(m)$.
%\end{enumerate}

%\begin{wrong}
%I needed to prove the following: let $A \in \Mat(n\times n,\mathbb C)$ be a matrix (one might assume $A = A^\dagger$ if needed), $D \in \Mat(n\times n,\mathbb C)$ be diagonal and let $A(m) := A + mD$, $m \in \mathbb C$. Let $\lambda_1(m),\ldots,\lambda_n(m)$ be the list of eigen-values of $A(m)$. Then $\lambda_i(m) = O(m)$ or $\lambda_i(m) \equiv \text{const}$. This is not at all true. I checked numerically in Matlab that $\lambda_i$ may not be even a polynomial and may contain any power of $m$, even fractional.
%\end{wrong}
%\begin{statement}
%Assume that we proved $\lim_{m \rightarrow \infty} m^{-1} |\gs\rangle = |\st_2\rangle$. As a consequence, we obtain $\lim_{m \rightarrow \infty}\omega_1(m,n,N) = 
%\end{statement}
\begin{lemma}\label{l:gs_dec}
Assume that $\lambda_1 \neq -\langle \st_2(s) | \gs\rangle^{-2}$. Then any ground state admits a decomposition
\[
|gs\rangle = \alpha |\st_2(s) \rangle + | v\rangle
\]
where $|v\rangle$ is orthogonal to $|\st_2(b_{-1})\rangle$ for any $b_{-1} \in \mathbb Z_n$ and $s$ is such that $n = 2s + 1$. In other words, the only staggered configuration (of second type) that might be present in $|\gs\rangle$ corresponds to $b_{-1} = s$. 
%If $|\gs \rangle$ is a ground state and $|\gs \rangle = \sum \langle \psi | \gs \rangle |\psi\rangle$ in terms of the basic kets $|\psi\rangle$, then $\langle \psi | \gs \rangle = 0$ for any $|\psi\rangle$ that has $b_{-1} \neq s$ (where $s$ is from $n = 2s+1$). In other words, the ground state corresponds to the situation when $b_{-1} = s$. 
\end{lemma}
There is no need to think about alleviating the mild condition $\lambda_1 \neq -\langle \st_2(s) | \gs\rangle^{-2}$, for this is vacuous when the mass is large, and the lemma itself is needed only to prove that $\lim_{m \rightarrow \infty}|\gs\rangle = |\st_2(s)\rangle$.
\begin{proof}
Only for this proof, write $H$ as $H = U + D + E$, where $U$, $D$ and $E$ correspond to respectively the first, the second and the third terms of the Hamiltonian (so, $D$ is as before except that I absorbed the constant into it). Pick a ground state and decompose it as 
\[
|\gs \rangle = \sum_{i \in \mathbb Z_n} \alpha_i |\st_2(i)\rangle + |v\rangle,
\]
where $|v\rangle \perp |\st_2(i)\rangle$ for all $i \in \mathbb Z_n$. Note that $H$ is not only Hermitian but also symmetric, so we can assume $\alpha_i$'s are real\footnote{In more detail, the ground state can be decomposed into its real and imaginary parts: $|\gs\rangle = \Re|\gs\rangle + i \Im |\gs\rangle$. Then, with an appropriate scaling, both parts are also ground states, but now with real components. Proving the result for the real parts, we prove the result for $|\gs\rangle$ as well (but then $\alpha$ might be complex).}. Define additionally a state $|\phi \rangle$ such that
\[
|\phi\rangle = \sum_{i \in \mathbb Z_n} \beta_i | \st_2(i)\rangle + |v\rangle, \ \beta_i \in \mathbb R, \ \sum_{i\in \mathbb Z_n}\beta_i^2 = \sum_{i\in \mathbb Z_n} \alpha_i^2 =: \alpha^2.
\]
So, $\langle \phi | \phi \rangle = 1$. The idea is to find a nice objective function to formulate a minimization problem for $\beta_i$'s on the $(n-1)$-dimensional sphere of radius $|\alpha|$, whose solution has to be the ground state $|\gs\rangle$. Notice that $U$ sends $|\st_2(j)\rangle$ to a vector that is ortogonal to all $|\st_2(i)\rangle$. This implies that
\begin{equation}\label{eq:sthst}
\langle \st_2(j) | H| \st_2(i) \rangle = \langle \st_2(j)|D+E|\st_2(i)\rangle = \delta_{ij}(-[N/2] + (j-s)^2(N-1)) =:\delta_{ij}c_j
\end{equation}
where $\delta_{ij}$ is the Kronecker symbol.  Next, since $|\gs\rangle$ is an eigen-vector,
\[
H|\gs\rangle = \lambda_1 |\gs\rangle = \sum_{i \in \mathbb Z_n} \alpha_i H|\st_2(i)\rangle + H|v\rangle
\]
hence
%\begin{equation}\label{eq:hv}
\[
H|v\rangle = \sum_{i \in \mathbb Z_n}\left(\lambda_1\alpha_i |\st_2(i)\rangle - \alpha_iH|\st_2(i)\rangle\right) +\lambda_1|v\rangle;
\]
Combining this with \eqref{eq:sthst}, we see that
\[
\langle \st_2(j) | H|v \rangle = \alpha_j(\lambda_1 - (j-s)^2(N-1) + [N/2]) = \alpha_j(\lambda_1-c_j).
\]
Consider $\langle \st_2(j) |H|\phi\rangle$.% It's given by\footnote{Note that $U$ sends $|\st_2(j)\rangle$ to something that is orthogonal to all $|\st_2(i)\rangle$.}
\[\begin{split}
\langle \st_2(j) |H|\phi\rangle& = \sum_{i \in \mathbb Z_n}\beta_i\langle \st_2(j) |H|\st_2(i)\rangle + \langle \st_2(j) | H | v \rangle = \\& = \beta_j \langle \st_2(j) | D|\st_2(j)\rangle + \beta_j \langle \st_2(j)|E|\st_2(j)\rangle + \langle \st_2(j) |H|v \rangle = \\ &= \beta_j(-[N/2] + (j-s)^2(N-1)) +  \alpha_j(\lambda_1 - (j-s)^2(N-1) + [N/2]) = \\ &= \beta_jc_j + \alpha_j(\lambda_1-c_j), %= \\ &= (\beta_j - \alpha_j)(-[N/2] + (j-s)^2(N-1)) + \alpha_j \lambda_1.
\end{split}
\]
For $\langle v | H |\phi \rangle$ we have
\[\begin{split}
\langle v | H | \phi \rangle& = \sum_{i \in \mathbb Z_n}\beta_i\langle v | H|\st_2(i)\rangle + \langle v |H|v \rangle = \\ &= \beta_j\alpha_j(\lambda_1 - c_j) + \langle v |H|v \rangle.
\end{split}
\]
Now we can evaluate $\langle \phi | H| \phi \rangle$:
\[
\langle \phi | H| \phi \rangle = \sum_{j \in \mathbb Z_n} \left(\beta_j^2 c_j + 2\beta_j\alpha_j(\lambda_1-c_j) \right) + n\langle v |H|v \rangle 
\]
We are ready to define an objective function $f: S^{n-1} \rightarrow \mathbb R$,
\[
f(\vec{\beta}) := f(\beta_0,\ldots,\beta_{n-1}) := \sum_{j \in \mathbb Z_n} (\beta_j^2 c_j + 2\beta_j\alpha_j(\lambda_1-c_j))
, \ \ \sum_{j\in \mathbb Z_n} \beta_j^2 = \alpha^2.
\]
Let's look at the chart given by $\beta_s := \sqrt{\alpha^2 - \sum_{j \neq s} \beta_j^2}$. Notice that $c_s = (s-s)^2 = 0$. In this chart, the coordinate representation of $f$ is 
\[
f(\vec{\beta}) = \sum_{j\neq s} (\beta_j^2 c_j + 2\beta_j \alpha_j(\lambda_1 - c_j)) + 2\frac{\alpha_s}{\sqrt{\alpha^2 - \sum_{j \neq s} \beta_j^2}}
\]
The partial derivative with respect to $\beta_j$ for $j \neq s$ is
\[
\frac{\partial f}{\partial \beta_j} = 2\beta_j c_j + 2\alpha_j(\lambda_1-c_j) + 2\frac{\beta_j\alpha_s}{\beta_s^3};
\]
Now the trick. By the very definition of $|\gs\rangle$, the collection $\beta_j := \alpha_j$ has to minimize $f$. Therefore, the partial derivatives of $f$ with respect to $\beta_j$'s (for $j \neq s$) in this chart at $\beta_j = \alpha_j$ are zero:
\[
2\alpha_j c_j + 2\alpha_j(\lambda_1-c_j) + 2\frac{\alpha_j\alpha_s}{\alpha_s^3} = 0.
\]
One of the solutions of the above equation is $\alpha_j = 0$. If $\lambda_1 = - \alpha_s^{-2}$, there might be something else, but we assumed $\lambda_1 \neq - \alpha_s^{-2}$ from the beginning. So, since $\alpha_j = 0$ ($j \neq s$) is the unique solution, it's the one that corresponds to the decomposition of $|\gs \rangle$. Thus $|\gs\rangle = \alpha_s |\st_2(s)\rangle + |v\rangle$.
\end{proof}

\noindent \textbf{Caution}. Without saying a few words beforehand, we can't talk about $\lim_{m\rightarrow \infty} |\gs\rangle = |\st_2(s)\rangle$, for this limit depends on the way we choose $|\gs\rangle$ for every value of mass. We can be whimsical and choose $|\gs \rangle = \alpha |\st_2\rangle + |v\rangle$ with $\arg \alpha$ (the argument of the complex number) running around $[0,2\pi)$. Then the limiting vector is not unique. A limiting value might be $|\st_2(s)\rangle$, but it also might be something like $e^{i\pi/3} |\st_2(s)\rangle$. One natural way to avoid this is to assume $\langle \gs | \st_2(s) \rangle > 0$, i.e. we normalize the argument of the component of $\gs$ corresponding to the configuration $|\st_2(s)\rangle$.


\begin{proposition}
Assume that the choice of the ground state for every mass is made in such a way that $\langle \gs|\st_2(s)\rangle > 0$. Then
the limit exists and is given by $\lim_{m\rightarrow \infty} |\gs\rangle = |\st_2(s)\rangle$.
\end{proposition}
%What's left to show is that we don't need to pass to a subsequence of masses and that $\lim_{m \rightarrow \infty} \langle \st_2 | \gs\rangle = 1$.
\begin{proof}
First, let's show that for any basic ket $|\psi\rangle$ that is not equal to $|\st_2(b_{-1})\rangle$ (for any $b_{-1}$), $\lim_{m \rightarrow \infty} \langle \psi | \gs\rangle = 0$. Indeed, we see that
\[
\langle \psi | \gs \rangle = \frac{1}{\lambda_1} \langle \psi | H|\gs\rangle = \frac{1}{\lambda_1} \langle \psi |A|\gs\rangle + \frac{2m}{\lambda_1} \sqrt{\frac{n}{2\pi}} \langle \psi | D | \gs \rangle = \frac{1}{\lambda_1} \langle \psi |A|\gs\rangle + \frac{2m}{\lambda_1} \sqrt{\frac{n}{2\pi}} \cdot a \langle \psi | \gs \rangle
\]
where $a$ is an eigen-value of $D$ (which is not equal to $-[N/2]$ since $\psi \neq |\st_2(b_{-1})\rangle$). The above can be rewritten as
\[
\langle \psi | \gs\rangle = \frac{1}{\lambda_1} \frac{1}{1- \frac{2ma}{\lambda_1}\sqrt{\frac{n}{2\pi}} } \langle \psi | A | gs\rangle.
\]
We see that 
\[
1- \frac{2ma}{\lambda_1}\sqrt{\frac{n}{2\pi}} \rightarrow 1 + a [N/2] \neq 0
\]
therefore
\[
\lim_{m \rightarrow \infty}\langle \psi | \gs \rangle = 0.
\]
By Lemma \ref{l:gs_dec}, $|\gs\rangle$ admits a decomposition $|\gs\rangle = \alpha|\st_2(s)\rangle + |v\rangle$ ($\alpha$ might be complex in general). Clearly, for $\psi = |\st_2(b_{-1})\rangle$ with $b_{-1}$, the value $\langle \psi | \gs_2\rangle$ tends to zero as well. Therefore,
\[
|\alpha|^2 = |\langle \st_2(s) | \gs \rangle|^2 \rightarrow |\langle \gs | \gs \rangle|^2 =1.
\]
Now the assumptiong $\alpha > 0$ comes into play. The only possible value is then $\alpha = 1$. Thus the limit exists and is equal to $|\st_2(s)\rangle$.
%Now, 
%\[
%1 = \langle \gs|\gs\rangle = \sum_{b_{-1} \in \mathbb Z_n}|\langle \st_2(b_{-1}) | \gs \rangle|^2 + \sum_{\psi \ \text{basic}, \ \psi \neq |\st_2\rangle} |\langle \psi | \gs\rangle|^2.
%\]
%Since the second sum goes to zero, as shown above, we see that 
%\[
%\lim_{m \rightarrow \infty} (1 - \sum_{b_{-1} \in \mathbb Z_n}
%|\langle \st_2(b_{-1}) | \gs \rangle|^2) = 0;
%\]
%....
%to which staggered configuration does $\gs$ converge?
%but this implies $\langle \st_2 | \gs \rangle \rightarrow 1$. Therefore, all coefficients of $|\gs\rangle$ converge to some numbers, which is enough to conclude (by observing those numbers) that $\lim_{m \rightarrow \infty} |\gs \rangle = |\st_2\rangle$.
\end{proof}
\begin{corp}
Let $\Sigma = N^{-1}\sum \langle E_{l,l+1} \rangle$ be the electric field operator. Then\footnote{No matter how we choose the ground states for every value of mass since we're under the Hermitian inner product.} $\lim_{m \rightarrow \infty} \Sigma = 0$.
\end{corp}
\begin{proof}
Indeed, we can assume that $\lim_{m \rightarrow \infty} |\gs\rangle = |\st_2(s)\rangle$. Then $\langle \gs | E_{l,l+1} | \gs \rangle \rightarrow \langle \st_2(s) | E_{l,l+1}|\st_2(s)\rangle = 0$.
\end{proof}
\begin{corp}
Let $\Sigma$ be again an electric field operator. If the mass $m$ is large, then $\Sigma$ is bounded for \emph{any} $N$ and $n$. In particular, for large $m$ there's a sequence $N_k \rightarrow \infty$ and $n_k \rightarrow \infty$ such that $\Sigma$ converges to a number.
\end{corp}
\begin{proof}
This follows right from the definition of the limit. For $\varepsilon = 1$, there is $m_0$ such that for all $m \geq m_0$ we have $|\Sigma| < 1$. So, $\Sigma$ is bounded by $1$ for all $N$, $n$ and $m \geq m_0$. A compactness argument then gives us subsequences $N_k \rightarrow \infty$ and $n_k \rightarrow \infty$ such that $\lim_{k \rightarrow \infty} \Sigma$ is a number.
\end{proof}
We do not specify the order of the limits in the previous proposition. Whatever order we choose (or whatever fancy subsequences we choose, may be $N$ depending on $n$) -- there is still a subsequence for which $\Sigma$ does converge. 
%Let me use Erolessi's notation $\Sigma$ for the expectation value of the Wilson loop. I will denote $\Sigma_{\infty} := \lim_{m\rightarrow \infty} \Sigma$. These depend on $n$ and $N$, but let me not drag those numbers along our way. Clearly, the limit exists, for we proved that the ground state converges to the staggered configuration $|\st_2\rangle$.
	
	\subsection{Idea: an equivalent model without links}
	Consider QED 1+1 with the Hamiltonian specified in Equation \ref{eq:1dQEDLatticeHamiltonian}, which reads 
\[
	H=-\frac{n}{2\pi} \sum_\ell (\psi_\ell^\dagger U_{\ell,\ell+1}\psi_{\ell+1}+H.c.) + 2m\sqrt{\frac{n}{2\pi}}\sum_\ell (-1)^\ell \psi_\ell^\dagger\psi_\ell+  \sum_{\ell} E_{\ell,\ell+1}^2
\]
Specifying the state of a boson on one chosen link assigns the states of bosons to all the other links. This suggests that we can consider a family of Hamiltonians parametrized by the state of the marked boson. View the chain as before:
	\[
	\xymatrix{
	\ar@{-}[r]^{-1} & \bigcirc^0 \ar@{-}[r]^{0} &  \bigcirc^1 \ar@{-}[r]^{1} & \cdots & \ar@{-}[l]_{N-2} \bigcirc^{N-1}
	}
	\]
	Let $q \in \mathbb Z_n$ be a state on the leftmost link. Recall that the physical space (determined by the Gauss law) is given by
	\[
	P = \Sp ( |n_0\, n_1\, \ldots \, n_{N-1} \, b_{-1}\, b_0\, \ldots\, b_{N-2}\rangle \ | \ b_{\ell} =b_{\ell-1} + n_{\ell} + \frac{1}{2}((-1)^\ell-1) \ \text{for every} \ \ell).
	\]
	We can split $P$ into the direct sum according to what value of $b_{-1}$ is:
	\[
	P = P_0 \oplus P_1 \oplus \cdots \oplus P_{n-1},
	\]
	where
	\[
	P_j = \Sp( |n_0\, n_1\, \ldots \, n_{N-1} \, b_{-1}\, b_0\, \ldots\, b_{N-2}\rangle \in P \ | \ b_{-1} = j).
	\]
	Now, let's observe what is the restriction of $H$ to each subspace $P_j$. Denote $H_j := H|_{P_j}$.
\begin{idea}
Does there exist a nice basis in each $P_j$ that allows us to write the action of $H_j$ as if there were no links?
\end{idea}

\subsubsection{Order of basis}
I show on an example what the order I choose. For $N=2$ and $n=3$, writing a basic ket as $|n_0\, n_1\, b_{-1}\, b_0\rangle$, we have
\[
|0\, 0\, 0\, 2\rangle \prec |0\, 0\, 1\, 0\rangle \prec |0\, 0\, 2\, 1\rangle \prec 
\prec |1\, 0\, 0\, 0\rangle \prec |1\, 0\, 1\, 1\rangle \prec |1\, 0\, 2\, 2\rangle \prec
\]\[
 |0\, 1\, 0\, 2\rangle \prec |0\, 1\, 1\, 0\rangle \prec |0\,1\, 2\, 1\rangle \prec
|1\, 1\, 0\, 0\rangle \prec |1\, 1\, 1\, 1\rangle \prec |1\, 1\, 2\, 2\rangle.
\]
So the rule is: first, split the basis into groups $F(n_0,\ldots,n_{N-1})$, each of which depends only on the states of the sites. We declare that
\[
F(n_0,\ldots,n_{N-1}) \prec F(\tilde{n}_0,\ldots,\tilde{n}_{N-1})
\]
if and only if, as a binary number,
\[
\underline{n_{N-1}\cdots n_{0}} \leq \underline{\tilde{n}_{N-1}\cdots \tilde{n}_{0}}.
\]
Inside of each group, we order the kets by $b_{-1}$ (the other $b_{\ell}$'s are completely determined by the states on the links and $b_{-1}$).

\subsubsection{Some observations}
\begin{statement}
Let $A_{\ell} := \psi_\ell^\dagger U_{\ell,\ell+1} \psi_{\ell+1}$. Fix the states $\underline{n}_0,\ldots,\hat{n}_{\ell},\hat{n}_{\ell+1},\ldots,\underline{n}_{N-1},\underline{b}_{-1},\underline{b}_0,\ldots,\hat{b}_{\ell},\ldots,\underline{b}_{N-2}$ (the hat means that we skip that site or link and I use the underline to mean that the state is fixed).
\[\begin{split}
V:=V(\underline{n}_0,\ldots,\hat{n}_{\ell},\hat{n}_{\ell+1},\ldots,\underline{n}_{N-1},\underline{b}_{-1},\underline{b}_0,\ldots,\hat{b}_{\ell},\ldots,\underline{b}_{N-2}) :=\\= \Sp \{ |n_0\, n_1\, \ldots\, n_{N-1}\, b_{-1}\, b_0\, \ldots\, b_{N-2} \in P \ | \ n_i = \underline{n}_i, \ i \neq \ell, \ell+1; \ b_i = \underline{b}_{i}, \ i \neq \ell; \ n_{\ell} \neq n_{\ell+1}  \}.
\end{split}
\]
Basically, I define a subspace in which everything is fixed except that $b_{\ell}$ can vary and $n_{\ell}$ with $n_{\ell+1}$ cannot be equal. Notice that the dimension of this subspace is equal to $2\cdot n$. We might say about the restriction of $A_{\ell}$ to $V$ the following:
\begin{enumerate}[1)]
\item As a matrix, the restriction of $A_{\ell}$ to $V$ might be represented as
\[
A_{\ell}|_V = \begin{pmatrix}
0 & 0 \\
I & 0
\end{pmatrix}
\]
where $I$ is the identity $n \times n$ matrix.
%where $S$ is $n \times n$ matrix given by
%\[
%S = \begin{pmatrix}
%0 & \cdots & 0 &0 & 1\\
%1 & & & &\\
%& 1 & & &\\
%& & \ddots && \\
%& & & 1 & 0
%\end{pmatrix}
%\]
\item We have
\[
(A_{\ell} + A_{\ell}^\dagger)^2|_V = AA^\dagger + A^\dagger A= I,
\]
where $I$ is the identity matrix. This is a simple computation.
\end{enumerate}
\end{statement}
%\end{document}
