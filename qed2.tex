%\documentclass{article}
%\newcommand{\Mod}[1]{\ \left(\mathrm{mod}\ #1\right)}
%\usepackage{braket}

%\begin{document}
	
	\section{The meaning of Gauge}
	The Quantum Ising model has a \emph{global} symmetry $\tilde{S}_i = -S_i$, that leaves
	the form of the Hamiltonian invariant in the new variables, and so it's the same system.
	By \emph{global} I mean independent of $i$.
	The Quantum Ising Gauge model (see Hunter L.'s write up) has a \emph{local} symmetry.
	
	\section{Gauge and Matter in 1D}
	\subsection{Definitions of the full space}
	We consider again a 1D chain, put bosons of dimension $n$ on the links, with group $Z(n)$ and
	define $E$, $A$, $V$ and $U$ as before. We won't use $A$ and $V$ in the Hamiltonian.
	This time we add fermions on the sites, where at site $l$, the operator $\psi_\ell^\dagger$
	creates a fermion, and the operator $\psi_\ell$ destroys a fermion at that site.
	The fermion operators \emph{anti-commute} on \emph{different} sites.
	Then let
		\begin{equation}\label{eq:1dQEDLatticeHamiltonian}
	H=-\frac{n}{2\pi} \sum_\ell (\psi_\ell^\dagger U_{\ell,\ell+1}\psi_{\ell+1}+H.c.) + 2\tilde{m}\sum_\ell (-1)^\ell \psi_\ell^\dagger\psi_\ell+  \sum_{\ell} E_{\ell,\ell+1}^2,		
	\end{equation}
	where $n$ is an odd non-negative integer, and $m$ is a real number.
	We leave the boundary conditions open for now.
	The full space on $N$ sites has dimension $(2n)^{N}$, the $2^N$ comes from the fermions (electrons and positrons), 
	and the $n^N$ from the bosons (phonons). But this space isn't the physical space, and is too big, as we now discuss.
	
	\subsection{Gauss Law and Physical Space}
	Let the Gauss law be
		\begin{equation}\label{eq:1dQED_Gauss_law}
	G_\ell := \psi_\ell^\dagger \psi_\ell + \frac{1}{2}((-1)^\ell -1)-(E_{\ell,\ell+1}-E_{\ell-1,\ell}).
	\end{equation}
	We restrict the full space to the physical space of states $|\psi\rangle$ that satisfy
	$G_\ell|\psi\rangle= 0 \Mod{n}$. 
	The physical space has dimension $n 2^N$; see Hunter L's explanation for why this is the case.
	
	\subsection{Limits}
	Ercolessi et al (see the reference in Hunter's writeup) numerically take both limits \emph{first} the $N\rightarrow\infty$ limit
   and then the $n\rightarrow\infty$ limit and find a ``phase transition'' at some value of $\tilde{m}$. Order of the limits here FIXME TODO.
	For now, perhaps is enough to take $N\rightarrow\infty$ for $n$ finite, and prove that the limiting (topological, C*, Von Neuman??)
	algebra exists, and the limiting $H$ exits.
	To define a phase transition I need an ``order parameter'' which will be something like a magnetization in a spin model.
	Let $W=N^{-1}\sum_l E_{l, l+1}^2$ be the ``Wilson loop'' operator. Let $|gs\rangle$ be the ground state of $H$.
	Let $w_1(\tilde{m}, n, N) = \langle gs | W |gs\rangle$, and let $w(\tilde{m}, n)=\lim_{N\rightarrow\infty} w_1(\tilde{m}, n, N)$.
	 This quantity $w$ as a function of $\tilde{m}$ will have a first derivative everywhere,
	except for a value of $\tilde{m}$, where the left and right derivatives both exist but are different.
	We say that the model goes through a phase transition at that value of $\tilde{m}$. Obviously, that value will be dependent on $n$,
	and we are interested in $n\rightarrow\infty$.
	
%\end{document}
