\section{The Main Stage}
\subsection{Statement of the problem and ideas}
The current statement, I guess, is the following:

\begin{statement}
Consider $\mathbb Z^n$ where to each node we attach an infinite-dimensional separable Hilbert space $H$. Let $\Lambda \subseteq \mathbb Z^n$ be an infinite sublattice. Consider the limit
\[
\mathfrak A_{\Lambda} := \varinjlim_{X \subset \Lambda, \ \ X\ \text{finite}} \mathfrak A_{X}.
\]
If it turns out that $\mathfrak A_{\Lambda}$ is a $C^*$-algebra, I'd like to do the following: find an ideal $J \triangleleft \mathfrak A_{\Lambda}$ such that the Hilbert space $H$ promised by the Gelfand-Naymark theorem is separable; i.e., $\mathfrak A_{\Lambda}/J \cong \End(H)$ for $H$ separable. It would be also nice to keep embeddings $\mathfrak A_{X} \rightarrow \mathfrak A_{\Lambda}$ for finite sublattices $X \subset \Lambda$.
\end{statement}

\begin{fur}
Given a unital $C^*$-algebra $A$, under which conditions on $A$ the Hilbert space given by Gelfand-Naymark theorem is separable?
\end{fur}

\begin{idea}
To keep everything physically meaningful, I think that $J$ can be tried out as the ideal generated by operators with all but finitely many eigen-states concentrated in a finite sublattice of $\Lambda$. Here, I need to refresh my mind with regards to eigen-states.
\end{idea}

\subsection{Tests on quantum Ising model}
In Ising model, to each node of $\mathbb Z^2$ we attach a $2$-dimensional Hilbert space $H$ with some a priori chosen orthonormal basis $e_1,e_2$. It corresponds to the states \emph{spin up} and \emph{spin down}.
\subsubsection{The issue with the infinite tensor product}
Let's consider first the algebraic tensor product $H_{\infty}$ of all Hilbert spaces attached to all sites. It's spanned by simple tensors of the form
\[
e_{\lambda(1)} \otimes e_{\lambda(2)} \otimes e_{\lambda(3)} \otimes \cdots 
\]
where $\lambda : \mathbb N \rightarrow \{e_1,e_2\}$ is a function. There are as my such simple tensors as functions $\lambda$; the cardinal number is equal to $|2^{\mathbb N}| = |\mathbb R|$, i.e., there are uncountably many of them. It's natural to declare such simple tensors an orthonormal basis of $H_{\infty}$. But then, it's a result of metric spaces theory that if there are uncountably many points such that the distance between any of two is bounded by a positive constant (that doesn't depend on the points), then the space is not separable. This is the case with $H_{\infty}$. Since it's not separable, its completion $\cl H_{\infty}$ can't be separable as well. By the way, the same idea is used when one proves $l_{\infty}$ is not separable.

Just out of curiosity, the same cardinal number occurs when all Hilbert spaces are infinite-dimensional but separable. In this case, we deal with all functions $\lambda : \mathbb N \rightarrow \mathbb N$; the cardinal number of them is again ${|\mathbb N}^{\mathbb N}| = |\mathbb R|$.

%\subsubsection{The limit of algebras of observables in combination with Gelfand-Naymark theorem}
\subsubsection{The limit and G-F theorem}

For an infinite sublattice $\Lambda \subseteq \mathbb Z^n$, the injective limit $\varinjlim_{X \subset \Lambda, \ X \ \text{finite}} \mathfrak A_{X}$ can be thought of as the completion of the union of those subalgebras. 

The following example explains how the union works.
\begin{example}The algebra $\mathfrak A_{1}$ attached to a single site can be identified with the algebra of $2\times 2$ matrices over $\mathbb C$. For two nodes, the algebra $\mathfrak A_{2}$ can be identified with matrices of dimension $4 \times 4$. The embedding $\mathfrak A_{1} \rightarrow \mathfrak A_{2}$ then does the following:
\[
A:=\begin{pmatrix} a_{11} & a_{12}\\ a_{21} & a_{22}\end{pmatrix}
\mapsto A \otimes 1 = \begin{pmatrix} A & 0 \\ 0 & A\end{pmatrix}.
\]
\end{example}

\begin{fur}
I see so far two ways one can go here, I need to delve into both of them. One way leads to Hilbert-Schmidt operators (if we had a different norm at the begininning), the other (assuming the operator norm), I guess, leads to compact operators. I think we need to study both these cases. Reference \cite{conway} tells how to realize the first way (page 268).
\end{fur}

%
%Therefore, the union is made of infinite matrices with the property that each contains a submatrix of a finite dimension outside of which the matrix entries equal to zero.
%
%\begin{fur}
%I haven't understood yet how the completion of such a space looks like. In the process of looking this up and at the same time I'm pondering over this.
%\end{fur}