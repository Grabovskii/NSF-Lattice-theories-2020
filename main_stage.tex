\section{$C^*$-algebras stage}
\subsection{Statement of the problem and ideas}
The current statement, I guess, is the following:

\begin{statement}
Consider $\mathbb Z^n$ where to each node we attach an infinite-dimensional separable Hilbert space $H$. Let $\Lambda \subseteq \mathbb Z^n$ be an infinite sublattice. Consider the limit
\[
\mathfrak A_{\Lambda} := \varinjlim_{X \subset \Lambda, \ \ X\ \text{finite}} \mathfrak A_{X}.
\]
If it turns out that $\mathfrak A_{\Lambda}$ is a $C^*$-algebra, I'd like to do the following: find an ideal $J \triangleleft \mathfrak A_{\Lambda}$ such that the Hilbert space $H$ promised by the Gelfand-Naymark theorem is separable; i.e., $\mathfrak A_{\Lambda}/J \cong \End(H)$ for $H$ separable. It would be also nice to keep embeddings $\mathfrak A_{X} \rightarrow \mathfrak A_{\Lambda}$ for finite sublattices $X \subset \Lambda$.
\end{statement}

\begin{fur}
Given a unital $C^*$-algebra $A$, under which conditions on $A$ the Hilbert space given by Gelfand-Naymark theorem is separable?
\end{fur}

\begin{idea}
To keep everything physically meaningful, I think that $J$ can be tried out as the ideal generated by operators with all but finitely many eigen-states concentrated in a finite sublattice of $\Lambda$. Here, I need to refresh my mind with regards to eigen-states.
\end{idea}

\subsection{Tests on quantum Ising model}
In Ising model, to each node of $\mathbb Z^2$ we attach a $2$-dimensional Hilbert space $H$ with some a priori chosen orthonormal basis $e_1,e_2$. It corresponds to the states \emph{spin up} and \emph{spin down}.
\subsubsection{The issue with the infinite tensor product}
Let's consider first the algebraic tensor product $H_{\infty}$ of all Hilbert spaces attached to all sites. It's spanned by simple tensors of the form
\[
e_{\lambda(1)} \otimes e_{\lambda(2)} \otimes e_{\lambda(3)} \otimes \cdots 
\]
where $\lambda : \mathbb N \rightarrow \{e_1,e_2\}$ is a function. There are as my such simple tensors as functions $\lambda$; the cardinal number is equal to $|2^{\mathbb N}| = |\mathbb R|$, i.e., there are uncountably many of them\footnote{More details on why it's uncountable. Any number $a \in \mathbb R$ can be represented as a power series $a = \sum_{i=-\infty}^\infty c_i 2^{i}$, where $c_i \in \{0,1\}$ and only finitely many $c_i$ for $i > 0$ might be non-zero. Restrict ourselves to $a = \sum_{i=1}^{\infty}c_i 2^{-i}$. Then we have a 1-1 correspondence between functions $\lambda : \mathbb N \rightarrow \{0,1\}$ and such numbers. }. It's natural to declare such simple tensors an orthonormal basis of $H_{\infty}$. But then, it's a result of metric spaces theory that if there are uncountably many points such that the distance between any of two is bounded by a positive constant (that doesn't depend on the points), then the space is not separable. This is the case with $H_{\infty}$. Since it's not separable, its completion $\cl H_{\infty}$ can't be separable as well. By the way, the same idea is used when one proves $l_{\infty}$ is not separable.

Just out of curiosity, the same cardinal number occurs when all Hilbert spaces are infinite-dimensional but separable. In this case, we deal with all functions $\lambda : \mathbb N \rightarrow \mathbb N$; the cardinal number of them is again ${|\mathbb N}^{\mathbb N}| = |\mathbb R|$.

%\subsubsection{The limit of algebras of observables in combination with Gelfand-Naymark theorem}
\subsubsection{The limit and G-F theorem}

For an infinite sublattice $\Lambda \subseteq \mathbb Z^n$, the injective limit $\varinjlim_{X \subset \Lambda, \ X \ \text{finite}} \mathfrak A_{X}$ can be thought of as the completion of the union of those subalgebras. 

The following example explains how the union works.
\begin{example}The algebra $\mathfrak A_{1}$ attached to a single site can be identified with the algebra of $2\times 2$ matrices over $\mathbb C$. For two nodes, the algebra $\mathfrak A_{2}$ can be identified with matrices of dimension $4 \times 4$. The embedding $\mathfrak A_{1} \rightarrow \mathfrak A_{2}$ then does the following:
\[
A:=\begin{pmatrix} a_{11} & a_{12}\\ a_{21} & a_{22}\end{pmatrix}
\mapsto A \otimes 1 = \begin{pmatrix} A & 0 \\ 0 & A\end{pmatrix}.
\]
\end{example}

Therefore, for an inifnite sublattice $\Lambda$, the algebra $\mathfrak A_{\Lambda}$ can be identified with the space of matrices of infinite size such that only finitely many entries of each of them are non-zero. \emph{Note} the difference with the attempt to take an infinite tensor product of Hilbert spaces: these infinite matrices naturally act upon the space 
\[
T:=\bigoplus_{n=1}^\infty \, \bigotimes_{i \in A \subset \Lambda, \ |A| = n} H_i 
\]
It's easy to see that this space has an infinite countable basis. This implies that, whatever norm we put on $T$, the space will not be complete (that's a standard result from functional analysis: in a Banach space, a vector space basis is at least uncountable).

\textbf{Way 1 (just a fantasy)}. Let $\mathfrak A_{\Lambda}^\prime$ be the union of all $\mathfrak A_X$ for $X \subset \Lambda$ and $X$ finite. We can substitute the norm on $\mathfrak A_{\Lambda}^\prime$ with the Hilbert-Schmidt norm (see appendix), and then complete $\mathfrak A_{\Lambda}^\prime$ with respect to it. The elements $A$ of the resulting space can be represented as infinite matrices $(a_{nk})_{k,n = 1}^{\infty}$ such that $\sum_{n,k} |a_{nk}|^2 < \infty$. We can act with these on a completion of $T$. However, it's not a $C^*$-algebra; it is a Banach algebra though. There might be something in this approach.

\textbf{Way 2.} We can complete with respect to the operator norm. I can't prove this, but the evidence is that we obtain the space of compact operators on $\cl T$. That's very good. If our Hamiltonian is normalized in such a way that in the limit it gives a bounded operator, then we can employ the Hilbert-Schmidt theorem and find an orthonormal basis in $\cl T$ of eigen-values of the limiting Hamiltonian. 

%\begin{fur}
%I see so far two ways one can go here, I need to delve into both of them. One way leads to Hilbert-Schmidt operators (if we had a different norm at the begininning), the other (assuming the operator norm), I guess, leads to compact operators. I think we need to study both these cases. Reference \cite{conway} tells how to realize the first way (page 268).
%\end{fur}

%
%Therefore, the union is made of infinite matrices with the property that each contains a submatrix of a finite dimension outside of which the matrix entries equal to zero.
%
%\begin{fur}
%I haven't understood yet how the completion of such a space looks like. In the process of looking this up and at the same time I'm pondering over this.
%\end{fur}

\subsection{Thoughts on $C^*$-algebras approach}
So the idea was the following: since it's not sometimes clear what a limiting Hilbert space should look like, we can take the limit of the corresponding algebras of observables and then, by Gelfand-Naymark theorem, find an underlying Hilbert space, hopefully a separable one. But the dream will not come true:
\begin{statement}
Even in quantum Ising model, the Gelfand-Naymark representation from the proof of the theorem (see Appendix) yields a non-separable Hilbert space when corresponds to an infinite lattice.
\end{statement}
\begin{proof}[Evidence]
Let's have a more careful look at how the representation is constructed. Let $A$ be a unital $C^*$-algebra, let's say. If $A$ is taken as the $C^*$-algebra corresponding to an infinite lattice in quantum Ising model, then $A$ is an AF-algebra (approximately finite). In particular, it is separable and infinite-dimensional. Now, to construct the representation, for every non-zero $z \in A$ we pick a representation $\pi_z$ such that $\|\pi_z(z)\xi_z\| = \|z\|$, where $\xi_z$ is the cyclic vector of $\pi_z$, and then we take the direct sum of those. Clearly, the sum is uncountable, for as a set the algebra $A$ is uncountable, so a basis of the resulting Hilbert space cannot be countable.
\end{proof}
So, the proof of Gelfand-Naymark theorem, even though more or less constructive, does not yield a way to construct a separable Hilbert space. The algorithm might be polished, I guess. Which $z \in A$ we might restrict to? Which are sufficient? The problem is that the choice of $z$'s is \emph{set-theoretic}, it's not \emph{functionally-analytic}. 

The $C^*$-algebras approach also has a downside that we lose unbounded observables. For example, in free fermions on infinite chain, if we don't normalize the Hamiltonian, the limit results in an unbounded operator, so the limiting $C^*$-algebra doesn't capture this. We've tried to use different normalizations, but this yielded either the zero operator or the identity, something trivial. One might tweak the eigen-values so that their absolute values are less than $1$; but then, what's the meaning of the limiting observable?

\emph{To sum up, the downside:} the limiting $C^*$-algebra loses both separability of the Hilbert space and does not contain unbounded observables like the total energy of the lattice.

I'd like to mention Segal's article \cite{segal}. From what I understood from his article, it's not necessary to find a faithful representation of the whole limiting $C^*$-algebra. There's the following result, \emph{the upside of the $C^*$-algebras approach}, which is a corollary of a more general statement that can be found in the article:
\begin{statement}
Let $A$ be a $C^*$-algebra and $u \in A$ be self-adjoint.Then for any $\alpha$ from the spectrum of $u$, there exists an irreducible representation $\phi$ of $A$ and a non-zero element $x$ of the space on which $A$ is represented such that $\phi(u)x = \alpha x$.
\end{statement}
The result is great in a sense that even the continuous spectrum of a self-adjoint element can be realized (at least partly) as a point spectrum. This reminds of the rigged Hilbert space approach that also gives a way to realize the continuum spectrum as a point one (through generalized eigen-values).


\begin{fur}
Segal in \cite{segal} mentions that it's actually not an issue that unbounded operators don't land in the limiting $C^*$-algebras, for they can be treated in terms of bounded operators. What did he mean?
\end{fur}