%\documentclass{article}
%\newcommand{\Mod}[1]{\ \left(\mathrm{mod}\ #1\right)}
%\usepackage{braket}

%\begin{document}

 	 \section{Additional comments}
 	 \subsection{Commutators and Anti-commutators}
 	 Let $[A, B] \equiv AB - BA$.
 	 In the bosonic truncated example, prove (well, ``prove'' is too strong for this straightforward result, I mean, just note)
 	 that $[a, a^\dagger] \neq I$. This is way more general: No operators $A$ and $B$ over a \emph{finite} dimensional
 	 Hilbert space can satisfy $[A, B] = I$. (Hint: Use the trace and the circular property of the trace.)
 	 For the non-truncated (non-truncated above, it's always truncated below) infinite dimensional bosonic example,
 	 note that $[a, a^\dagger] = I$.
 	 Physics jargon: This commutator is the
 	 starting point for the  ``Heisenberg uncertainty principle'' and some models require it out of principle, hence its importance.
 
 	 Let $\{A, B\}=AB + BA$. This is physics notation and unfortunately clashes with the set $\{\}$ notation. Some use $[A, B]_+$ but I won't here,
 	 at least for now.
 	 In the fermionic examples, note that $\{c^\dagger_i, c_j\} = \delta_{i,j}I$,
 	 and that $\{c_i, c_j\} = 0$ for finite and infinite number of sites $N$.
 	 Remember that for $N$ sites, $c^\dagger_i$ actually acts on a $2^N$ dimensional space, and \emph{includes} the ``fermionic sign,'' that is the parity
 	 operator.  (I suppose it'd be too tedious to write $d_iP_i\equiv c_i$.)
 
 	 \subsection{Quantum Field Theory}
 	 You wanted me to explain what QFT means. This is going to be opinionated, so feel free to disagree.
 	 I'll give you two ``definitions'': one good that I like and one bad that can be found in some (bad) literature,
 	 where people try to ``patch'' it to make it sensible without success. As I'm biased toward the lattice, I'd prefer to work on the lattice definition.
 	 A QFT is the double infinity: that is, a model where we have infinite sites with an already infinite (separable) Hilbert space on each single site.
 	 For example, take the (already infinite and separable) bosonic example above with one site as defined, and add $N$ sites by tensor product, and take $N$ to infinity. You'll already get the desired commutation $[a_i^\dagger,a_j]=\delta_{i,j}I$, and $[a_i, a_j]=0$.
 	 Now for the bad definition: Take the bosonic example above with one site, and instead of adding sites, put a \emph{continuum} variable to each
 	 $a$ and $a^\dagger$ and make it $a(x)$, and define an algebra there. You want the following property: $[a(x)^\dagger, a(y)]=\delta(x-y)I$, where you
 	 have to define that ``Dirac delta $\delta$'' in some sort of Schwartz space of tempered distributions. You'll see that there's no way
 	 to have operators like that, I mean, that obey that commutation relation in the continuum. People have worked to salvage this definition, but nothing has worked in the continuum
 	 if there are interactions, that is, beyond the simplest quadratic Hamiltonian you can write.
 
 	 You can also ``define'' QFT in the Lagrangian formalism, with both good and bad definitions: that is, with lattice and continuum definitions.
 	 On the lattice, you'll have a countable and infinite number of integrals that you need to make sense of.
 	 On the continuum, you'll have a non-countable number of integrals to deal with, and make sense of it.
 	 According to A. Jaffe and E. Witten, (the guys that wrote the text for the Yang-Mills millenium problem), Balaban was the one that has had
 	 more success with proving existence of YM theories, (existence in lower dimensions than 3+1), and Balaban has worked with the good definition
 	 (the lattice one), albeit in the Lagrangian or analysis or partition function formalism, however you want to call it.
 
 	 \subsection{Hamiltonian versus Lagrangian}
 	 We'll use only the Hamiltonian approach and thus mostly algebra (plus topology) as opposed to analysis. But here I write this, just in
 	 case, we find that some things are better seen with analysis.
 	 How are the Hamiltonian (algebraic) and Lagrangian (analysis, partition function) approaches connected?
 	 Answer: By the transfer matrix method; see Hunter L.'s write up on this topic.
 	 Are the Hamiltonian and Lagrangian formulations of a given model equivalent?
 	 Short Answer: Yes. Longer answer: We would have to define ``equivalent'' something like ``yields the same physics''; we would also have to have a definition
 	 of the model in question in at least one approach (Hamiltonian or Lagrangian) and convert to the other using the Transfer Matrix.
 	 With all that said, the answer is yes.
 
 	 \subsection{Mapping from First Quantization to Second Quantization}
 	 We won't work in First Quantization but I can write this mapping for a general $H$ or at least for some examples if you are interested in this info.
 	 FIXME. TODO.
 
	\section{Pure Gauge in 1D}
    OK, here I'll borrow from Hunter L.'s 6.2. Lattice $Z(n)$-QED Model in 1-d.
	I want to do this thing first WITHOUT fermions so just pure gauge; and this may be too trivial given that we're in one dimension.
	As in the Hamiltonian formalism we don't include physical time this is pure QED in 1+1 (if you read the literature), but it needs
	to be qualified, actually \emph{doubly} qualified: on the lattice and with a $Z(n)$ group instead of $U(1)$ (we'll go to U(1), soon, though).

	But let's start with all finite, so we have two finite integers: $n$ in $Z(n)$ so our one-site Hilbert space has size $n$, and then we have
	$N$: the number of sites. Instead of putting our Hilbert space on the sites, we'll put them on the links.
	We assume periodic boundary conditions (PBC) (for now I think it's probably better than open boundary conditions), and so we
	have also $N$ links: there's a link connecting the last site with the first, and our lattice looks more like a ring than a line.
	Now, I need to define some operators. I will define operators on each link, and then think of them as acting on the $n^N$ Hilbert space
	by tensor product. If I look at the Hamiltonian of $Z(n)$ lattice QED in 1D in Eq.~(6.2) of Hunter's write up, I see that only the last
	term $E^2_{i,i+1}$ remains. So I have to define this so-called ``electric field'' operator. Here the subindex $i,i+1$ refers to the lattice link that connects sites
	$i$ with $i+1$; Let's rather write that as $E^2_{i,x}$ where $i,x$ says: the link that starts at site $i$ and goes in the forward $x$ direction.
	But for simplicity and because we're in 1D, I'll drop the $x$ from the subindex.

	Scaling the Hamiltonian will be very important here. While in the previous examples we had only one term, and so only an overall
	factor that would make the spectrum be in $[0,1]$ as opposed to $[0,\infty]$ with straightforward conversion between the two
	definitions, in the following examples, we'll have more than one term in $H$, and therefore, we would get unrelated results
	if the \emph{relative} scaling of the various Hamiltonian terms is different. I will then choose $E_i$ to be the one with \emph{tilde} in
	Hunter's writing. Moreover, I restrict this thing to $n$ odd. So here we go: Let $\{|0\rangle, |1\rangle,\cdots,|n-1\rangle\}$ with $n$ odd,
	be the basis of the Hilbert space on link $i,x$, we then define
	the non-negative integer $s$ such that $n=2s+1$, and the operator
	$E_i|k\rangle=(k-s)|k\rangle$, which is of course diagonal, with integer eigenvalues $-s$ to $s$ in steps of 1.
	We extend this definition by tensor product
	to the space of $N$ sites and dimension $n^N$. For example if $N=3$, and $|v\rangle=|k_0\rangle|k_1\rangle|k_2\rangle$,
	then $E_1|v\rangle=(k_1-s)|v\rangle$, and so $E_i$ is a square matrix with $n^N$ rows. Note that in this tensor product extension, there isn't a
	fermionic sign or Parity operator, because the $E$ operators are bosons and commute on different sites.

	We of course need a Hamiltonian here; Because we don't have fermions, just pure gauge, I'll define it as
	\begin{equation}
	H=\sum_{i=0}^{i<N} E_i^2,
	\end{equation} so that it's just another square matrix with $n^N$ rows, and note that I omitted the identities in the tensor product,
	because I would have to write $I\otimes I\otimes\cdots E_i^2 \otimes I \otimes I$ with $E_i^2$ at location $i$, instead of just $E_i^2$.
	Before finding eigenvalues and eigenvectors of $H$, we need to deal with the gauge symmetry.

	\subsection{Gauge Symmetry}
	Unlike in non-gauge models, here we need to weed out states that don't follow Gauss Law. So, we'll define the Gauss law operator, and
	consider its kernel as a (Hilbert) subspace of the $n^N$ dimensional Hilbert space considered before.
	Because this \emph{is} a subspace, things should be fine, and we should reconsider all operators defined before to be constrained to that space.
	This Hilbert subspace is going to be called ``physical'' as opposed to the original space that contains physical and unphysical states.
	We will then forget forever the original big space; I say big, because obviously the physical space will be smaller than the original.
	Let $G$ acting on $N$ sites be
	\begin{equation}
	G = \sum_{i=0}^{i=N-1} E_{i+1} - E_i,
	\end{equation}
	with $E_N\equiv E_0$.
	Physical jargon justification: Because we don't have fermions (electric charge in QED) then the divergence of the electric field
	over a closed loop should give zero. And because on a one dimensional lattice, the only loop seems to be the one going from beginning to end, we
	get the sum over all sites of the difference.

	The operator $G$ so defined is exactly the zero operator, so we don't have to restrict anything. Oops.

	\subsection{Algebra of Operators}
	Let me define another operator called $U_i|k\rangle=\ket{k+1 \Mod{n}}$ and I also define $U^\dagger_i|k\rangle=\ket{k - 1 \Mod{n}}$, so
	that these operators wrap around. Obviously this operator is extended to $N$ sites by tensor products, and is a bosonic operator,
	which means it commutes on different sites. On the same site, $U^\dagger U = U U^\dagger = I$ so that it is unitary.
	But note that if $n$ is finite, then (on the same site) $[E, U]\neq U$ due to the ``border'' basis states;
	note that it's not equal but ``almost'' equal, and by that I mean equal on states of the basis other than the border ones.
	Now, as for the physical jargon, we need an operator $A$, such that $[E, A] = iI$ where $i=\sqrt{-1}$ (same site here),
	 so that we need the ``counterpart'' magnetic field $A$
	to the electric field $E$. We already know (see previous bosonic example, cross ref. here FIXME TODO) that this $A$ doesn't
	exist if $n$ is finite. But it exists when $n$ is infinite, as we will see.
	But before that let me define the exponentiation of $E$ as $V=\exp(i \beta E)$ (one site only, and extend by tensor products), and
	I need to think about the $\beta$ variable, let's set it to one for now $\beta\equiv 1$.

	It may be better to work on just one site with $N=1$ for this. Prove that the $V$ above exists and is unique for $n$ finite.
	Prove that if $n$ is finite, then there exists a unique $A$ such that $U=\exp(i \alpha A)$, where I need to adjust the constant $\alpha$ which
	may depend on $n$, and let just set it to 1 $\alpha\equiv1$ for now. This is all on one fixed site.
	This is just saying that the exponential and logarithm of certain matrices exist.
	I guess I could put the exercise: Find the generators of the algebra that includes $U, V, E, A$,  and $H$ as defined above, \emph{and} the identity $I$.
	What's the dimension of this algebra. Note that $H$ is symmetric, and find its eigenvalues.

	Now we take the $n$ to infinity but keep $N$ finite. We can even take $N=1$ for simplicity.
	Prove that  $[E, A] = iI$ and that equivalently $[E, U]= U$ so that we have the QED ``actual'' commutations.
	I think we can say that $U$ is a representation of the infinite group $U(1)$, so we have even that.
	I guess we'd like to show that computations done for $n$ finite converge to this $n$ infinity case that is separable, and does
	represent QED in one spacial dimension (and is equivalent to QED 1+1 in Lagrangian formalism, where the last 1 is the time dimension).

	What happens if we take both limits $n$ to infinity but also $N$ to infinity? We may want to postpone this question for latter, not sure.

	I'm not sure if going to 2D now, which is the smallest dimension that has plaquettes... or I should perhaps introduce fermions to this 1D case.
	So the order of the below sections may change.


	\section{Pure Gauge in 2D}
	TBW
	\section{Gauge and Matter in 1D}
	TBW

%\end{document}
