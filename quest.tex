\section{Questions}
\subsection{For a discussion}

\begin{quest}
Continuous spectrum corresponds to scattering energies. What are those? How do they differ from energy levels?
\end{quest}
%
%\begin{quest}
%Inappropriate usage of continuous spectrum? There are no corresponding eigenvectors. If $T$ is a linear operator and $\lambda$ is in its continuous spectrum, then $T-\lambda$ is injective and has a dense range. Also, it seems like that think of eigenkets as belonging to $\Phi^*$, i.e. that they are linear functionals..
%\end{quest}

\begin{quest}
Rigged Hilbert spaces do not make sense to me. Let's say $(H,\Phi)$ is a rigged Hilbert space. First, the topology on $\Phi$ that they claim to be finer is actually the induced one from $H$. So, $\Phi$ is a pre-Hilbert space, and $H$ is its completion. Now, any continuous linear (or antilinear) functional on $\Phi$ is bounded; therefore, it uniquely extends to a functional on $H$, so $\Phi^*$ is naturally $H^*$. So, what's the point? 

Probably people do not describe it in the right way. I've not seens a description of the topology on $\Phi$. It can't be just stated as "the inclusion map is continuous".
\end{quest}


\subsection{Just for myself}
\begin{quest}
Derive the grand canonical Gibbs distribution to see why the sum is over $N$. It bothers me, I just can't stand this formula before I see its derivation.
\end{quest}
\begin{quest}
I do not see any issue with regards to defining the Gelfand spectrum for non-commutative Banach algebras. Why is it defined only for commutative ones? Only because it appears in the set-up of G-N theorem?
\end{quest}
\begin{quest}
Does the 2nd G-N theorem require a unit? I mean, if not, then something is still mapped to the unit of the operator algebra; that doesn't look natural, but might happen.
\end{quest}

In one article found this: the totality of numbers associated with a given observable is called its \emph{spectrum}. And indeed, if we regard the algebra of continuous functions as a Banach algebra, then the spectrum of anything is its image; in quantum mechanics, we indeed look at the eigenvalues, which from Banach algebras point of view form their spectra.

In Segal's article I saw that a product of two observables may not be an observable, so he proposed an algebraic structure where a multiplication is not assumed but only powers of elements.

Also from one article:
More recent studies have indicated that all self-adjoint operators may not be adequate as the model for the algebra of observables of every physical system. The $C*$-algebras are a long step from this special model, but still not into the chaos of abstract structures consistent with the general features of physical systems. 

From Kadison:
Given a state $f$ and an observable $A$, the value $f(A)$ is the expectation of the observable $A$ when these two have physical meanings.