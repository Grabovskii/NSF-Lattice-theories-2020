\section{Questions}
\subsection{For a discussion}

\subsection{Just for myself}

In one article found this: the totality of numbers associated with a given observable is called its \emph{spectrum}. And indeed, if we regard the algebra of continuous functions as a Banach algebra, then the spectrum of anything is its image; in quantum mechanics, we indeed look at the eigenvalues, which from Banach algebras point of view form their spectra.

In Segal's article I saw that a product of two observables may not be an observable, so he proposed an algebraic structure where a multiplication is not assumed but only powers of elements.

Also from one article:
More recent studies have indicated that all self-adjoint operators may not be adequate as the model for the algebra of observables of every physical system. The $C*$-algebras are a long step from this special model, but still not into the chaos of abstract structures consistent with the general features of physical systems. 

From Kadison:
Given a state $f$ and an observable $A$, the value $f(A)$ is the expectation of the observable $A$ when these two have physical meanings.

\subsection{Notes from conversations}
\subsubsection{07/08}
We might want:
\begin{enumerate}
\item Prove that the limits for the Wilson loop taken in different orders yield different results;
\item Is there a gap between the ground state and the first excited state in QED 1+1?
\item Look at the papers of people who tried to prove these statements in other models.
\end{enumerate}
In free fermions, the gap is $O(N^{-1})$. \emph{Spontaneous symmetry breaking} means that the ground state is \emph{degenerate} (i.e., the corresponding eigen-space is of dimension bigger than 1). One can indeed encounter this phenomenon in nature. There are arguments from thermodynamics (not rigorous) about phase transitions (the most challenging problems).