\section{Questions}

\begin{quest}
I'd like to discuss the derivation of the Gibbs measure a bit. So, we assume that the system exchanges its energy with some thermal reservoir, so the extra information we have is that $\left\langle \hh \right\rangle = U$ and that's why we get the Gibbs measure. But if the system is isolated, then $\hh \equiv U$. But in this case the measure we choose is "worse", it's the uniform distribution. Looks like the system becomes better, but the measure becomes worse. 
\end{quest}

\begin{quest}
Is it ok that in the grand canonical Gibbs distribution we sum over all possible $N$? I didn't attempt to derive the formula.
\end{quest}

\begin{quest}
In Ising model, since we have a partition function of the form $Z = \sum_{\omega \in \Omega} e^{-\beta \hh (\omega)}$, we assume that the model might exchange its energy with the environment. I guess that's because we can switch on an external magnetic field, which changes the internal energy.
\end{quest}

\begin{quest}
Would like to discuss the difference between classical and quantum lattices. In particular, the Ising model is classical, why? I guess it's so simple that there's no difference.
\end{quest}

\begin{quest}
The energy is lower when the spins are aligned. Would like to discuss.
\end{quest}

\begin{quest}
Just for myself, if a Hamiltonian is defined for a quantum lattice via so-called interaction, then they do not say how $\Phi(X)$ is related to $\Phi(Y)$ when $X \subseteq Y$. I guess we have to impose something on $\Phi$. 
\end{quest}

\begin{quest}
The difference between IRF and vertex models in Ising. Would like to discuss.
\end{quest}

\begin{quest}
I'd like to discuss the formalism for the long-range orientational order in $\text{O}(N)$-models. To be confident my guess is right. Maybe the spontaneous magnetization as well.
\end{quest}